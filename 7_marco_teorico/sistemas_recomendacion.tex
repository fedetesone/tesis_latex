\subsection{Sistemas de Recomendación}
\subsubsection{Contexto Histórico}
Es muy frecuente tener que tomar decisiones sin la suficiente experiencia personal sobre las alternativas disponibles. En la vida cotidiana, confiamos en recomendaciones de otras personas, ya sea de boca en boca o cartas de recomendación, reseñas de libros y películas, o encuestas generales. Los Sistemas de Recomendación o RS asisten este proceso natural en el ámbito de los sistemas de información \citep{resnick1997recommender}. El primer RS, Tapestry \citep{goldberg1992using}, fue un sistema experimental destinado a resolver el problema de manejar grandes cantidades de correos electrónicos filtrando según cuán interesantes eran los documentos, utilizando un enfoque \textit{basado en el contenido} de los mismos y también \textit{filtros colaborativos}. Se ha trabajado mucho en mejorar y desarrollar nuevos enfoques con respecto a los RS en los últimos años, y el interés en esta área sigue vigente debido a la abundancia de aplicaciones prácticas en las cuales es necesario ayudar a los usuarios a lidiar con la sobrecarga de información\footnote{El concepto de sobrecarga de información, del inglés information overload, hace referencia a cuando los usuarios reciben demasiada información, por lo cual, la precisión en sus decisiones empieza a decrecer \citep{eppler2004concept}.} y proveer para este fin recomendaciones personalizadas, contenidos, y servicios. Sin embargo, a pesar de todos estos avances, la generación actual de RS todavía requiere mejoras para que los métodos de recomendación sean más efectivos y aplicables a una gama más amplia de sistemas y/o sitios. Aunque las raíces de los RS se remontan a trabajos en ciencia cognitiva \citep{rich1979user}, teoría de aproximación \citep{powell1981approximation}, recuperación de información \citep{salton1989automatic}, ciencias de las predicciones \citep{armstrong2001principles}, ciencias de la gestión \citep{murthi2003role}, y también al modelado de la elección de consumidor en marketing \citep{lilien1992marketing}, los RS recién surgen como un área de investigación independiente en la década de 1990, cuando los investigadores comenzaron a centrarse en problemas de recomendación que se basan específicamente en \textit{calificaciones} \citep{adomavicius2005toward}. En su formulación más común, el problema de recomendación se reduce a estimar calificaciones para los ítems que no han sido vistos por un usuario.

\subsubsection{Funciones de un Sistema de Recomendación}
Como se mencionó anteriormente, un RS es un conjunto de herramientas de software que sugiere ítems a un usuario, quien posiblemente utilizará algunos de ellos. Haciendo énfasis particularmente en un RS comercial, probablemente la función más importante es incrementar el número de ítems vendidos, lo cual es posible porque el RS ofrecerá los ítems sobre los cuales el usuario tiene más probabilidades de deseo o necesidad. Esto implica aumentar el denominado \textit{ratio de conversión}, es decir, la cantidad de ventas que es posible efectuar sobre un ítem sobre el total de veces que un usuario selecciona el mismo. Por ejemplo, para un sitio de delivery de comida online, esta medida corresponde a cuántas veces un usuario realiza un pedido en un restaurante en particular, sobre el total de veces que observó el menú del mismo. Indudablemente, la conversión va a ser mayor si el usuario recibe recomendaciones de restaurantes que están más cercanos a su gusto personal. Otra función de un RS comercial complementaria y muy relacionada a la anterior es vender productos más diversos, ya que sería muy difícil para un usuario encontrarlos sin una recomendación precisa.

\bigskip Desde el punto de vista del usuario, un conjunto de recomendaciones precisas y relevantes aumentarán su satisfacción y fidelidad. El usuario disfrutará el uso de un sistema donde cada ítem o característica que utiliza está diseñada teniendo en cuenta sus intereses. Aumentar la fidelidad del usuario con el sitio significa que habrá mucha más interacción y, por lo tanto, el modelo de recomendación se volverá más refinado. Este conocimiento, a su vez, puede ser utilizado para mejorar otros sistemas y procesos relacionados, como el sistema de control de stock o la publicidad \citep{ricci2011introduction}.

\subsubsection{Técnicas de Recomendación}
Para implementar su función principal, un RS debe \textit{predecir} si vale la pena recomendar un ítem en particular. Para esto, este sistema debe ser capaz de predecir la utilidad de algunos de los ítems, o al menos poder comparar la utilidad entre algunos de ellos, y entonces decidir qué ítems recomendar basándose en esta comparación. Para realizar estas comparaciones, los RS basan sus estrategias de recomendaciones en 6 técnicas básicas \citep{ricci2011introduction}:

\paragraph{Basados en contenido}
Los RS basados en contenido intentan recomendar ítems similares a los que el usuario eligió anteriormente. Como su nombre lo indica, el proceso básico llevado a cabo por estos RS consiste en hacer coincidir atributos del perfil de usuario que posean preferencias e intereses en la búsqueda actual con los atributos del ítem que se va a recomendar \citep{lops2011content}. Este tipo de RS es especialmente útil cuando se conocen características de los ítems a recomendar pero no se conocen características del usuario. En otras palabras, estos sistemas intentan recomendar ítems similares a los que el usuario ha elegido anteriormente.

\bigskip Uno de los limitantes conocidos de estos RS es que recomiendan ítems del mismo tipo al que el usuario está solicitando. Por ejemplo, no sería posible recomendar música, utilizando videos ya que el perfil de contenido es distinto. Para solucionar esto, muchos de los RS basados en contenido están utilizando algoritmos híbridos con otro tipo de técnicas de recomendación.

\paragraph{Filtrado Colaborativo}
El Filtrado Colaborativo (Collaborative Filtering en inglés o CF) es el proceso de filtrado o evaluación de ítems usando las opiniones de los demás \citep{schafer2007collaborative}. El Filtrado Colaborativo es el enfoque original y el más simple de todas las técnicas de recomendación, y el más utilizado. Se basa en recomendar al usuario activo los ítems que otros usuarios con gustos similares eligieron en el pasado.

\paragraph{Demográficos}
La mayoría de los RS utilizan enfoques basados en conocimiento o en contenido. Esto implica que se necesita la suficiente información o un conocimiento adicional para poder llevar a cabo las recomendaciones. Los RS \textit{demográficos} hacen recomendaciones basadas en clases demográficas, por ejemplo, una ciudad en común. La ventaja es que la información histórica no es necesaria. Por ejemplo, una aplicación para este enfoque podría ser la de utilizar información demográfica para predecir el rating de distintos turistas a atracciones, basándose en enfoques predictivos de \textit{Machine Learning} \citep{wang2012applicability}. Las técnicas demográficas forman correlaciones “persona-a-persona”, como los sistemas colaborativos, pero utilizando una naturaleza distinta para los datos, en este caso, el perfil demográfico del usuario.

\paragraph{Basados en conocimiento}
Los RS \textit{basados en conocimiento} (Knowledge-based en inglés) usan el conocimiento acerca de los propios usuarios y los ítems a recomendar para generar la recomendación, razonando acerca de qué ítems satisfacen los requerimientos del usuario \citep{burke2000knowledge}.

\bigskip Los RS de filtrado colaborativo, al utilizar datos de otros usuarios, deben ser inicializados con un conjunto de datos considerablemente grande, ya que un sistema con una base de datos pequeña es improbable que sea útil. Además, la precisión del sistema es muy sensible al número de ítems asociados con un usuario dado \citep{shardanand1995social}. Esto conlleva a un problema de inicialización: hasta que no exista un número considerable de usuarios cuyas elecciones y hábitos sean conocidos, el sistema no será útil para un nuevo usuario. Lo mismo sucede para los RS que toman enfoques de Machine Learning. Típicamente, este tipo de sistemas se convierten en buenos clasificadores una vez que han aprendido desde una gran base de datos. Los RS basados en conocimientos evitan estas desventajas. No existe un problema de inicialización, ya que las recomendaciones no dependen de un conjunto de datos grande. Para este tipo de RS, no es necesario recolectar información acerca de un usuario en particular porque las recomendaciones que realizan son exclusivamente basadas en las elecciones de ese usuario. Estas características no solo hacen a este tipo de RS muy valioso en sí mismo, sino también como complemento de otros RS que utilicen distintas técnicas.

\paragraph{Basados en comunidades}
Un \textit{RS basado en comunidades} (community-based en inglés) hace uso del método “boca a boca” digital para construir una comunidad de individuos que comparten opiniones personales y experiencias relacionadas con sus recomendaciones de ítems. Estos sistemas presentan y agregan opiniones generadas por los usuarios en un formato organizado, las cuales son consultadas a la hora de tomar decisiones (por ejemplo, comprar un producto) \citep{chen2009community}. La evidencia sugiere que las personas están más inclinadas para seguir una sugerencia de sus amigos que una sugerencia similar que viene desde una persona anónima \citep{sinha2001comparing}. Este tipo de RS toma importancia cuando se tiene en cuenta la creciente popularidad de las redes sociales abiertas, tal que estos sistemas también son conocidos como \textit{Sistemas de Recomendación Sociales}.

\paragraph{Sistemas Híbridos}
Una variedad de técnicas fueron propuestas como base de los Sistemas de Recomendación. Cada una de ellas tiene desventajas conocidas, como el ya mencionado problema de inicialización de los sistemas colaborativos y basados en contenido. Un \textit{RS Híbrido} combina múltiples técnicas de recomendación para encontrar sinergia entre las mismas. Por ejemplo, un sistema basado en conocimiento puede compensar el problema de inicialización de los sistemas colaborativos para nuevos perfiles de usuario; así como también, el componente colaborativo puede utilizar sus habilidades estadísticas para encontrar pares de usuarios que compartan preferencias no esperadas, las cuales no podrían haber sido predichas por habilidades basadas en conocimiento \citep{burke2007hybrid}.