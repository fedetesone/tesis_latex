\subsection{Sitios de CQA}
Los servicios de Community Question Answering CQA, son un tipo especial de servicios \textit{Question Answering} (QA), los cuales permiten a los usuarios registrados responder a preguntas formuladas por otras personas. Los mismos atrajeron a un número creciente de usuarios en los últimos años \citep{li2010routing}. Una pregunta formulada en Quora, y respondida por su fundador y CEO, Adam D'Angelo, revela que el sitio recibe más de 200 millones de visitantes únicos mensualmente (información actualizada a Junio de 2017), lo que denota la popularidad de este tipo de portales\footnote{Pregunta formulada en el sitio Quora “How many people use Quora?”: \url{https://www.quora.com/How-many-people-use-Quora-3}. Último acceso: Febrero 2021.}. Desde la creación de este tipo de servicios, se han aplicado diferentes técnicas de software para que los usuarios encuentren respuestas a sus preguntas en el menor tiempo posible y aprovechar al máximo el valor de las bases de conocimiento, por ejemplo, un framework para predecir la calidad de las respuestas con características no textuales \citep{jeon2006framework}, incorporar información de legibilidad en el proceso de recomendación \citep{anuyah2017can}, encontrar a los expertos apropiados \citep{li2010routing} o recomendar la mejor respuesta a una pregunta dada, entre otros. Sin embargo, el mecanismo existente en el cual se responden las preguntas en los sitios de CQA todavía no alcanza a satisfacer las expectativas de los usuarios por varias razones: (1) baja probabilidad de encontrar al experto: una nueva pregunta, en muchos casos, puede no encontrar a la persona con la habilidad de responder de manera correcta, resultando en respuestas tardías y que distan de ser óptimas; (2) respuestas de baja calidad: los sitios de CQA suelen contener respuestas de baja calidad, maliciosas y spam. Estas suelen recibir baja calificación de los miembros de la comunidad; (3) preguntas archivadas y poco consultadas: muchas preguntas de los usuarios son similares. Antes de formular una pregunta, un usuario podría beneficiarse de buscar ya formuladas, y por consiguiente, sus respuestas \citep{yang2013cqarank}.
