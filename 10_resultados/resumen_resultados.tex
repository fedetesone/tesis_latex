\subsection{Resumen de resultados}
En el análisis del método EQuAL, mediante experimentación con distintos parámetros (tamaño de muestra y número de clusters), se obtuvo una media de error de \(0.312\) para el tamaño de muestra de \(100\) pares de preguntas y \(k = 50\). Estos resultados demuestran dos particularidades: el método EQuAL tuvo buen rendimiento con tamaños pequeños de muestras y con un alto número de clusters. Esta tendencia en tamaños de muestra chicos puede ser consecuencia del agregado de variabilidad que este método proporciona. Por otro lado, la tendencia a la baja de media de error es mucho más clara a medida que los valores de \(k\) se hacen más altos, obteniendo medias de error muy buenas cuando \(k > 25\) (\(0.309\), \(0.304\), \(0.307\), \(0.309\), \(0.311\), \(0.308\) para \(k = 25, 30, 35, 40, 45, 50\)). Por otro lado, comparando el método EQuAL con los algoritmos del estado del arte, se concluye que posee indicadores aptos para su aplicación en RS, en cuanto a medias de error y varianza. La media de error total del método EQuAL (teniendo en cuenta todas las ejecuciones realizadas en los experimentos) es de \(0.32286\), la cual no difiere demasiado de la mejor (bow con \(0.31162\)) y supera a FastText y TF-IDF con \(0.33124\) y \(0.32409\) respectivamente.

\bigskip Se realizó un Análisis de Varianza del método propuesto en contraste con los métodos del estado del arte para poder afirmar que el método EQuAL es apto para ser aplicado en RS, de forma eficiente y eficaz, de forma estadísticamente significativa. Los métodos analizados fueron bow, TF-IDF, FastText, Word2Vec y Semantic Distance y los tamaños de muestras tomados para el análisis fueron 100, 500, 1000, 1500 y 2000 pares de preguntas, completando así 25 intervalos de confianza. En la mayoría de los tamaños de muestra analizados, la media de error no presentó diferencias estadísticamente significativas en comparación con cada uno de los métodos del estado del arte, con excepción de la comparación EQuAL-bow para el tamaño de muestra de 2000 pares de preguntas, en el cual bow obtuvo una media de error significativamente más baja que el resto de los algoritmos.

\bigskip Adicionalmente, es posible afirmar como hallazgo que el método propuesto depende tanto de los algoritmos subyacentes como del conjunto de datos de entrada. Por un lado, es altamente probable que arroje buenos resultados si los algoritmos subyacentes también lo hacen, y viceversa. Por otro lado, es posible adaptar el método al conjunto de datos y elegir los algoritmos subyacentes adecuados para el mismo, dando versatilidad y un resultado compuesto por características aportadas por cada uno de ellos.

\bigskip Para finalizar, se desarrolló una arquitectura de software con enfoque Big Data que realiza los cálculos de similaridad y procesamiento del ensamble de clustering de manera adaptable. Esto significa que es posible ejecutar cálculos de matrices de similaridad con un número variable de técnicas de comparación de análisis de texto (es posible agregar o quitar una nueva técnica fácilmente) y luego ensamblar sus resultados. La arquitectura obtenida se ejecuta en forma distribuida y escalable horizontalmente, permitiendo incrementar el rendimiento de manera considerable, aplicando solamente un cambio de configuración de la infraestructura base (número de nodos en el cluster, memoria RAM asignada a cada ejecutor, almacenamiento, y/o tipo de servidores) y de la configuración Spark, sin tener que modificar el código en absoluto.

\bigskip De esta manera, los experimentos realizados lograron probar la factibilidad del método EQuAL presentado en esta Tesis, implementado exitosamente sobre una arquitectura Big Data.