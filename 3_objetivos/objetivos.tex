\newpage \section{Objetivos del trabajo de tesis}\label{ch:objetivos}
\subsection{Objetivo general}
\noindent El presente trabajo de investigación tiene como objetivo construir una arquitectura Big Data que se aplique a grandes conjuntos de datos de preguntas de CQA y permita encontrar nuevas medidas de similaridad entre textos que puedan ser utilizadas en sistemas de recomendación.
\subsection{Objetivos específicos}
\noindent Se detallan a continuación, los objetivos específicos que son necesarios para lograr el objetivo principal.
\begin{enumerate}
	\item Identificar medidas de similaridad de texto existentes y un método efectivo de aplicación de las mismas en grandes volúmenes de datos.
	\item Diseñar y desarrollar una arquitectura Big Data para cálculo de similaridad en grandes matrices, que requerirá nuevas estrategias para recolectar, procesar y manejar grandes volúmenes de datos.
	\item Encontrar nuevas medidas de similaridad de texto que sean mejores que las existentes respecto al manejo del volumen, variedad, velocidad y veracidad inherentes a grandes volúmenes de datos.
	\item Mejorar las medidas de desempeño y error en sistemas de recomendación del estado del arte.
\end{enumerate}