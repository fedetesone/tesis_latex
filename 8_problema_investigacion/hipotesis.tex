\subsection{Hipótesis de trabajo}
A partir del relevamiento del estado del arte se infiere que los algoritmos utilizados para generación de datos para sistemas de recomendación, podrían presentar márgenes de mejora en cuanto a medidas de rendimiento y a la eficiencia de sus implementaciones en arquitecturas Big Data, especialmente teniendo en cuenta grandes volúmenes de datos de entrada.

\bigskip Por tal motivo, y como respuesta a la hipótesis planteada, se presenta un desarrollo de un método basado en una arquitectura Big Data que pueda aplicarse a grandes conjuntos de datos con el fin de obtener un procedimiento eficiente y eficaz que aproveche las ventajas adimensionales y de variabilidad de datos inherentes del ensamble de clustering con el fin de validarlo y realizar un análisis comparativo con las medidas del estado del arte.