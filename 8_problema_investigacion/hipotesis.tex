\subsection{Hipótesis de trabajo}
A partir del relevamiento del estado del arte se establece la hipótesis de que los algoritmos utilizados para generación de datos para sistemas de recomendación presentarán márgenes apreciables de mejora en cuanto a medidas de rendimiento y a la eficiencia de sus implementaciones si se implementan en arquitecturas Big Data, especialmente teniendo en cuenta grandes volúmenes de datos de entrada.

\bigskip Por tal motivo, y como respuesta a la hipótesis planteada, se presenta un desarrollo de un método basado en una arquitectura Big Data que pueda aplicarse a grandes conjuntos de datos con el fin de obtener un procedimiento eficiente y eficaz que aproveche las ventajas adimensionales y de variabilidad de datos inherentes del ensamble de clustering con el fin de validarlo y realizar un análisis comparativo de las medidas de rendimiento implementadas en arquitecturas de Big Data con algoritmos del estado del arte.