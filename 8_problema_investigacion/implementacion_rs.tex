\subsection{Implementación en un sistema de recomendación de tiempo real}

Como punto de partida a esta sección, se aclara que este trabajo de tesis no cubre la implementación del RS propiamente dicho. No obstante, persigue como objetivo el desarrollo de un nuevo método para implementar la matriz de distancias asociada al RS, a partir de un gran volumen de datos de entrada, en particular en CQA, y utilizando una arquitectura Big Data. Para tal fin, se propondrá un ejemplo de RS a tiempo real, que dará contexto al trabajo en cuestión. El ejemplo será desarrollado teniendo en cuenta 3 casos de uso diferentes: \begin{enumerate*} [label=(\roman*)] \item un proceso batch ejecutado periódicamente que actualizará la base de datos, caso de uso central de este trabajo y en el cual se sustenta toda la arquitectura propuesta, y que también puede pensarse como una parte integrante de la implementación propuesta en este apartado; \item el punto de vista del usuario que consulta una pregunta en el sitio y \item el caso en que el usuario agrega una nueva pregunta al sitio.\end{enumerate*}

\subparagraph{El proceso completo del sistema de recomendación}
Un sistema de recomendación completo se puede analizar teniendo en cuenta 3 casos de uso, uno de los cuales es el planteado en esta Tesis.
\bigskip El primer caso de uso consiste en un proceso que se ejecuta fuera de línea, el mismo es el encargado de actualizar las relaciones de similaridad entre todas las preguntas existentes en el sitio, utilizando el método EQuAL para perseguir tal fin, y utilizando la arquitectura Big data planteada. Este es el caso analizado en esta Tesis. El segundo caso de uso consiste en la consulta de una pregunta en el sitio. Este caso se nutre del caso de uso anterior, y utiliza sus resultados para obtener una lista de preguntas similares a la consultada. Por último, si bien el procesamiento batch tiene en cuenta todas las preguntas existentes, no posee la capacidad de recomendar preguntas similares, en tiempo real, a una pregunta recién agregada al sistema. Por lo anterior, el tercer caso de uso se trata de una actualización, en la que un usuario agrega una nueva pregunta al sitio, y la resolución como implementación tecnológica para nutrir la base de datos original con nuevas preguntas y relaciones entre las mismas, en tiempo real.

\bigskip Se dará una propuesta de implementación de un RS completo, para dar contexto y tener una idea de funcionamiento en conjunto del mismo. En el mismo, se incluye como parte componente el método propuesto en esta Tesis y su arquitectura subyacente.

\subsubsection{Arquitectura general}
La arquitectura del RS embebido en un sitio de CQA colaborativo, consiste básicamente de un microservicio dedicado a recomendación, un proceso de \textit{streaming}\footnote{En computación, un \textit{stream} (en español, \textit{flujo}) es una secuencia de elementos de datos disponibles durante un periodo de tiempo.} que realizará la función de recomendación a tiempo real de forma distribuida utilizando el método EQuAL, una base de datos altamente escalable NoSQL, y un proceso batch que generará una actualización de la misma de forma periódica.
\begin{figure}
	\def\svgwidth{\linewidth}
	\input{imagenes/figura2.pdf_tex}
	\caption{Arquitectura de un sistema de recomendación a tiempo real utilizando el método EQuAL.}
\end{figure}

La Figura 8 muestra cómo interactúa la totalidad de componentes relativos al RS, divididos fundamentalmente en procesamiento a tiempo real y batch; y además, muestra todas las interacciones posibles entre los componentes. Más adelante, se explicaran cada uno de los 3 casos de uso más importantes e identificables dentro de este sistema.

\subsubsection{Procesamiento batch}
El procesamiento batch para actualización y carga inicial de la base de datos que contendrá la matriz de coasociación en un formato adecuado para consulta, se ejecuta periódicamente pero con una carga inicial del conjunto de datos Quora en una base de datos. A continuación se detallaran cada uno de los casos.
\begin{figure}
	\def\svgwidth{\linewidth}
	\input{imagenes/figura2.pdf_tex}
	\caption{Procesamiento batch de actualización de la base de datos de preguntas similares.}
\end{figure}

Como punto de partida y en solo una oportunidad, se carga una base de datos con el conjunto de datos Quora original. Esta base de datos podría ser relacional, o de otro tipo, pero debería tener dos características principales: rápida inserción de registros y debe permitir obtener todos los registros de una tabla mediante una sola consulta. La inserción rápida es importante en el caso del agregado de una nueva pregunta en tiempo real. La capacidad de seleccionar todos los registros de una tabla, tiene importancia para el proceso batch que utilizara el método EQuAL para la actualización periódica del RS.

\bigskip De forma periódica, proceso batch en el cual se basa este trabajo, obtendrá todas las preguntas de la base de datos de preguntas individuales para crear la matriz de coasociación. La periodicidad podrá variar dependiendo del número de clusters elegidos para el método de clustering y la arquitectura en la cual se ejecuta, mientras más rápido sea, más frecuente podría ser programado. Podría ejecutarse diariamente, en momentos de bajo tráfico en el sitio en cuestión.

\bigskip Se agrega un proceso ETL\footnote{Siglas para Extract, Transform and Load «extraer, transformar y cargar».} que tendrá como origen la matriz de coasociación y lo cargará en la base de datos que guardará la información de similaridad entre preguntas.  Las características que debe tener esta base de datos será descrita en el apartado siguiente, pero la clave principal debe ser el ID de una pregunta individual para poder ser consultada fácilmente. El proceso ETL podrá ser lanzado mediante un evento que indique que la nueva matriz de coasociación ha sido generada, o bien podría ser una extensión del proceso EQuAL. Un ejemplo de la entrada y salida de este procesamiento puede ser el que se muestra en la \textbf{Tabla 1}.

\begin{table}[]
	\centering
	\begin{tabular}{|c|c|c|c|c|}
		\hline
		\textbf{question\_id\_1} & \textbf{question\_id\_2} & \textbf{question\_1} & \textbf{question\_2} & \textbf{similarity} \\ \hline
		1 & 2 & question\_desc\_1 & question\_desc\_2 & 0.34 \\ \hline
		1 & 3 & question\_desc\_1 & question\_desc\_3 & 0.67 \\ \hline
		1 & 4 & question\_desc\_1 & question\_desc\_4 & 0.92 \\ \hline
	\end{tabular}
	\caption{Matriz de coasociación salida del proceso EQuAL.}
	\label{tab:table-co-asociation}
\end{table}

Luego del proceso ETL, los registros de la base de datos para el RS podría tener la siguiente forma:
