\subsection{Metodología de investigación}
Este trabajo comenzará con una búsqueda de material científico relacionado a RS en general, RS no personalizados basados en análisis de texto, su aplicación en sitios de CQA, un análisis de algoritmos de comparación de texto del estado del arte y su aplicación a grandes volúmenes de datos mediante métodos de ensamble de clustering y, también, una evaluación de arquitecturas de software adecuadas para un enfoque Big Data e infraestructuras acordes. Esto puede ser realizado mediante sitios o librerías digitales, tales como Google Scholar\footnote{Google Scholar: \url{https://scholar.google.com.ar}. Último acceso Febrero 2021.}, IEEExplore Digital Library\footnote{IEEExplore Digital Library: \url{http://ieeexplore.ieee.org}. Último acceso Febrero 2021.}, ScIELO\footnote{SciELO: \url{http://www.scielo.org}. Último acceso Febrero 2021.}, Harvard Library\footnote{Harvard library: \url{https://library.harvard.edu}. Último acceso Febrero 2021.} o el portal del CAICYT-CONICET\footnote{Centro Argentino de Información Científica y Tecnológica del CONICET: \url{http://www.caicyt-conicet.gov.ar/sitio}. Último acceso Agosto Febrero 2021.}, entre otros.

\bigskip Definida la hipótesis correctamente y el plan de trabajo, se iniciará el desarrollo de un software de código abierto partiendo del proyecto "text comparison"\footnote{Repositorio GitHub: \url{https://github.com/Departamento-Sistemas-UTNFRRO/text_comparison}.} perteneciente al repositorio Git del departamento de Ingeniería en Sistemas de Información de la UTN FRRo. Se importarán las piezas de software del código del proyecto del estado del arte recientemente mencionado para usarlas mediante un enfoque Big Data, con nuevas herramientas basadas en Cloud Computing, Hadoop y una arquitectura de software completamente nueva que optimice este tipo de desarrollo. Una vez que se inicie el desarrollo del proyecto, serán evaluadas distintas opciones de herramientas y entornos que se utilizarán, Esto incluye:

\begin{itemize}
	\item Lenguajes de programación y librerías inherentes al mismo.
	\item Almacenes de datos, frameworks y proyectos de terceros que puedan ser incorporados en la arquitectura Big Data.
	\item Arquitecturas de software, patrones, modelos y buenas prácticas.
	\item Infraestructura: local, distribuida en una red de computadoras físicas, o distribuida y virtualizada en la nube.
\end{itemize}

\bigskip Paralelamente al desarrollo, se identificará y documentará la nueva solución de acuerdo con los requerimientos de la Maestría en Ingeniería en Sistemas de Información, a fin de obtener un trabajo de investigación de tesis de maestría de excelencia, y acorde con los parámetros que caracterizan a la institución.

\bigskip Por último, una vez finalizado el desarrollo, se realizará un registro con los indicadores resultantes, se validará la propuesta, se explicitarán los resultados obtenidos y se elaborarán las conclusiones, a fin de abrir y/o profundizar en nuevas líneas de investigación.