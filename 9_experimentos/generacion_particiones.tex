\subsection{Generación de particiones}
\subsubsection{Cálculo de similaridades}
Se generaron particiones desde los subconjuntos de muestreos generados desde el conjunto de datos original. Teniendo en cuenta que se utilizó un algoritmo de clustering para desarrollar el método de este trabajo, fue necesario descomponer la estructura de los archivos de entrada ---el cual contiene un identificador por cada par de preguntas--- en preguntas individuales, con el fin de poder identificarlas unívocamente y utilizarlas como los objetos de entrada del análisis de clustering. La estructura de los subconjuntos de muestreo se clarifica en la Tabla \ref{tab:archivo-entrada}. Cada una de las filas contiene un identificador secuencial \(sequence\_id\), el identificador de par de preguntas proveniente del conjunto de datos original \(question\_pair\_id\), y las dos preguntas correspondientes \(question\_1\) y \(question\_2\).

\begin{table}[h!]
	\footnotesize
	\caption{Ejemplo de la estructura de los subconjuntos de muestreo.}
	\begin{tabularx}{\textwidth}{*{7}{>{\centering\arraybackslash}X}}
		\toprule
		\textbf{question\_pair\_id} & \textbf{question\_1} & \textbf{question\_2} \\
		\midrule
		123004                      & question\_0         & question\_2         \\
		98776                       & question\_1         & question\_3         \\
		\bottomrule
	\end{tabularx}
	\label{tab:archivo-entrada}
\end{table}

Luego, el conjunto el subconjunto de preguntas preparado para el cálculo de distancia fue generado de la siguiente forma:
\begin{enumerate}
	\item Se crea una matriz con la unión de las columnas \(question\_1\) y \(question\_2\), generando una fila por cada pregunta individual y un número secuencial que las identifica, como se muestra en la Tabla \ref{tab:preguntas-individuales}.
	\item Se realiza una combinación de cada una de las filas contra la matriz en sí misma (eliminando resultados repetidos).
\end{enumerate}

\begin{table}[h!]
	\footnotesize
	\caption{Ejemplo de la estructura del conjunto de preguntas individuales de la muestra en curso.}
	\begin{tabularx}{\textwidth}{*{7}{>{\centering\arraybackslash}X}}
		\toprule
		\textbf{sequence\_id} & \textbf{question} \\
		\midrule
		0                     & question\_0       \\
		1                     & question\_1       \\
		2                     & question\_2       \\
		3                     & question\_3       \\
		\bottomrule
	\end{tabularx}
	\label{tab:preguntas-individuales}
\end{table}

\begin{table}[h!]
	\footnotesize
	\caption{Combinación  de todas las preguntas individuales de una muestra.}
	\begin{tabularx}{\textwidth}{*{7}{>{\centering\arraybackslash}X}}
		\toprule
		\textbf{sequence\_id\_1} & \textbf{question\_id\_1} & \textbf{sequence\_id\_2} & \textbf{question\_id\_2} \\
		\midrule
		0 & question\_0 & 1 & question\_1 \\
		0 & question\_0 & 2 & question\_2 \\
		0 & question\_0 & 3 & question\_3 \\
		1 & question\_1 & 2 & question\_2 \\
		1 & question\_1 & 3 & question\_3 \\
		2 & question\_2 & 3 & question\_3 \\
		\bottomrule
	\end{tabularx}
	\label{tab:matriz-triangular}
\end{table}

\bigskip La Tabla \ref{tab:matriz-triangular} da un ejemplo de la combinación de todas las preguntas individuales de la muestra, utilizando el identificador de secuencia y el contenido de cada una de ellas. Esta estructura sirve de entrada para el cálculo de similaridad para cada una de las técnicas previamente mencionadas. Consecuentemente, es posible realizar un cálculo de similaridad entre las dos preguntas que pertenecen a una misma fila, y agregar esta información a la estructura de datos anterior. La Tabla \ref{tab:matriz-similaridad} posee la misma estructura que la Tabla \ref{tab:matriz-triangular}, con el agregado del valor de similaridad entre el par de preguntas de cada una de las filas, en la columna \(similarity\).

\bigskip
\begin{table}[h!]
	\footnotesize
	\caption{Ejemplo de la estructura de matriz de similaridad en formato de tabla.}
	\begin{tabularx}{\textwidth}{*{7}{>{\centering\arraybackslash}X}}
		\toprule
		\textbf{sequence\_id\_1} & \textbf{question\_id\_1} & \textbf{sequence\_id\_2} & \textbf{question\_id\_2} & \textbf{similarity} \\
		\midrule
		0 & question\_0 & 1 & question\_1 & similarity\_01 \\
		0 & question\_0 & 2 & question\_2 & similarity\_02 \\
		0 & question\_0 & 3 & question\_3 & similarity\_03 \\
		1 & question\_1 & 2 & question\_2 & similarity\_12 \\
		1 & question\_1 & 3 & question\_3 & similarity\_13 \\
		2 & question\_2 & 3 & question\_3 & similarity\_23 \\
		\bottomrule
	\end{tabularx}
	\label{tab:matriz-similaridad}
\end{table}

Si se piensa el subconjunto de datos de la Tabla \ref{tab:matriz-similaridad} en forma de matriz, en lugar de estructura de tabla, al calcular la distancia de cada una de las filas, se estaría formando una \textit{matriz triangular superior}, en la cual cada uno de los elementos es la distancia entre un par de preguntas:
\[\begin{bmatrix}0 & similarity\_01 & similarity\_02 & similarity\_03 \\ 0 & 0 & similarity\_12 & similarity\_13  \\ 0 & 0  & 0 & similarity\_23  \\ 0 & 0 & 0 & 0 \end{bmatrix}.\]
La estructura de tabla, en lugar de una estructura matricial como la anterior, es utilizada por cuestiones de tecnología. Al utilizarse Python, Apache Spark y el sistema de archivos, es posible generar archivos delimitados por comas (archivos CSV) utilizando el soporte nativo que poseen las tecnologías anteriormente mencionadas. Adicionalmente, esto posibilitó crear particiones de datos de manera sencilla y procesarlas de forma distribuida.

\subsubsection{Clustering y etiquetado}
Una vez que las similaridades son calculadas en forma distribuida los resultados se almacenan y son colectados para realizar un etiquetado a partir de un análisis de clustering.

\bigskip El algoritmo elegido fue Clustering de Partición Alrededor de Medoides (PAM) como forma de “etiquetar” cada una de las preguntas como pertenecientes a un determinado cluster, por cada una de las ejecuciones. El motivo de esta elección es porque los elementos no pueden representarse de forma directa en un espacio dimensional, dado que se comparan elementos de texto. Por esta razón, se usa un método que pueda recibir una matriz de similaridad como entrada, y los medoides son elementos reales que se encuentran en el conjunto de datos original. Se implementó de la forma que se describe a continuación.

\paragraph{Entrada y configuración inicial}
Como entrada fue utilizada la matriz de similaridad resultado del paso anterior, que sirve para obtener las distancias precalculadas entre cada uno de los elementos que participan en el proceso de clustering. Además, en otro arreglo en memoria, se utilizó el conjunto de preguntas individuales con un identificador secuencial, con el fin de facilitar el algoritmo, tal como se muestra en la Tabla \ref{tab:preguntas-individuales}.

\bigskip Por cada una de las corridas de clustering (parámetro de configuración) se proporciona un \(k\) inicial, que representa al número de clusters (o medoides) que define el algoritmo, y en los cuales las preguntas son distribuidas (proceso de etiquetado); y además, son identificadas con un identificador único universal (UUID\footnote{Identificador único universal \url{https://es.wikipedia.org/wiki/Identificador\_\%C3\%BAnico_universal}} por sus siglas en inglés) con el fin de poder recuperar cada uno de los resultados cuando se realice el ensamble de Clustering de cada una de estas ejecuciones.

\paragraph{Proceso de clustering}
Como se mencionó anteriormente, el algoritmo PAM tiene dos etapas, denominadas BUILD y SWAP. El algoritmo realiza una cantidad de iteraciones máxima (parámetro configurable), y por cada una de ellas se realiza el proceso descrito a continuación.

\subparagraph{Generación de etiquetas (BUILD)}
La generación de etiquetas significa asignar un medoide a una pregunta en particular, con el objetivo del armado de clusters. Como cada medoide es exclusivamente representativo de un cluster, la asignación de un medoide a una pregunta es equivalente a la asignación de la pregunta al cluster correspondiente.
\begin{enumerate}
	\item Se obtiene su similaridad con cada uno de los medoides. Para esto, se busca cada par de preguntas en una de las matrices de similaridad generadas por el paso anterior.
	\item Se asigna la pregunta al medoide (cluster) en la cual su similaridad es máxima.
	\item Se obtiene la suma de las similaridades totales por cada medoide, y se almacena en un arreglo que será utilizado en el siguiente paso.
	\item Se generan pares (ID pregunta, ID pregunta medoide) que representarán las etiquetas utilizadas por el algoritmo de ensamble.
\end{enumerate}

\subparagraph{Actualización de medoides (SWAP)}
Se busca actualizar los medoides con el fin de evaluar si, luego del cambio, se consiguen mejores resultados. Esto es, por cada uno de los clusters, se obtiene la similaridad de las preguntas todas contra todas, de la siguiente manera:

\begin{enumerate}
	\item Se toma una pregunta \(i\) de la lista, y se compara la similaridad con todas las preguntas restantes del cluster.
	\item Se suman todas las similaridades calculadas.
	\item Si esta suma es mayor a la suma de las similaridades que obtuvo el medoide actual en el paso anterior, la pregunta \(i\) pasa a ser el nuevo medoide.
	\item Se recalculan las etiquetas \((id\_pregunta, id\_pregunta\_medoide)\)
\end{enumerate}

La \textit{generación de etiquetas} y la \textit{actualización de medoides} se realizan de forma iterativa hasta que \begin{enumerate*} [label=(\roman*)] \item los medoides convergen o; \item se llega al límite máximo de iteraciones.\end{enumerate*} La convergencia de medoides significa que los medoides calculados por la iteración actual, son exactamente los mismos que la iteración anterior, lo que indica que el resultado es óptimo (dentro de los parámetros y las capacidades del algoritmo). Por otro lado, en caso de que no se haya conseguido la convergencia, el conjunto de clusters generado en la última iteración es el que se toma como válido.

\paragraph{Estructura de los resultados}
Los resultados son almacenados en un archivo CSV que posee la estructura de la Tabla \ref{tab:salida-clustering}. La primera columna es un UUID que identifica una ejecución en particular del algoritmo de clustering, la segunda columna es un identificador de pregunta individual, y la tercera columna es un identificador de pregunta real que actúa como medoide de un cluster.

\begin{table}[h!]
	\footnotesize
	\caption{Ejemplo de la estructura del resultado de la ejecución del algoritmo de clustering.}
	\begin{tabularx}{\textwidth}{ccc}
		\toprule
		\textbf{run\_uuid}                       & \textbf{question\_id} & \textbf{assigned\_medoid} \\
		\midrule
		63815467136575428551131593057064980770 & 336 & 856 \\
		63815467136575428551131593057064980770 & 342& 856 \\
		63815467136575428551131593057064980770 & 26 & 358 \\
		63815467136575428551131593057064980770 & 1364 & 437 \\
		\bottomrule
	\end{tabularx}
	\label{tab:salida-clustering}
\end{table}
Cada uno de los archivos almacena un UUID único (\(run\_uuid\)), es cual es idéntico en cada una de las filas, para facilitar el algoritmo de ensamble. Además, cada archivo contiene todas las preguntas individuales de la muestra en cuestión (\(question\_id\)), y el cluster a la cual pertenecen, el cual es representado por el ID de la pregunta que fue tomada como medoide para ese cluster (\(assigned\_medoid\)). Se generan tantos archivos por la cantidad de ejecuciones configuradas para cada una de las técnicas de similaridad del estado del arte. Por ejemplo, si utilizamos Word2Vec, TF, TFIDF, FastText y Semántica (5 técnicas) y se configuraron 100 ejecuciones por cada una de ellas, se obtendrán 500 archivos de etiquetas; los cuales serán la única entrada del algoritmo de ensamble de clustering.





