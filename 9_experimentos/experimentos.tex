\chapter*{Experimentos}\label{ch:experimentos}
\addcontentsline{toc}{chapter}{Capitulo 5. Experimentos}

\section*{}
\addtocounter{section}{1}
\setcounter{subsection}{0}

Los experimentos descritos en este capítulo se encuentran en los siguientes recursos:
\begin{enumerate}
	\item Repositorio GitHub \url{https://github.com/Departamento-Sistemas-UTNFRRO/big_data_text_similarity}, en la cuenta oficial de la UTN FRR, ya que servirá de aporte para el departamento de investigación. El código está basado en Python y es utilizado para la generación y validación de matrices de co-asociación utilizando el método EQuAL en modo fuera de línea.
	\item Repositorio GitHub \url{https://github.com/fedetesone/anova_equal}, perteneciente a la cuenta del tesista. El código está basado en R y es utilizado para la validación de rendimiento del método EQuAL.
	\item Los datos generados por los experimentos, que sirven como entrada para el análisis de validación realizado a continuación, están disponibles en Google Drive\footnote{Datos resultados de los experimentos en \url{https://drive.google.com/drive/folders/1oY8NP1k7y0Yw2ZoNUMq22SV716tgL6wi?usp=sharing}}.
\end{enumerate}

\subsection{Estado del arte}
El proyecto en el cual se realizaron los experimentos que este trabajo tiene como estado del arte, es un proyecto basado en código Python que se puede encontrar en el siguiente repositorio GitHub \url{https://github.com/Departamento-Sistemas-UTNFRRO/text_comparison}.

\bigskip Tiene como características principales la utilización de los 5 algoritmos de similaridad mencionados anteriormente (y evaluados a continuación) y la ejecución de cada uno de ellos en el marco de un patrón master-worker en un solo microprocesador. Este patrón es utilizado para el procesamiento paralelo, en el cual una tarea es enviada a cada uno de los “workers” para ser procesada. En este caso en particular, la cantidad de “workers” es fija y especificada como un parámetro en tiempo de ejecución.

\subsubsection{Análisis de rendimiento}
Se muestra al análisis de rendimiento para el cálculo de distancias con los el proyecto del estado del arte, con el fin de tener un punto comparativo para los experimentos que se realizarán con la nueva arquitectura. Las pruebas de los algoritmos de similaridad de rendimiento se realizaron con el siguiente equipo:

\begin{verbatim}
Modelo: MacBook Pro
Procesador: Intel Core i7
Velocidad del procesador: 2.6 GHz
Numero de nucleos: 6
Caché L2 (por núcleo): 256 KB
Caché L3: 12 MB
Tecnología Hyper-Threading: Habilitada
Memoria: 16 GB RAM
Almacenamiento: APPLE SSD AP0512M
\end{verbatim}

El cual también será utilizado para las pruebas de la nueva arquitectura. Se utilizará, como parámetro de rendimiento, el cálculo de similaridad para cada uno de los 404290 pares de preguntas. Luego de varias pruebas, se llega a una configuración de 8 hilos de ejecución y lotes de 10000 pares de preguntas, para obtener los tiempos más bajos posibles con cada uno de los algoritmos, los mismos fueron:

\subparagraph{Bag of Words}
\begin{verbatim}
[13:34:51] Starting script.
[13:34:51] Loading Quora questions file...
[13:34:52] ----- Run number 1 ------
[13:34:56] First 100000 distances calculated.
[13:34:58] First 200000 distances calculated.
[13:35:01] First 300000 distances calculated.
[13:35:03] First 400000 distances calculated.
[13:35:04] First 404290 distances calculated.
[13:35:04] Script finished. Total time: 0:00:12.895492
\end{verbatim}

\subparagraph{TF/IDF}
\begin{verbatim}
[13:31:05] Starting script.
[13:31:05] Loading Quora questions file...
[13:31:07] ----- Run number 1 ------
[13:31:07] Generating a sample of 0 questions.
[13:31:08] Training model...
[13:32:43] First 100000 distances calculated.
[13:33:20] First 200000 distances calculated.
[13:33:58] First 300000 distances calculated.
[13:34:38] First 400000 distances calculated.
[13:34:39] First 404290 distances calculated.
[13:34:39] Script finished. Total time: 0:03:33.665420
\end{verbatim}

\subparagraph{FastText}
\begin{verbatim}
[13:39:25] Starting script.
[13:39:25] Loading Quora questions file...
[13:39:26] ----- Run number 1 ------
[13:39:28] Training model...
Read 9M words
Number of words:  30823
Number of labels: 0
Progress: 100.0% words/sec/thread:  147553 lr:  0.000000 loss:  1.791601 ETA:   0h 0m
[13:40:29] First 100000 distances calculated.
[13:40:47] First 200000 distances calculated.
[13:41:06] First 300000 distances calculated.
[13:41:24] First 400000 distances calculated.
[13:41:25] First 404290 distances calculated.
[13:41:25] Script finished. Total time: 0:02:00.335008
\end{verbatim}

\subparagraph{Word2Vec}
\begin{verbatim}
[13:42:29] Starting script.
[13:42:29] Loading Quora questions file...
[13:42:30] ----- Run number 1 ------
[13:42:34] First 100000 distances calculated.
[13:42:36] First 200000 distances calculated.
[13:42:38] First 300000 distances calculated.
[13:42:39] First 400000 distances calculated.
[13:42:40] First 404290 distances calculated.
[13:42:40] Script finished. Total time: 0:00:10.719605
\end{verbatim}

\subparagraph{Semantic Distance}
\begin{verbatim}
[16:04:34] Starting script.
[16:04:34] Loading Quora questions file...
[16:04:35] ----- Run number 1 ------
[16:26:11] First 100000 distances calculated.
[16:47:20] First 200000 distances calculated.
[17:08:41] First 300000 distances calculated.
[17:30:08] First 400000 distances calculated.
[17:31:06] First 404290 distances calculated.
[17:31:06] Script finished. Total time: 1:26:32.941992
\end{verbatim}

\begin{table}[]
	\centering
	\begin{tabular}{|c|c|c|}
		\hline
		& \textbf{Tiempo total (segundos)} & \textbf{Velocidad aproximada (calc/seg)} \\ \hline
		\textbf{Bag of words} & 12.895492 & 31351.26601 \\ \hline
		\textbf{TF/IDF} & 213.66542 & 1892.163926 \\ \hline
		\textbf{FastText} & 120.335008  & 3359.703936 \\ \hline
		\textbf{Word2Vec}  & 10.719605 & 37715.00909 \\ \hline
		\textbf{Semantic Distance} & 5192.941992 & 77.85374853 \\ \hline
	\end{tabular}
	\caption{Análisis de rendimiento de los algoritmos de similaridad del estado del arte.}
	\label{tab:performance-estado-del-arte}
\end{table}

Los resultados en \textbf{Tabla \ref{tab:performance-estado-del-arte}}  muestran una clara ventaja en términos de rendimiento para el algoritmo de similaridad Word2Vec, inmediatamente seguido por Bag of Words. TF/IDF y FastText, muestran un rendimiento aceptable pero, en comparación, aproximadamente diez veces más lentos que los anteriormente mencionados. El rendimiento del algoritmo de similaridad Semantic Distance, es muy bajo y representa un gran problema al momento de analizar grandes conjuntos de datos, pero puede ser necesario utilizarlo debido a sus buenas medidas de desempeño y error y su análisis de similaridad desde una perspectiva distinta.

\subsubsection{Medidas de desempeño y error}
Utilizando cada para de preguntas del conjunto de datos original, se calculó la matriz de confusión para cada uno de los algoritmos de similaridad. Como se puede ver en la \textbf{Tabla \ref{tab:desempeno-estado-del-arte}}.

\begin{table}[]
	\small
	\centering
	\begin{tabular}{|c|c|c|c|c|c|c|}
		\hline
		\multicolumn{3}{|l|}{\multirow{2}{*}{}} &
		\multicolumn{2}{c|}{\textbf{Predicho}} &
		\multirow{2}{*}{\textbf{Precisión}} &
		\multirow{2}{*}{\textbf{Error}} \\ \cline{4-5}
		\multicolumn{3}{|l|}{} &
		\textbf{0} &
		\textbf{1} &
		&
		\\ \hline
		\multirow{2}{*}{\textbf{TF}} &
		\multirow{2}{*}{\textbf{Real}} &
		\textbf{0} &
		0.4355 &
		0.1953 &
		\multirow{2}{*}{0.6776} &
		\multirow{2}{*}{0.3224} \\ \cline{3-5}
		&
		&
		\textbf{1} &
		0.1271 &
		0.2421 &
		&
		\\ \hline
		\multirow{2}{*}{\textbf{TF/IDF}} &
		\multirow{2}{*}{\textbf{Real}} &
		\textbf{0} &
		0.4477 &
		0.1831 &
		\multirow{2}{*}{0.6685} &
		\multirow{2}{*}{0.3315} \\ \cline{3-5}
		&
		&
		\textbf{1} &
		0.1484 &
		0.2208 &
		&
		\\ \hline
		\multirow{2}{*}{\textbf{Word2Vec}} &
		\multirow{2}{*}{\textbf{Real}} &
		\textbf{0} &
		0.4343 &
		0.1965 &
		\multirow{2}{*}{0.6788} &
		\multirow{2}{*}{0.3212} \\ \cline{3-5}
		&
		&
		\textbf{1} &
		0.1247 &
		0.2445 &
		&
		\\ \hline
		\multirow{2}{*}{\textbf{FastText}} &
		\multirow{2}{*}{\textbf{Real}} &
		\textbf{0} &
		0.5033 &
		0.1275 &
		\multirow{2}{*}{0.6725} &
		\multirow{2}{*}{0.3275} \\ \cline{3-5}
		&
		&
		\textbf{1} &
		0.2 &
		0.1692 &
		&
		\\ \hline
		\multirow{2}{*}{\textbf{Semantic Distance}} &
		\multirow{2}{*}{\textbf{Real}} &
		\textbf{0} &
		0.4877 &
		0.1431 &
		\multirow{2}{*}{\textbf{0.6797}} &
		\multirow{2}{*}{\textbf{0.3203}} \\ \cline{3-5}
		&
		&
		\textbf{1} &
		0.1772 &
		0.192 &
		&
		\\ \hline
	\end{tabular}
	\caption{Matrices de confusión para los cinco algoritmos de medidas de similaridad.}
	\label{tab:desempeno-estado-del-arte}
\end{table}

\bigskip Los resultados porcentuales de las matrices de confusión se muestran en cada intersección de filas y columnas. Los resultados reales se muestran en las filas, siendo los dos posibles 0, cuando las preguntas son distintas y 1 cuando las preguntas son iguales. Los resultados predichos, se muestran en las columnas.
La precisión obtenida de las matrices de confusión, que se calcula como la suma de los resultados acertados entre los valores predichos y reales, en todos los casos exceden el \(66\%\), alcanzando un máximo de \(68\% \) para Word2Vec. Por otro lado, con respecto al error promedio, se llega a un valor máximo de \(33\%\) en el caso de TF/IDF.

\bigskip Observando las matrices de confusión, se presenta un desbalance en los dos valores tomados para calcular la precisión, siendo que los resultados obtenidos para la clase “0” considerablemente mejores que para la clase “1”. Por ejemplo, el desbalance más grande se encuentra en la matriz de confusión de FastText, el cual es \(0,504 – 0,170 = 0,334\). Este tipo de desbalance puede ser originado por la distribución del conjunto de datos de entrada (\(36,9\%\) pares de preguntas son clase 1 y el \(63,1\%\) restante es clase 0), el cual pretende ser corregido con un método de muestreo explicado más adelante.















\subsection{Preprocesamiento del conjunto de datos}

La calidad de un conjunto de datos del mundo real depende del número de errores que contenga. Los errores de entrada de datos, tanto simples como complejos, son muy frecuentes, más allá de las validaciones que se les hayan realizado a los mismos \citep{maletic2000data}. Además de los errores de datos, que es necesario eliminarlos, también es crucial preprocesar los mismos con el fin de estandarizar algunos factores y así obtener mejores resultados.

\bigskip Para el conjunto de datos de preguntas del sitio web Quora, se realizaron los siguientes trabajos de preprocesamiento:

\begin{enumerate}
	\item Convertir el texto en minúscula; esto habilita a que las comparaciones de texto entre preguntas con algoritmos que son sensibles al cambio entre letras mayúscula y minúscula, sean efectivas.
	\item Eliminar fórmulas; las cuales están encerradas entre etiquetas [math][/math] y [code][/code] ya que su análisis es muy complejo y contraproducente en términos de rendimiento.
	\item Reemplazar números por letras; posibilita poner en el mismo plano preguntas que contienen números y otras que no lo hacen. Además, al utilizar palabras del inglés es posible que las mismas se encuentren en taxonomías para comparaciones semánticas.
	\item Eliminar caracteres especiales, ya que los datos deben ser uniformes. Un signo de exclamación o de pregunta no cambiaría la semántica de la pregunta (desde la perspectiva del análisis de similaridad) y agregaría ruido al momento de procesarlas.
\end{enumerate}

Para resumir, cada una de las preguntas, será parámetro de una función que aplica las técnicas anteriormente mencionadas, tal como la que se muestra en el fragmento de codigo \ref{lst:funcion-limpieza}

\bigskip

\begin{python}[caption={Ejemplo de función de limpieza.}, captionpos=b,label={lst:funcion-limpieza}]
def clean_text(text):
	text = text.lower()
	text = remove_formulas(text)
	text = replace_numbers(text)
	text = remove_special_delimiters(text)

	return text
\end{python}

\bigskip Ejemplos de la salida de la función de limpieza, pueden ser:

\begin{verbatim}
i: How do I find the zeros of the polynomial function
[math]f(x)=\dfrac{1}{2}x^{3}-3x[/math]?
o: how do i find the zeros of the polynomial function
\end{verbatim}

\begin{verbatim}
i: Would you switch from Canon 6D to Leica D-LUX 109?
o: would you switch from canon six d to leica d-lux one zero nine
\end{verbatim}

En la primera, es posible ver cómo se elimina la fórmula polinómica, mientras que en la segunda, todos los números fueron transformados a letras, con espacios entre ellos. En ambas, se puede ver que el signo de interrogación se elimina y todas las letras están en minúscula.

\subsection{Muestreo del conjunto de datos}
\subsection{Generación de particiones}
\subsubsection{Cálculo de similaridades}
Se generan particiones desde los subconjuntos de muestreos generados desde el conjunto de datos original. Teniendo en cuenta que se va a utilizar un método de clustering para desarrollar el método de este trabajo, es necesario descomponer la estructura de los archivo de entrada, el cual contiene un identificador por par de preguntas, en preguntas individuales, con el fin de poder identificarlas unívocamente y poder utilizarlas como los objectos entrada del análisis de clustering. La estructura de los subconjuntos de muestreo se clarifican en la \textbf{Tabla \ref{tab:archivo-entrada}}.

\begin{table}[]
	\centering
	\begin{tabular}{|c|c|c|c|}
		\hline
		\textbf{sequence\_id} & \textbf{question\_pair\_id} & \textbf{question\_1} & \textbf{question\_2} \\ \hline
		0                     & 123004                      & question\_10         & question\_20         \\ \hline
		1                     & 98776                       & question\_11         & question\_21         \\ \hline
	\end{tabular}
	\caption{Ejemplo de la estructura de los subconjuntos de muestreo.}
	\label{tab:archivo-entrada}
\end{table}

Luego, el conjunto el subconjunto de preguntas preparado para el cálculo de distancia será generado de la siguiente forma:
\begin{enumerate}
	\item Se crea una matriz con la unión de las columnas \(question_1\) y \(question_2\), generando una fila por cada pregunta individual y un número secuencial que las identificará.
	\item Se realiza una combinación de cada una de las filas contra la matriz en sí misma (eliminando resultados repetidos), generando como resultado una estructura de matriz triangular como se muestra en la Tabla \ref{tab:matriz-triangular}.
\end{enumerate}

\begin{table}[]
	\centering
	\begin{tabular}{|c|c|c|c|}
		\hline
		\textbf{sequence\_id\_1} & \textbf{question\_id\_1} & \textbf{sequence\_id\_2} & \textbf{question\_id\_2} \\ \hline
		0 & question\_0 & 1 & question\_1 \\ \hline
		0 & question\_0 & 2 & question\_2 \\ \hline
		0 & question\_0 & 3 & question\_3 \\ \hline
		1 & question\_1 & 2 & question\_2 \\ \hline
	\end{tabular}
	\caption{Ejemplo de la estructura de matriz triangular en formato de tabla.}
	\label{tab:matriz-triangular}
\end{table}

\bigskip La estructura de matriz triangular sirve de entrada para el cálculo de similaridad por cada una de las técnicas previamente mencionadas. Por lo cual, es posible realizar un cálculo de similaridad entre las dos preguntas que pertenecen a una misma fila, y agregar esta información a la estructura de datos anterior, tal como se muestra en la Tabla \ref{tab:matriz-similaridad}.

\begin{table}[]
	\centering
	\begin{tabular}{|c|c|c|c|c|}
		\hline
		\textbf{sequence\_id\_1} & \textbf{question\_id\_1} & \textbf{sequence\_id\_2} & \textbf{question\_id\_2} & \textbf{similarity} \\ \hline
		0 & question\_0 & 1 & question\_1 & similarity\_01 \\ \hline
		0 & question\_0 & 2 & question\_2 & similarity\_02 \\ \hline
		0 & question\_0 & 3 & question\_3 & similarity\_03 \\ \hline
		1 & question\_1 & 2 & question\_2 & similarity\_12 \\ \hline
		1 & question\_1 & 3 & question\_3 & similarity\_13 \\ \hline
		2 & question\_2 & 3 & question\_3 & similarity\_23 \\ \hline
	\end{tabular}
	\caption{Ejemplo de la estructura de matriz de similaridad en formato de tabla.}
	\label{tab:matriz-similaridad}
\end{table}

\bigskip Si se piensa el subconjunto de datos anterior como en forma de matriz, en lugar de estructura de tabla, al calcular la distancia de cada una de las filas, se estaría formando una matriz triangular superior, en la cual cada uno de los elementos es la distancia entre un par de preguntas:

\[\begin{bmatrix}0 & similarity\_01 & similarity\_02 & similarity\_03 \\ 0 & 0 & similarity\_12 & similarity\_13  \\ 0 & 0  & 0 & similarity\_23  \\ 0 & 0 & 0 & 0 \end{bmatrix}\]

\bigskip La estructura de tabla, en lugar de una estructura matricial como la anterior, es utilizada por cuestiones de tecnología. Al utilizarse Python, Apache Spark y el sistema de archivos, es posible generar archivos delimitados por comas (archivos CSV) utilizando el soporte nativo que poseen las tecnologías anteriormente mencionadas. Adicionalmente, esto posibilita crear particiones de datos de manera sencilla y procesarlas de forma distribuida. Por ejemplo, si tuviésemos un archivo CSV con \(1.000.000\) (un millón) de filas y un cluster con \(8\) nodos ejecutores, cada uno procesaría \(125.000\) filas, es decir, \(1/8\) del conjunto de datos total.

\subsubsection{Clustering y etiquetado}
Una vez que las similaridades son calculadas en forma distribuida los resultados se almacenan y son colectados por el nodo maestro para realizar un etiquetado a partir de un análisis de clustering. Si bien, la implementación de un algoritmo de clustering completamente distribuido podría ser una mejora, sobre todo si se utiliza un algoritmo que requiere un gran costo computacional y es posible crear varias particiones de datos, se optó por un algoritmo relativamente simple que se ejecuta en un entorno multihilo en el nodo principal.

\bigskip El algoritmo elegido es Clustering de Partición Alrededor de Medoids (PAM) como forma de “etiquetar” cada una de las preguntas en un determinado cluster, por cada una de las ejecuciones. Se implementó de la forma que se describe a continuación.

\paragraph{Entrada y configuración inicial}
Como entrada es utilizada la matriz de similaridades que es resultado del paso anterior, y que servirán para obtener las distancias precalculadas entre cada uno de los elementos que participan en el proceso clustering. Además, en otro arreglo en memoria, el conjunto de preguntas individuales con un identificador secuencial, con el fin de facilitar el algoritmo, tal como se muestra en la \textbf{Tabla \ref{tab:preguntas-individuales}}.

\begin{table}[]
	\centering
	\begin{tabular}{|c|c|}
		\hline
		\textbf{sequence\_id} & \textbf{question} \\ \hline
		0                     & question\_0       \\ \hline
		1                     & question\_1       \\ \hline
		2                     & question\_2       \\ \hline
		3                     & question\_3       \\ \hline
	\end{tabular}
	\caption{Ejemplo de la estructura del conjunto de preguntas individuales de la muestra en curso.}
	\label{tab:preguntas-individuales}
\end{table}

\bigskip Por cada una de las corridas de clustering (parametro de configuracion) se proporcionará un \(k\) inicial, que representa al número de clusters (o medoides) que definirá el algoritmo, y en los cuales las preguntas serán distribuidas (proceso de etiquetado); y además, son identificadas con un identificador único universal (UUID\footnote{Identificador único universal \url{https://es.wikipedia.org/wiki/Identificador\_\%C3\%BAnico_universal}} por sus siglas en inglés) con el fin de poder recuperar cada uno de los resultados cuando se realice el ensamble de Clustering de cada una de estas ejecuciones.

\paragraph{Proceso de clustering}
Como se mencionó anteriormente, el algoritmo PAM tiene dos etapas, denominadas BUILD y SWAP. Por un motivos de rendimiento y simplicidad, los experimentos son realizados solo utilizando la etapa BUILD para construir cada uno de los clusters. El algoritmo realiza una cantidad de iteraciones máxima (parámetro configurable), y por cada una de ellas se realiza el proceso descrito a continuación.

\subparagraph{Generación de etiquetas}
La generación de etiquetas significa asignar un medoide a una pregunta en particular, con el objetivo del armado de clusters.
\begin{enumerate}
	\item Se busca su similaridad con cada uno de los medoides.
	\item Se asigna la pregunta al medoide (cluster) en la cual su similaridad es máxima.
	\item Se obtiene la suma de las similaridades totales por cada medoide, y se almacena en un arreglo que será utilizado en el siguiente paso.
	\item Se generan pares (ID pregunta, ID pregunta medoide) que representarán las etiquetas utilizadas por el algoritmo de ensamble.
\end{enumerate}

\subparagraph{Actualización de medoides}
Se busca actualizar los medoides con el fin de evaluar si, luego del cambio, se consiguen mejores resultados. Esto es, por cada uno de los clusters, se obtiene la similaridad de las preguntas todas contra todas, de la siguiente manera:

\begin{enumerate}
	\item Se toma una pregunta \textit{i} de la lista, y se compara la similaridad con todas las preguntas restantes del cluster.
	\item Se suman todas las similaridades calculadas.
	\item Si esta suma es mayor a la suma de las similaridades que obtuvo el medoide actual en el paso anterior, la pregunta \textit{i} pasa a ser el nuevo medoide.
	\item Se recalculan las etiquetas \((id\_pregunta, id\_pregunta\_medoide)\)
\end{enumerate}

La \textit{generación de etiquetas} y la \textit{actualización de medoides} se realiza de forma iterativa hasta que \begin{enumerate*} [label=(\roman*)] \item los medoides convergen o; \item se llega al límite máximo de iteraciones.\end{enumerate*} La convergencia de medoides significa que los medoides calculados por la iteración actual, son exactamente los mismos que la iteración anterior, lo que indica que el resultado es óptimo (dentro de los parámetros y las capacidades del algoritmo). Por otro lado, en caso de que no se haya conseguido la convergencia, el último conjunto de clusters generado el que se tomará como válido.

\paragraph{Estructura de los resultados}
Los resultados son almacenados en un archivo CSV que posee la estructura de la \textbf{Tabla \ref{tab:salida-clustering}}.
\begin{table}[]
	\centering
	\begin{tabular}{|c|c|c|}
		\hline
		\textbf{run\_id}                       & \textbf{question\_id} & \textbf{assigned\_medoid} \\ \hline
		63815467136575428551131593057064980770 & 336                   & 856                       \\ \hline
		63815467136575428551131593057064980770 & 342                   & 856                       \\ \hline
		63815467136575428551131593057064980770 & 26                    & 358                       \\ \hline
		63815467136575428551131593057064980770 & 1364                  & 437                       \\ \hline
	\end{tabular}
	\caption{Ejemplo de la estructura del resultado de la ejecución del algoritmo de clustering.}
	\label{tab:salida-clustering}
\end{table}
Cada uno de los archivos almacena un UUID único (identico en cada una de las filas, para facilitar el algoritmo de ensamble), todas las preguntas individuales del muestreo en cuestión, y el cluster a la cual pertenecen, el cual es representado por el ID de la pregunta que fue tomada como medoide para ese cluster. Se generan tantos archivos por la cantidad de ejecuciones configuradas para cada una de las técnicas de similaridad del estado del arte. Por ejemplo, si utilizamos Word2Vec, TF, TFIDF, FastText y Semántica (5 técnicas) y se configuraron 100 ejecuciones por cada una de ellas, se obtendrán 500 archivos de etiquetas; los cuales serán la única entrada del algoritmo de ensamble de clustering.






\subsection{Ensamble de clustering}
\subsection{Método de validación}\label{metodo-validacion}

\subsubsection{Generación de conjuntos estadísticamente significativos}

Para generar resultados estadísticamente significativos se ejecutó el proceso completo de modo iterativo, variando dos parámetros principales: \begin{enumerate*} [label=(\roman*)] \item el tamaño de la muestra y \item el número de clusters \(k\)\end{enumerate*}. Como experimentos para este trabajo se realizaron ejecuciones con conjuntos de datos aleatorios de 100, 500, 1000, 1500 y 2000 pares de preguntas (200, 1000, 2000, 3000 y 4000 preguntas individuales). Para cada tamaño de muestra, se realizaron 10 muestras aleatorias manteniendo un \(k\) fijo. Por ejemplo, para un número de clusters \(k = 5\) y para un tamaño de muestra 100, se realizaron 10 ejecuciones con un conjunto de datos de entrada aleatorio. Dando un total de \(5 \times 10 = 50\) matrices de co-asociación resultado, para un \(k\) fijo.

\paragraph{Elección de los tamaños de muestra}
EEn cuanto a la elección del tamaño de muestra en el apartado anterior, se tomó en cuenta que los conjuntos de datos sean lo suficientemente grandes como para generar resultados estadísticamente significativos, pero con un tamaño apropiado para la ejecución de los experimentos de forma local, en favor de la facilidad que para la depuración de los mismos. Cuando se aumenta el tamaño de la muestra en forma lineal, la cantidad de cómputos por cada uno de los algoritmos de similaridad aumentan de forma cuadrática. Como se mencionó anteriormente, Siendo \(n\) el número de pares de preguntas de una muestra, se realizarán \(2n^2-n\) cálculos. Además, si, por ejemplo, utilizamos 5 algoritmos de similaridad el número de cálculos de similaridad es de \(5(2n^2-n)\), con el agregado de que al algoritmos de clustering PAM y el ensamble de clustering también aumentan su complejidad de una forma considerable, dependiendo de nuestro número de clusters \(k\).

\paragraph{Elección del número de clusters}
La elección del número de clusters \(k\) en los experimentos realizados sigue una lógica que busca estandarizar a los mismos a lo largo de todos los tamaños de muestras, es decir, no variar el número de clusters entre ellas, con fines de comparación. Como regla general (\textit{rule of thumb} o \textit{regla del pulgar}) se considera como número ``óptimo'' de clusters un valor de alrededor de \(\sqrt{n/2}\) \citep{kodinariya2013review}, siendo \(n\) el tamaño de la muestra. Para el número de muestras elegido (200, 1000, 2000, 3000 y 4000 preguntas individuales), los valores siguiendo esta regla serían \(k = 10\), \(k = 22\), \(k = 31\), \(k = 38\), \(k =44\).

\bigskip Los valores generados por la regla del pulgar se encuentran en un rango \([10, 44]\), por lo cual, finalmente se optó por elegir los valores de \(k\) con una separación uniforme de los mismos, para facilitar su interpretación y visualización, respetando ese rango de cobertura. Los experimentos para este trabajo se realizaron con los valores \(k = 5, 10, 15, 20, 25, 30, 35, 40, 45, 50\).

\bigskip La validación de los valores de \(k\) elegidos para realizar los experimentos serán validados en conjunto mediante el rendimiento de la matriz de co-asociación. Partiendo de la base del \textit{método del codo} para la evaluación de la performance de un algoritmo de clustering, el cual evalúa el porcentaje de variabilidad (suma de cuadrados de distancias) explicada en función del número de clusters, mediante la idea de encontrar el número mínimo de clusters por el cual agregando un cluster adicional no modelaría mejor los datos. El porcentaje de variabilidad explicado por los clusters es graficado contra el número de clusters. Los primeros clusters agregaran una información considerable al modelo, pero en cierto punto la ganancia marginal caerá dramáticamente, dando un ángulo en el gráfico \citep{bholowalia2014ebk}. La Figura \ref{fig:codo} muestra un gráfico bidimensional en el cual se compara el número de clusters (en abscisas) y la suma de cuadrados de distancias (en ordenadas) como medida de variabilidad de los clusters. En este gráfico se puede ver como luego de \(k = 5\), la suma de cuadrados de distancias varía muy poco, ya que los valores esbozan una curva con valores aproximadamente constantes luego de este punto.

\begin{filecontents*}{codo.csv}
1,400
2,200
3,130
4,80
5,40
6,25
7,24
8,23
9,22
10,21
11,20
12,19
\end{filecontents*}

\begin{figure}
	\centering
	\scriptsize
	\resizebox{\textwidth}{!}{%
		\begin{tikzpicture}
			\begin{axis}[
				xlabel={Número de clusters (k)},
				ylabel={Suma de cuadrados de distancias},
				xmin=0, xmax=12,
				ymin=0, ymax=420,
				xtick={1,2,...,12},
				ytick={0,50,...,400},
				legend pos=north west,
				ymajorgrids=true,
				grid style=dashed,
				]

				\addplot table [mark=square,x index=0, y index=1, col sep=comma] {codo.csv};
				\label{codo}

				\draw [dashed] (50,0) -- (50,400);
			\end{axis}
		\end{tikzpicture}
	}
	\caption{Ejemplo de gráfico para método del codo. Valor óptimo \(k = 5\).}
	\label{fig:codo}
\end{figure}

\bigskip Con el fin de dar una aplicación más integral al método del codo, en lugar de calcularlo por cada una de las técnicas de clustering aplicadas, el mismo se aplicará utilizando la matriz de coasociación generada a partir del ensamble. Para cuantificar y evaluar el performance, se utilizarán matrices de confusión denotarán el error entre los resultados obtenidos y la clasificación proveniente del conjunto de datos original, por cada uno de los valores de k anteriormente mencionados.

\subsubsection{Estructura de las matrices de confusión}\label{estructurasconfusion}
Una matriz de confusión, es una matriz que muestra clasificaciones predichas y reales. Una matriz de confusión puede ser de tamaño \(L \times L\) donde \(L\) es el número de diferentes valores de etiqueta o clase \citep{provost1998glossary}. En este trabajo, las clases son: 0 (las preguntas comparadas no son iguales) y 1 (las preguntas son iguales). Por lo cual, la matriz de confusión que se deriva, se muestra en la Tabla \ref{tab:matriz-confusion}. Los valores “a” y “d” representan el porcentaje de veces que los valores predichos fueron iguales a los reales, es decir, los casos en que las preguntas fueron predichas, respectivamente, iguales o distintas y realmente lo eran. Por otro lado, “c” es la proporción de preguntas predichas distintas, que en realidad son iguales; y “d” la proporción de preguntas predichas iguales, pero son realmente distintas. El objetivo de un algoritmo con buen desempeño entonces, es maximizar “a” y “d” y, por lo tanto, minimizar “b” y “c”.

\bigskip
\begin{table}[h!]
	\footnotesize
	\centering
	\caption{Matriz de confusión para validación de resultados.}
	\begin{tabularx}{0.35\textwidth}{*{7}{>{\centering\arraybackslash}X}}
		\toprule
		\multicolumn{2}{l}{\multirow{2}{*}{}} & \multicolumn{2}{c}{\textbf{Predicho}}                             \\ \cmidrule(l){3-4}
		\multicolumn{2}{l}{}                  & \multicolumn{1}{c}{\textbf{0}} & \multicolumn{1}{c}{\textbf{1}} \\ \midrule
		\multicolumn{1}{c}{\multirow{2}{*}{\textbf{Real}}} & \multicolumn{1}{c}{\textbf{0}} & \multicolumn{1}{c}{a} & \multicolumn{1}{c}{b} \\ \cmidrule(l){2-4}
		\multicolumn{1}{c}{}  & \textbf{1}  & c                               & d                               \\ \bottomrule
	\end{tabularx}
	\label{tab:matriz-confusion}
\end{table}

Los indicadores de desempeño que se evaluarán en los experimentos realizados con fines comparativos, son los siguientes:
\begin{itemize}
	\item \textbf{Exactitud:} \((a+d)/(a+b+c+d)\)
	\item \textbf{Error:} \((b+c)/(a+b+c+d)\)
	\item \textbf{Precisión positivos:} \(d/(d + b)\)
	\item \textbf{Precisión negativos:} \(a/(a+c)\)
	\item \textbf{Sensitividad:} \(d/(c+d)\)
	\item \textbf{Especificidad:} \(a/(a+b)\)
\end{itemize}
donde \(a+b+c+d=1\).

\paragraph{Preparación de los datos}
Para evaluar el rendimiento del algoritmo de ensamble, se toma la matriz de coasociación generada como resultado del proceso total y la muestra de pares de preguntas que se utilizó como entrada para ese proceso. Por ejemplo, se considera la muestra de preguntas de la Tabla \ref{tab:muestra-validacion}, la cual muestra dos pares de preguntas (\(123004\) y \(98776\) - 4 preguntas en total) con su identificador de par y un indicador, en la columna \(equal\), que tiene dos valores posibles: 1 (preguntas iguales) y 0 (preguntas distintas). Por otro lado, en la Tabla \ref{tab:coasociacion-validacion}, se muestra la matriz de coasociación generada, por el método EQuAL, a partir de la Tabla \ref{tab:muestra-validacion}, teniendo en cuenta todas las combinaciones tomadas de a 2 de las 4 preguntas originales, para las cuales se agrega la similaridad entre pares, en la columna similarity. Por último, se filtran solo los pares de preguntas de la matriz de coasociación que se encuentran en la Tabla \ref{tab:muestra-validacion}, ya que son las únicas con la cual su similaridad puede compararse con fines de validación, es decir, el conjunto de datos mostrado en la Tabla \ref{tab:filtrado-validacion}. Lo que nos deja con un conjunto de pares de preguntas que es posible comparar en su totalidad con la muestra original.

\begin{table}[h!]
	\footnotesize
	\centering
	\caption{Muestras de pares de preguntas que se utilizó como entrada del método EQuAL.}
	\begin{tabularx}{0.8\textwidth}{*{7}{>{\centering\arraybackslash}c}}
		\toprule
		\textbf{sequence\_id} & \textbf{question\_pair\_id} & \textbf{question\_1} & \textbf{question\_2} & \textbf{equal} \\
		\midrule
		0                     & 123004                      & question\_10         & question\_20         & 1              \\
		1                     & 98776                       & question\_11         & question\_21         & 0              \\
		\bottomrule
	\end{tabularx}
	\label{tab:muestra-validacion}
\end{table}

\begin{table}[h!]
	\footnotesize
	\caption{Matriz de co-asociación generada a partir de la muestra de la Tabla \ref{tab:muestra-validacion}.}
	\begin{tabularx}{\textwidth}{*{7}{>{\centering\arraybackslash}X}}
		\toprule
		\textbf{question\_id\_1} & \textbf{question\_id\_2} & \textbf{question\_1} & \textbf{question\_2} & \textbf{similarity} \\
		\midrule
		question\_10 & question\_11 & contenido & contenido & 0.857 \\
		question\_10 & question\_20 & contenido & contenido & 0.210 \\
		question\_10 & question\_21 & contenido & contenido & 0.126 \\
		question\_11 & question\_20 & contenido & contenido & 0.006 \\
		question\_11 & question\_21 & contenido & contenido & 0.368 \\
		question\_20 & question\_21 & contenido & contenido & 0.146 \\
		\bottomrule
	\end{tabularx}
	\label{tab:coasociacion-validacion}
\end{table}

\begin{table}[h!]
	\footnotesize
	\caption{Filtrado de la Tabla \ref{tab:coasociacion-validacion} con los pares de preguntas que se encuentran en la Tabla \ref{tab:muestra-validacion}.}
	\begin{tabularx}{\textwidth}{*{7}{>{\centering\arraybackslash}X}}
		\toprule
		\textbf{question\_id\_1} & \textbf{question\_id\_2} & \textbf{question\_1} & \textbf{question\_2} & \textbf{similarity} \\
		\midrule
		question\_10             & question\_11             & contenido            & contenido            & 0.857               \\
		question\_11             & question\_21             & contenido            & contenido            & 0.368               \\
		\bottomrule
	\end{tabularx}
	\label{tab:filtrado-validacion}
\end{table}

Ya se realizaron los cálculos de similaridad, los algoritmos de clustering y la matriz de co-asociación. En la siguiente sección, se procederá a interpretar los resultados obtenidos.

\paragraph{Construcción y elección del umbral correcto}
La problemática que se intenta resolver teniendo en cuenta todas las similaridades obtenidas a partir del ensamble de clustering es ¿Cuando consideramos a esas preguntas iguales y cuando no? La respuesta es simple, cuando la similaridad \(S\) entre un par de preguntas (\(q_1,q_2)\) es igual o superior a cierto umbral \(t\) se considera que son iguales (1) y distintas si sucede lo contrario (0), clarificando:
\[f(x) = \left\{ \begin{array}{lcc} 1 & si & S(q_1, q_2)\geq t
	\\ 0 & si & S(q_1, q_2) < t
\end{array} \right.\]

La elección del mejor umbral, se realiza eligiendo valores en el intervalo \((0,1)\) y evaluando cual de ellos conlleva a un mejor desempeño, es decir, que los valores calculados a partir del umbral coincidan, en una mayor medida, con el valor real proveniente de la muestra de datos. Por ejemplo, tomando valores con intervalos \(0.05\) formando un arreglo como \([0.05, 0.1, 0.15, ..., 0.90, 0.95]\), por cada uno de ellos:
\begin{enumerate}
	\item Se iteran todos los pares de preguntas.
	\item Por cada uno de los valores de similaridad, se asigna \(1\) si son mayores o iguales al umbral, \(0\) si pasa lo contrario.
	\item Si los valores asignados en el paso anterior coinciden con el valor real, se asigna un valor \textit{true} (verdadero), si no coinciden, se asigna \textit{false} (falso).
	\item Se calcula la proporción de pares de preguntas asignadas con true, es decir, que el valor real coincide con el predicho, y se obtiene la \textit{exactitud} del método.
\end{enumerate}
El valor de umbral que arroje la mayor exactitud, será el utilizado para evaluar el desempeño del método, al construir la matriz de confusión.

\bigskip Volviendo al ejemplo anterior, un umbral elegido de \(0.65\) aplicado a la Tabla \ref{tab:filtrado-validacion}, arrojaría el resultado de la Tabla \ref{tab:umbral-validacion-1} (el único par de preguntas que presenta una similaridad mayor al umbral es \((question\_10, question\_11)\), por lo cual se le asigna el valor 1). La Tabla \ref{tab:umbral-validacion-1}, muestra un resultado idéntico al conjunto de entrada, es decir, a la Tabla \ref{tab:muestra-validacion}. Lo anterior significa que los pares predichos son iguales a los pares reales. En otro caso, si el umbral que tiene mejor rendimiento fuese \(0.9\), el resultado obtenido luego del cómputo hubiese sido el de la Tabla \ref{tab:umbral-validacion-2}, el cual expone una diferencia entre el valor real y el predicho del par de preguntas \((question\_10, question\_11)\). Lo anterior denota el mayor rendimiento del umbral \(0.65\) contra el umbral \(0.9\) y su importancia en la elección del valor correcto.

\bigskip

\begin{table}[h!]
	\footnotesize
	\caption{Asignación binaria de los resultados de similaridad obtenidos en la Tabla \ref{tab:filtrado-validacion}, teniendo en cuenta un umbral de \(0.65\).}
	\begin{tabularx}{\textwidth}{*{7}{>{\centering\arraybackslash}X}}
		\toprule
		\textbf{question\_id\_1} & \textbf{question\_id\_2} & \textbf{question\_1} & \textbf{question\_2} & \textbf{equal} \\
		\midrule
		question\_10             & question\_11             & contenido            & contenido            & 1              \\
		question\_11             & question\_21             & contenido            & contenido            & 0              \\
		\bottomrule
	\end{tabularx}
	\label{tab:umbral-validacion-1}
\end{table}


\begin{table}[h!]
	\footnotesize
	\caption{Asignación binaria de los resultados de similaridad obtenidos en la Tabla \ref{tab:filtrado-validacion}, teniendo en cuenta un umbral de \(0.9\).}
	\begin{tabularx}{\textwidth}{*{7}{>{\centering\arraybackslash}X}}
		\toprule
		\textbf{question\_id\_1} & \textbf{question\_id\_2} & \textbf{question\_1} & \textbf{question\_2} & \textbf{equal} \\
		\midrule
		question\_10             & question\_11             & contenido            & contenido            & 0              \\
		question\_11             & question\_21             & contenido            & contenido            & 0              \\
		\bottomrule
	\end{tabularx}
	\label{tab:umbral-validacion-2}
\end{table}

\bigskip En conclusión, el rendimiento del algoritmo se medirá comparando la variable de clase (1 o 0) del conjunto de datos de entrada, con la variable de clase construida desde la comparación de las similaridades resultados del método y un umbral apropiado. Cuanto más pares de preguntas coincidan, mejor será el rendimiento del algoritmo, el cual se podrá visualizar mediante matrices de confusión.

\paragraph{Construcción de las matrices de confusión}
Con el fin de poder ilustrar cómo se construyen las matrices de confusión a partir de la comparación de los conjuntos de datos, se utilizará el siguiente ejemplo. Supongamos que el conjunto de datos de entrada es el que se muestra en la Tabla \ref{tab:validacion-reales}, Y el resultado obtenido luego de la elección del mejor umbral, es la Tabla \ref{tab:validacion-predichos}. Por lo cual, comparando los valores de la columna “equal", obtendremos el resultado de la Tabla \ref{tab:validacion-comparacion}. Sumarizando los resultados, la matriz de confusión derivada se muestra en la Tabla \ref{tab:validacion-confusion-ejemplo}. La exactitud obtenida a partir de este conjunto de datos y la ejecución hipotética del método es \(0.75\) (y por lo tanto, el error es \(0.25\)).

\begin{table}[h!]
	\footnotesize
	\centering
	\caption{Ejemplo de conjunto de datos de entrada (reales) para validación.}
	\begin{tabularx}{0.8\textwidth}{*{7}{>{\centering\arraybackslash}c}}
		\toprule
		\textbf{sequence\_id} & \textbf{question\_pair\_id} & \textbf{question\_1} & \textbf{question\_2} & \textbf{equal} \\
		\midrule
		0 & 123004 & question\_10 & question\_20 & 1 \\
		1 & 98776  & question\_11 & question\_21 & 1 \\
		2 & 14422  & question\_12 & question\_22 & 1 \\
		3 & 12321  & question\_13 & question\_23 & 1 \\
		4 & 999    & question\_14 & question\_24 & 0 \\
		5 & 7448   & question\_15 & question\_25 & 0 \\
		6 & 69553  & question\_16 & question\_26 & 0 \\
		7 & 2447   & question\_17 & question\_27 & 1 \\
		\bottomrule
	\end{tabularx}
	\label{tab:validacion-reales}
\end{table}

\begin{table}[h!]
	\footnotesize
	\centering
	\caption{Ejemplo de conjunto de datos de predichos por el método EQuAL.}
	\begin{tabularx}{0.8\textwidth}{*{7}{>{\centering\arraybackslash}c}}
		\toprule
		\textbf{question\_id\_1} & \textbf{question\_id\_2} & \textbf{question\_1} & \textbf{question\_2} & \textbf{equal} \\
		\midrule
		question\_10 & question\_20 & contenido & contenido & 1 \\
		question\_11 & question\_21 & contenido & contenido & 1 \\
		question\_12 & question\_22 & contenido & contenido & 0 \\
		question\_13 & question\_23 & contenido & contenido & 1 \\
		question\_14 & question\_24 & contenido & contenido & 1 \\
		question\_15 & question\_25 & contenido & contenido & 0 \\
		question\_16 & question\_26 & contenido & contenido & 0 \\
		question\_17 & question\_27 & contenido & contenido & 1 \\
		\bottomrule
	\end{tabularx}
	\label{tab:validacion-predichos}
\end{table}

\begin{table}[h!]
	\footnotesize
	\centering
	\caption{Resultado de comparación de las tablas \ref{tab:validacion-reales} y \ref{tab:validacion-predichos} para validación y construcción de matrices de confusión.}
	\begin{tabularx}{0.6\textwidth}{*{7}{>{\centering\arraybackslash}c}}
		\toprule
		\textbf{sequence\_id} & \textbf{real} & \textbf{predicho} & \textbf{resultado} & \textbf{equal} \\
		\midrule
		0 & 1 & 1 & true  & 1 \\
		1 & 1 & 1 & true  & 1 \\
		2 & 1 & 0 & false & 0 \\
		3 & 1 & 1 & true  & 1 \\
		4 & 0 & 1 & false & 1 \\
		5 & 0 & 0 & true  & 0 \\
		6 & 0 & 0 & true  & 0 \\
		7 & 1 & 1 & true  & 1 \\
		\bottomrule
	\end{tabularx}
	\label{tab:validacion-comparacion}
\end{table}

\begin{table}[h!]
	\footnotesize
	\centering
	\caption{Matriz de confusión obtenida a partir de la comparación de las tablas \ref{tab:validacion-reales} y \ref{tab:validacion-predichos}.}
	\begin{tabularx}{0.35\textwidth}{*{7}{>{\centering\arraybackslash}X}}
		\toprule
		\multicolumn{2}{l}{\multirow{2}{*}{}} & \multicolumn{2}{c}{\textbf{Predicho}}                             \\ \cmidrule(l){3-4}
		\multicolumn{2}{l}{}                  & \multicolumn{1}{c}{\textbf{0}} & \multicolumn{1}{c}{\textbf{1}} \\ \midrule
		\multicolumn{1}{c}{\multirow{2}{*}{\textbf{Real}}} & \multicolumn{1}{c}{\textbf{0}} & \multicolumn{1}{c}{0.25} & \multicolumn{1}{c}{0.125} \\ \cmidrule(l){2-4}
		\multicolumn{1}{c}{}  & \textbf{1}  & 0.125                               & 0.5                               \\ \bottomrule
	\end{tabularx}
	\label{tab:validacion-confusion-ejemplo}
\end{table}
