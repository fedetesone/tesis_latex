\subsection{Ensamble de Clustering}
El proceso de ensamble comienza obteniendo todos los archivos de etiquetas generados por el paso anterior. El resultado es una colección Spark, con la misma estructura que los archivos físicos, pero residente en memoria y en todos los nodos del cluster Hadoop, con el fin de poder realizar el procesamiento de una forma eficiente y escalable.

\bigskip El conjunto de etiquetas es agrupado por ID de pregunta, y se aplica una función de grupo que obtendrá una lista de tuplas o parejas \((run\_uuid, cluster\_id)\). Con el fin de clarificar esta idea, se va a dar un ejemplo: Si se realizan 3 ejecuciones del algoritmo de clustering, cada una de ellas es representada como un archivo en formato CSV y procesada de manera distribuida. Cada uno de esos archivos generados para cada ejecución, contiene información del UUID de ejecución y la asignación de cada una de las preguntas individuales a un cluster resultado. En el caso de PAM, cada identificador de cluster es un identificador real de una pregunta individual (o medoide desde la perspectiva algorítmica). En caso de que se realicen 3 ejecuciones, a modo de ejemplo, los archivos de etiqueta generados pueden representarse como se muestra en las Tablas \ref{tab:run1}, \ref{tab:run2} y \ref{tab:run3}, con identificadores de ejecución \(run\_uuid\_1\), \(run\_uuid\_2\) y \(run\_uuid\_3\), respectivamente.

\begin{table}[h!]
	\footnotesize
	\caption{Ejemplo de asignación de clusters a preguntas individuales para la ejecución 1.}
	\begin{tabularx}{\textwidth}{*{7}{>{\centering\arraybackslash}X}}
		\toprule
		\textbf{run\_uuid} & \textbf{question\_id} & \textbf{cluster\_id} \\
		\midrule
		run\_uuid\_1       & 1                     & 1                    \\
		run\_uuid\_1       & 2                     & 1                    \\
		run\_uuid\_1       & 3                     & 1                    \\
		run\_uuid\_1       & 4                     & 4                    \\
		\bottomrule
	\end{tabularx}
	\label{tab:run1}
\end{table}

\begin{table}[h!]
	\footnotesize
	\caption{Ejemplo de asignación de clusters a preguntas individuales para la ejecución 2.}
	\begin{tabularx}{\textwidth}{*{7}{>{\centering\arraybackslash}X}}
		\toprule
		\textbf{run\_uuid} & \textbf{question\_id} & \textbf{cluster\_id} \\
		\midrule
		run\_uuid\_2       & 1                     & 1                    \\
		run\_uuid\_2       & 2                     & 2                    \\
		run\_uuid\_2       & 3                     & 1                    \\
		run\_uuid\_2       & 4                     & 2                    \\
		\bottomrule
	\end{tabularx}
	\label{tab:run2}
\end{table}

\begin{table}[h!]
	\footnotesize
	\caption{Ejemplo de asignación de clusters a preguntas individuales para la ejecución 3.}
	\begin{tabularx}{\textwidth}{*{7}{>{\centering\arraybackslash}X}}
		\toprule
		\textbf{run\_uuid} & \textbf{question\_id} & \textbf{cluster\_id} \\
		\midrule
		run\_uuid\_3       & 1                     & 3                    \\
		run\_uuid\_3       & 2                     & 2                    \\
		run\_uuid\_3       & 3                     & 3                    \\
		run\_uuid\_3       & 4                     & 2                    \\
		\bottomrule
	\end{tabularx}
	\label{tab:run3}
\end{table}

Considerando \(run\_uuid\_1\), se puede observar que las preguntas 1, 2 y 3 fueron asignadas al cluster representado por la pregunta 1. En cambio, la pregunta 4 fue asignada a un cluster distinto, representado por ella misma. La intuición para esta ejecución en particular, es que las preguntas 1, 2 y 3 son “más similares” entre sí, que la pregunta 4.

\bigskip En el siguiente paso, el conjunto de datos es agrupado por \(question\_id\), y se aplica una función de grupo que genera tuplas \((run\_uuid, cluster\_id)\). Para el ejemplo anterior resultaría tal como se muestra en la Tabla \ref{tab:tuplas}. La columna \(tuples\) contiene un arreglo de tuplas que indica, por cada pregunta individual, el conjunto de ocurrencias ejecución-cluster que fueron generadas a lo largo del experimento.

\begin{table}[h!]
	\footnotesize
	\caption{Conjunto de datos de asignación de clusters agrupados por pregunta individual para generación de tuplas \((run\_uuid, cluster\_id)\) .}
	\begin{tabularx}{\textwidth}{>{\centering\arraybackslash}p{2.0cm}>{\centering\arraybackslash}p{15cm}}
		\toprule
		\textbf{question\_id} & \textbf{tuples}                                          \\
		\midrule
		1                     & {[}(run\_uuid\_1,1),(run\_uuid\_2,1),(run\_uuid\_3,3){]} \\
		2                     & {[}(run\_uuid\_1,1),(run\_uuid\_2,2),(run\_uuid\_3,2){]} \\
		3                     & {[}(run\_uuid\_1,1),(run\_uuid\_2,1),(run\_uuid\_3,3){]} \\
		4                     & {[}(run\_uuid\_1,4),(run\_uuid\_2,2),(run\_uuid\_3,2){]} \\
		\bottomrule
	\end{tabularx}
	\label{tab:tuplas}
\end{table}

Hasta ahora, el conjunto de datos representado por la Tabla \ref{tab:tuplas} muestra, por cada una de las preguntas, a qué cluster fue asignada por cada ejecución del algoritmo. Siguiendo con el ejemplo, la pregunta 1 y la pregunta 2 fueron asignadas al cluster con identificador 1 para \(run\_uuid\_1\), ya que las dos preguntas poseen la tupla \((run\_uuid\_1,1)\). Este formato de representación facilita el cálculo de la cantidad de veces que dos preguntas fueron asignadas al mismo cluster, para una misma ejecución. Para obtener el resultado de este cálculo, es necesario realizar una intersección de arreglos para cada una de las combinaciones de preguntas. Para este ejemplo, la intersección de arreglos se muestra en la Tabla \ref{tab:interseccion}.

\begin{table}[h!]
	\footnotesize
	\caption{Conjunto de datos intermedio que indica cuando dos preguntas coinciden en el mismo cluster/ejecución mediante un arreglo de tuplas.}
	\begin{tabularx}{\textwidth}{>{\centering\arraybackslash}p{2.5cm}>{\centering\arraybackslash}p{2.5cm}>{\centering\arraybackslash}p{10cm}}
		\toprule
		\textbf{question\_id\_1} & \textbf{question\_id\_2} & \multicolumn{1}{c|}{\textbf{tuples}}                     \\
		\midrule
		1 & 2 & {[}(run\_uuid\_1,1){]} \\
		1                        & 3                        & {[}(run\_uuid\_1,1),(run\_uuid\_2,1),(run\_uuid\_3,3){]} \\
		1 & 4 & {[}{]}                 \\
		2 & 3 & {[}(run\_uuid\_1,1){]} \\
		2 & 4 & {[}(run\_uuid\_2,2){]} \\
		3 & 4 & {[}{]}                 \\
		\bottomrule
	\end{tabularx}
	\label{tab:interseccion}
\end{table}

La longitud de cada una de las listas indica la cantidad de veces que una pregunta coincide con otra pregunta en el mismo cluster para una misma ejecución. Por ejemplo, las preguntas 1 y 3 resultaron en el mismo cluster para todas las ejecuciones, lo que indicaría una gran similaridad entre ellas. Por el contrario, las preguntas 1 y 4 no presentan ninguna intersección, lo cual es interpretado como una similaridad muy baja entre las preguntas en cuestión.

\bigskip La cantidad de veces que una pregunta coincide con otra, dividido la cantidad total de ejecuciones, nos indica, en un rango normalizado \([0,1]\), cuán similares son entre cada una de ellas, de manera adimensional. Aplicando este concepto, calculamos:

\begin{python}
len(set(tuples_1).intersection(set(tuples_2))) / total_runs
\end{python}
Este es un cociente donde la expresión \pythoninline{len(set(tuples\_1).intersection(set(tuples\_2)))} en el numerador representa la cantidad de veces que dos preguntas coincidieron en el mismo cluster para una misma ejecución, y la expresión \pythoninline{total_runs} en el denominador representa la cantidad total de ejecuciones, respondiendo a la fórmula de Ensamble de Clustering de Acumulacion de Evidencias:
\[C(i,j)=\frac{n_{ij}}{N}.\]
donde \(n_{ij}\) y \(N\) corresponden a las expresiones del numerador y denominador arriba mencionadas, respectivamente.

\bigskip Considerando el ejemplo en cuestión, el conjunto de datos resultado de aplicar la fórmula anterior se muestra en la Tabla \ref{tab:coasociacion}, que representa a la matriz de co-asociación de salida del proceso de ensamble de clustering. Esta matriz se representa con tres columnas: las dos primeras identifican a las preguntas y la tercera es la información de similaridad entre ellas.

\begin{table}[h!]
	\footnotesize
	\caption{Ejemplo de matriz de co-asociación salida del proceso de ensamble de clustering.}
	\begin{tabularx}{\textwidth}{*{7}{>{\centering\arraybackslash}X}}
		\toprule
		\textbf{question\_id\_1} & \textbf{question\_id\_2} & \textbf{similarity} \\
		\midrule
		1                        & 2                        & 0.3333              \\
		1                        & 3                        & 1.0                 \\
		1                        & 4                        & 0                   \\
		2                        & 3                        & 0.3333              \\
		2                        & 4                        & 0.3333              \\
		3                        & 4                        & 0                   \\
		\bottomrule
	\end{tabularx}
	\label{tab:coasociacion}
\end{table}
Particularmente, la matriz de co-asociación mostrada en la Tabla \ref{tab:coasociacion}, indica que las preguntas 1 y 3 coincidieron en el \(100\%\) de las ejecuciones (valor \(1.0\), segunda fila), mientras que las preguntas 1 y 2 solo en un tercio de las mismas (valor \(0.3333\), primera fila), y las preguntas 1 y 4 nunca coincidieron (valor \(0\), tercera fila). Estos indicadores son tomados entonces como la similaridad resultante del método propuesto.
