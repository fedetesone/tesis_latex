\section{Resultados}

\subsection{Análisis del método propuesto}
\begin{frame}[allowframebreaks]
	\frametitle{Análisis del método propuesto}
	\begin{table}[h!]
		\scriptsize
		\begin{tabularx}{\textwidth}{Xccccccc}
			\toprule
			\textbf{k / Tam. muestra} &
			\textbf{100} &
			\textbf{500} &
			\textbf{1000} &
			\textbf{1500} &
			\textbf{2000} &
			\textbf{Media} &
			\textbf{Varianza} \\ \midrule
			\textbf{5}  & 0.322 & 0.3366 & 0.3463 & 0.3486 & 0.3407 & 0.33884                         & 0.0004429                         \\
			\textbf{10} & 0.318 & 0.3254 & 0.3406 & 0.3395 & 0.3389 & 0.33248                         & \cellcolor[HTML]{D9EAD3}0.0004162 \\
			\textbf{15} & 0.314 & 0.325  & 0.3419 & 0.341  & 0.3354 & 0.33146                         & 0.0005621                       \\
			\textbf{20} & 0.318 & 0.3192 & 0.3393 & 0.3424 & 0.3375 & 0.33128                         & 0.0005489                        \\
			\textbf{25} & 0.309 & 0.3226 & 0.3378 & 0.3397 & 0.3354 & 0.3289                          & 0.0006738                           \\
			\textbf{30} & 0.304 & 0.3218 & 0.3364 & 0.3384 & 0.3353 & 0.32718                         & 0.0008430                        \\
			\textbf{35} & 0.307 & 0.3188 & 0.3366 & 0.3364 & 0.3315 & 0.32606                         & 0.0006635                       \\
			\textbf{40} & 0.309 & 0.317  & 0.3397 & 0.3384 & 0.3296 & 0.32674                         & 0.0007216                      \\
			\textbf{45} & 0.311 & 0.3162 & 0.3361 & 0.335  & 0.3317 & 0.326                           & 0.000536                      \\
			\textbf{50} & 0.308 & 0.314  & 0.3324 & 0.3293 & 0.3306 & \cellcolor[HTML]{D9EAD3}0.32286 & 0.0004917                         \\
			\midrule
			\textbf{Media} &
			\cellcolor[HTML]{D9EAD3}0.312 &
			0.32166 &
			0.33871 &
			0.33887 &
			0.33466 &
			\\
			\textbf{Varianza} &
			0.0003 &
			0.0003736 &
			0.0001299 &
			0.0002258 &
			\cellcolor[HTML]{D9EAD3}0.0001248 &
			\\
			\bottomrule
		\end{tabularx}
		\label{tab:analisis-error-vs-k}
	\end{table}

	\framebreak

	\begin{filecontents*}{erroresktamanomuestra.csv}
		100,0.322,0.318,0.314,0.318,0.309,0.304,0.307,0.309,0.311,0.308
		500,0.3366,0.3254,0.325,0.3192,0.3226,0.3218,0.3188,0.317,0.3162,0.314
		1000,0.3463,0.3406,0.3419,0.3393,0.3378,0.3364,0.3366,0.3397,0.3361,0.3324
		1500,0.3486,0.3395,0.341,0.3424,0.3397,0.3384,0.3364,0.3384,0.335,0.3293
		2000,0.3407,0.3389,0.3354,0.3375,0.3354,0.3353,0.3315,0.3296,0.3317,0.3306
	\end{filecontents*}

	\begin{figure}
		\centering
		\scriptsize
		\resizebox{\textwidth}{!}{%
			\begin{tikzpicture}
				\begin{axis}[
					xlabel={Tamaños de muestra},
					ylabel={Error},
					xmin=100, xmax=2000,
					ymin=0.30, ymax=0.36,
					xtick={100,500,1000,1500,2000},
					ytick={0.30,0.31,...,0.36},
					legend pos=north west,
					ymajorgrids=true,
					grid style=dashed,
					]

					\addplot table [mark=square,x index=0, y index=1, col sep=comma] {erroresktamanomuestra.csv};
					\label{k5}
					\addplot table [mark=square,x index=0, y index=2, col sep=comma] {erroresktamanomuestra.csv};
					\label{k10}
					\addplot table [mark=square,x index=0, y index=3, col sep=comma] {erroresktamanomuestra.csv};
					\label{k15}
					\addplot table [mark=square,x index=0, y index=4, col sep=comma] {erroresktamanomuestra.csv};
					\label{k20}
					\addplot table [mark=square,x index=0, y index=5, col sep=comma] {erroresktamanomuestra.csv};
					\label{k25}
					\addplot table [mark=square,x index=0, y index=5, col sep=comma] {erroresktamanomuestra.csv};
					\label{k30}
					\addplot table [mark=square,x index=0, y index=7, col sep=comma] {erroresktamanomuestra.csv};
					\label{k35}
					\addplot table [mark=square,x index=0, y index=8, col sep=comma] {erroresktamanomuestra.csv};
					\label{k40}
					\addplot table [mark=square,x index=0, y index=9, col sep=comma] {erroresktamanomuestra.csv};
					\label{k45}
					\addplot table [mark=square,x index=0, y index=10, col sep=comma] {erroresktamanomuestra.csv};
					\label{k50}
				\end{axis}

				% Primer cuadro de leyendas.
				\node [draw,fill=white] at (rel axis cs: 0.55,0.2) {\scriptsize\shortstack[l]{
						\ref{k5} $k = 5$ \\
						\ref{k10} $k = 10$ \\
						\ref{k15} $k = 15$ \\
						\ref{k20} $k = 20$ \\
						\ref{k25} $k = 25$}};

				% Segundo cuadro de leyendas.
				\node [draw,fill=white] at (rel axis cs: 0.83,0.2) {\scriptsize\shortstack[l]{
						\ref{k30} $k = 30$ \\
						\ref{k35} $k = 35$ \\
						\ref{k40} $k = 40$ \\
						\ref{k45} $k = 45$ \\
						\ref{k50} $k = 50$}};
			\end{tikzpicture}
		}
		\caption{Errores de los valores de \(k\) en los distintos tamaños de muestra.}
		\label{fig:errores-k-tamanos-muestra}
	\end{figure}
\end{frame}

\subsection{Análisis del método propuesto y algoritmos del estado del arte}
\begin{frame}[allowframebreaks]
	\frametitle{Análisis del método propuesto y algoritmos del estado del arte}
	Lorem ipsum dolor sit amet, consectetur adipisicing elit, sed do eiusmod tempor incididunt ut labore et dolore magna aliqua.
\end{frame}

\subsection{Otras observaciones de interés}
\begin{frame}[allowframebreaks]
	\frametitle{Análisis del método propuesto y algoritmos del estado del arte}
	Lorem ipsum dolor sit amet, consectetur adipisicing elit, sed do eiusmod tempor incididunt ut labore et dolore magna aliqua.
\end{frame}

\subsection{Análisis de desempeño}
\begin{frame}[allowframebreaks]
	\frametitle{Análisis de desempeño}
	Lorem ipsum dolor sit amet, consectetur adipisicing elit, sed do eiusmod tempor incididunt ut labore et dolore magna aliqua.
\end{frame}

\subsection{Resumen de resultados}
\begin{frame}[allowframebreaks]
	\frametitle{Análisis de desempeño}
	Lorem ipsum dolor sit amet, consectetur adipisicing elit, sed do eiusmod tempor incididunt ut labore et dolore magna aliqua.
\end{frame}