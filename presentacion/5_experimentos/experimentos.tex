\section{Experimentos}

\subsection{Estado del arte}
\begin{frame}
	\frametitle{Estado del arte}
		\begin{table}
			\footnotesize
			\begin{tabularx}{\textwidth}{*{7}{>{\centering\arraybackslash}X}}
				\toprule
				\multicolumn{3}{l}{\multirow{2}{*}{}} &
				\multicolumn{2}{c}{\textbf{Predicho}} &
				\multirow{2}{*}{\textbf{Exactitud}} &
				\multirow{2}{*}{\textbf{Error}} \\ \cmidrule(lr){4-5}
				\multicolumn{3}{l}{} &
				\multicolumn{1}{c}{\textbf{0}} &
				\multicolumn{1}{c}{\textbf{1}} &
				&
				\\ \midrule
				\multicolumn{1}{c}{\multirow{2}{*}{\textbf{TF}}} &
				\multicolumn{1}{c}{\multirow{2}{*}{\textbf{Real}}} &
				\multicolumn{1}{c}{\textbf{0}} &
				\multicolumn{1}{c}{0.4355} &
				\multicolumn{1}{c}{0.1953} &
				\multicolumn{1}{c}{\multirow{2}{*}{0.6776}} &
				\multicolumn{1}{c}{\multirow{2}{*}{0.3224}} \\ \cmidrule(lr){3-5}
				\multicolumn{1}{c}{} &
				\multicolumn{1}{c}{} &
				\multicolumn{1}{c}{\textbf{1}} &
				\multicolumn{1}{c}{0.1271} &
				\multicolumn{1}{c}{0.2421} &
				\multicolumn{1}{c}{} &
				\multicolumn{1}{c}{} \\ \midrule
				\multicolumn{1}{c}{\multirow{2}{*}{\textbf{TF/IDF}}} &
				\multicolumn{1}{c}{\multirow{2}{*}{\textbf{Real}}} &
				\multicolumn{1}{c}{\textbf{0}} &
				\multicolumn{1}{c}{0.4477} &
				\multicolumn{1}{c}{0.1831} &
				\multicolumn{1}{c}{\multirow{2}{*}{0.6685}} &
				\multicolumn{1}{c}{\multirow{2}{*}{0.3315}} \\ \cmidrule(lr){3-5}
				\multicolumn{1}{c}{} &
				\multicolumn{1}{c}{} &
				\multicolumn{1}{c}{\textbf{1}} &
				\multicolumn{1}{c}{0.1484} &
				\multicolumn{1}{c}{0.2208} &
				\multicolumn{1}{c}{} &
				\multicolumn{1}{c}{} \\ \midrule
				\multicolumn{1}{c}{\multirow{2}{*}{\textbf{Word2Vec}}} &
				\multicolumn{1}{c}{\multirow{2}{*}{\textbf{Real}}} &
				\multicolumn{1}{c}{\textbf{0}} &
				\multicolumn{1}{c}{0.4343} &
				\multicolumn{1}{c}{0.1965} &
				\multicolumn{1}{c}{\multirow{2}{*}{0.6788}} &
				\multicolumn{1}{c}{\multirow{2}{*}{0.3212}} \\ \cmidrule(lr){3-5}
				\multicolumn{1}{c}{} &
				\multicolumn{1}{c}{} &
				\multicolumn{1}{c}{\textbf{1}} &
				\multicolumn{1}{c}{0.1247} &
				\multicolumn{1}{c}{0.2445} &
				\multicolumn{1}{c}{} &
				\multicolumn{1}{c}{} \\ \midrule
				\multicolumn{1}{c}{\multirow{2}{*}{\textbf{FastText}}} &
				\multicolumn{1}{c}{\multirow{2}{*}{\textbf{Real}}} &
				\multicolumn{1}{c}{\textbf{0}} &
				\multicolumn{1}{c}{0.5033} &
				\multicolumn{1}{c}{0.1275} &
				\multicolumn{1}{c}{\multirow{2}{*}{0.6725}} &
				\multicolumn{1}{c}{\multirow{2}{*}{0.3275}} \\ \cmidrule(lr){3-5}
				\multicolumn{1}{c}{} &
				\multicolumn{1}{c}{} &
				\multicolumn{1}{c}{\textbf{1}} &
				\multicolumn{1}{c}{0.2} &
				\multicolumn{1}{c}{0.1692} &
				\multicolumn{1}{c}{} &
				\multicolumn{1}{c}{} \\ \midrule
				\multicolumn{1}{c}{\multirow{2}{*}{\textbf{Semantic Distance}}} &
				\multicolumn{1}{c}{\multirow{2}{*}{\textbf{Real}}} &
				\multicolumn{1}{c}{\textbf{0}} &
				\multicolumn{1}{c}{0.4877} &
				\multicolumn{1}{c}{0.1431} &
				\multicolumn{1}{c}{\multirow{2}{*}{\textbf{0.6797}}} &
				\multicolumn{1}{c}{\multirow{2}{*}{\textbf{0.3203}}} \\ \cmidrule(lr){3-5}
				\multicolumn{1}{c}{} &
				\multicolumn{1}{c}{} &
				\multicolumn{1}{l}{1} &
				\multicolumn{1}{l}{0.1772} &
				\multicolumn{1}{l}{0.192} &
				\multicolumn{1}{c}{} &
				\multicolumn{1}{c}{} \\ \bottomrule
			\end{tabularx}
			\label{tab:desempeno-estado-del-arte}
		\end{table}
\end{frame}

\subsection{Preprocesamiento y muestreo del conjunto de datos}
\begin{frame}
	\frametitle{Preprocesamiento y muestreo del conjunto de datos}
	\begin{itemize}
		\item
		Preprocesamiento:
		\begin{enumerate}[<*>]
			\item Convertir el texto en minúscula.
			\item Eliminar fórmulas; las cuales están encerradas entre etiquetas [math][/math] y [code][/code].
			\item Reemplazar números por letras.
			\item Eliminar caracteres especiales, ya que los datos deben ser uniformes.
		\end{enumerate}

		\bigskip
		\item
		Muestreo:
		\begin{itemize}[<*>]
			\item Generación pseudoaleatoria con criterios de aceptación.
			\item Garantizar subconjuntos estadistícamente significativos.
		\end{itemize}
	\end{itemize}

\end{frame}

\subsection{Generación de particiones}
\begin{frame}[allowframebreaks]
	\frametitle{Generación de particiones}

	Ejemplo de la estructura de los subconjuntos de muestreo:
	\begin{table}[h!]
		\footnotesize
		\begin{tabularx}{\textwidth}{*{7}{>{\centering\arraybackslash}X}}
			\toprule
			\textbf{question\_pair\_id} & \textbf{question\_1} & \textbf{question\_2} \\
			\midrule
			123004                      & question\_0         & question\_2         \\
			98776                       & question\_1         & question\_3         \\
			\bottomrule
		\end{tabularx}
		\label{tab:archivo-entrada}
	\end{table}

	Combinación  de todas las preguntas individuales de una muestra:
	\begin{table}[h!]
		\footnotesize
		\begin{tabularx}{\textwidth}{*{7}{>{\centering\arraybackslash}X}}
			\toprule
			\textbf{sequence\_id\_1} & \textbf{question\_id\_1} & \textbf{sequence\_id\_2} & \textbf{question\_id\_2} \\
			\midrule
			0 & question\_0 & 1 & question\_1 \\
			0 & question\_0 & 2 & question\_2 \\
			0 & question\_0 & 3 & question\_3 \\
			1 & question\_1 & 2 & question\_2 \\
			1 & question\_1 & 3 & question\_3 \\
			2 & question\_2 & 3 & question\_3 \\
			\bottomrule
		\end{tabularx}
		\label{tab:matriz-triangular}
	\end{table}

	\framebreak
	\textbf{Cálculo de similaridad}
	Ejemplo de la estructura de matriz de similaridad en formato de tabla.
	\begin{table}[h!]
		\footnotesize
		\begin{tabularx}{\textwidth}{*{7}{>{\centering\arraybackslash}X}}
			\toprule
			\textbf{sequence\_id\_1} & \textbf{question\_id\_1} & \textbf{sequence\_id\_2} & \textbf{question\_id\_2} & \textbf{similarity} \\
			\midrule
			0 & question\_0 & 1 & question\_1 & similarity\_01 \\
			0 & question\_0 & 2 & question\_2 & similarity\_02 \\
			0 & question\_0 & 3 & question\_3 & similarity\_03 \\
			1 & question\_1 & 2 & question\_2 & similarity\_12 \\
			1 & question\_1 & 3 & question\_3 & similarity\_13 \\
			2 & question\_2 & 3 & question\_3 & similarity\_23 \\
			\bottomrule
		\end{tabularx}
		\label{tab:matriz-similaridad}
	\end{table}

	También se puede ver como una matriz triangular superior:
	\[\begin{bmatrix}0 & similarity\_01 & similarity\_02 & similarity\_03 \\ 0 & 0 & similarity\_12 & similarity\_13  \\ 0 & 0  & 0 & similarity\_23  \\ 0 & 0 & 0 & 0 \end{bmatrix}.\]
\end{frame}

\begin{frame}[fragile]
	\frametitle{Clustering y etiquetado}
	\begin{itemize}
		\item Por cada una de las matrices de similaridad se realizan varias ejecuciones de clustering PAM.
		\item Por cada una de las ejecuciones PAM se proporciona un \(k\) inicial.
		\item Los resultados poseen la siguiente estructura:
		\bigskip
			\begin{table}
				\footnotesize
				\begin{tabularx}{\textwidth}{ccc}
					\toprule
					\textbf{run\_uuid}& \textbf{question\_id} & \textbf{assigned\_medoid} \\
					\midrule
					63815467136575428551131593057064980770 & 336 & 856 \\
					63815467136575428551131593057064980770 & 342& 856 \\
					63815467136575428551131593057064980770 & 26 & 358 \\
					63815467136575428551131593057064980770 & 1364 & 437 \\
					\bottomrule
				\end{tabularx}
				\label{tab:salida-clustering-1}
			\end{table}
	\end{itemize}
\end{frame}

\subsection{Ensamble de Clustering}
\begin{frame}[allowframebreaks]
	\frametitle{Ensamble de Clustering}
	\begin{footnotesize}
	Para explicar el procedimiento de \textbf{Ensamble de Clustering}, se consideran 3 resultados de ejecuciones ejemplo:
	\end{footnotesize}

	\begin{table}[h!]
		\footnotesize
		\begin{tabularx}{\textwidth}{*{7}{>{\centering\arraybackslash}X}}
			\toprule
			\textbf{run\_uuid} & \textbf{question\_id} & \textbf{cluster\_id} \\
			\midrule
			run\_uuid\_1       & 1                     & 1                    \\
			run\_uuid\_1       & 2                     & 1                    \\
			run\_uuid\_1       & 3                     & 1                    \\
			run\_uuid\_1       & 4                     & 4                    \\
			\bottomrule
		\end{tabularx}
		\label{tab:run1}
	\end{table}

	\begin{table}[h!]
		\footnotesize
		\begin{tabularx}{\textwidth}{*{7}{>{\centering\arraybackslash}X}}
			\toprule
			\textbf{run\_uuid} & \textbf{question\_id} & \textbf{cluster\_id} \\
			\midrule
			run\_uuid\_2       & 1                     & 1                    \\
			run\_uuid\_2       & 2                     & 2                    \\
			run\_uuid\_2       & 3                     & 1                    \\
			run\_uuid\_2       & 4                     & 2                    \\
			\bottomrule
		\end{tabularx}
		\label{tab:run2}
	\end{table}

	\begin{table}[h!]
		\footnotesize
		\begin{tabularx}{\textwidth}{*{7}{>{\centering\arraybackslash}X}}
			\toprule
			\textbf{run\_uuid} & \textbf{question\_id} & \textbf{cluster\_id} \\
			\midrule
			run\_uuid\_3       & 1                     & 3                    \\
			run\_uuid\_3       & 2                     & 2                    \\
			run\_uuid\_3       & 3                     & 3                    \\
			run\_uuid\_3       & 4                     & 2                    \\
			\bottomrule
		\end{tabularx}
		\label{tab:run3}
	\end{table}

	\begin{footnotesize}
		Luego, el resultado de todas las ejecuciones se agrupa por pregunta individual, de la siguiente forma:
	\end{footnotesize}

	\begin{table}[h!]
		\footnotesize
		\begin{tabularx}{\textwidth}{>{\centering\arraybackslash}p{1.5cm}>{\centering\arraybackslash}p{10cm}}
			\toprule
			\textbf{question\_id} & \textbf{tuples}                                          \\
			\midrule
			1                     & {[}(run\_uuid\_1,1),(run\_uuid\_2,1),(run\_uuid\_3,3){]} \\
			2                     & {[}(run\_uuid\_1,1),(run\_uuid\_2,2),(run\_uuid\_3,2){]} \\
			3                     & {[}(run\_uuid\_1,1),(run\_uuid\_2,1),(run\_uuid\_3,3){]} \\
			4                     & {[}(run\_uuid\_1,4),(run\_uuid\_2,2),(run\_uuid\_3,2){]} \\
			\bottomrule
		\end{tabularx}
		\label{tab:tuplas}
	\end{table}

	Se genera un conjunto de datos intermedio con la interseccion de los conjuntos para la combinación de todas las preguntas individuales:
	\begin{table}[h!]
		\footnotesize
		\begin{tabularx}{\textwidth}{>{\centering\arraybackslash}p{2.0cm}>{\centering\arraybackslash}p{2.0cm}>{\centering\arraybackslash}p{7cm}}
			\toprule
			\textbf{question\_id\_1} & \textbf{question\_id\_2} & \multicolumn{1}{c|}{\textbf{tuples}}                     \\
			\midrule
			1 & 2 & {[}(run\_uuid\_1,1){]} \\
			1                        & 3                        & {[}(run\_uuid\_1,1),(run\_uuid\_2,1),(run\_uuid\_3,3){]} \\
			1 & 4 & {[}{]}                 \\
			2 & 3 & {[}(run\_uuid\_1,1){]} \\
			2 & 4 & {[}(run\_uuid\_2,2){]} \\
			3 & 4 & {[}{]}                 \\
			\bottomrule
		\end{tabularx}
		\label{tab:interseccion}
	\end{table}

	\begin{footnotesize}
		Por ejemplo: pregunta 1 = {[}\textbf{(run\_uuid\_1,1)},(run\_uuid\_2,1),(run\_uuid\_3,3){]} y la pregunta 2 = {[}\textbf{(run\_uuid\_1,1)},(run\_uuid\_2,2),(run\_uuid\_3,2){]}.
	\end{footnotesize}

	\framebreak

	Se cuenta la cantidad de veces que una pregunta coincide con otra para una misma ejecución.
	\[len(set(tuples\_1).intersection(set(tuples\_2))) / total\_runs\]

	Respondiendo a la formula de Ensamble de Clustering de Acumulación de Evidencias.
	\[C(i,j)=\frac{n_{ij}}{N}.\]

	\framebreak

	Y se genera la siguiente estructura como resultado (\(total\_runs = 3\)):
	\begin{table}[h!]
		\footnotesize
		\begin{tabularx}{\textwidth}{*{7}{>{\centering\arraybackslash}X}}
			\toprule
			\textbf{question\_id\_1} & \textbf{question\_id\_2} & \textbf{similarity} \\
			\midrule
			1                        & 2                        & 0.3333              \\
			1                        & 3                        & 1.0                 \\
			1                        & 4                        & 0                   \\
			2                        & 3                        & 0.3333              \\
			2                        & 4                        & 0.3333              \\
			3                        & 4                        & 0                   \\
			\bottomrule
		\end{tabularx}
		\label{tab:coasociacion}
	\end{table}

	La estructura resultante es una \textbf{\emph{matriz de co-asociación}}.
\end{frame}

\subsection{Método de validación}
\begin{frame}
	\frametitle{Método de validación}
	\textbf{Validando los resultados}
	\bigskip
	\begin{itemize}
		\item Generación de resultados estadísticamente significativos se ejecutó el proceso completo de modo iterativo, variando dos parámetros principales: \begin{enumerate}[<*>] \item El tamaño de la muestra. \item El número de clusters \(k\). \end{enumerate}
		\item Como experimentos para este trabajo se realizaron ejecuciones con conjuntos de datos aleatorios de 100, 500, 1000, 1500 y 2000 pares de preguntas.
		\item Para cada tamaño de muestra, se realizaron 10 muestras aleatorias manteniendo un \(k\) fijo.
		\item Dando un total de \(5 \: (\text{distintos tamaños de muestra}) \times 10 \: (\text{cantidad de ejecuciones por tamaño de muestra}) = 50\) matrices de co-asociación resultado, para cada valor de \(k\) dado.
	\end{itemize}
\end{frame}

\begin{frame}
	\frametitle{Matrices de confusión}
	\textbf{Matriz de confusión utilizada}
	\bigskip

	\begin{table}[h!]
		\footnotesize
		\centering
		\begin{tabularx}{0.35\textwidth}{*{7}{>{\centering\arraybackslash}X}}
			\toprule
			\multicolumn{2}{l}{\multirow{2}{*}{}} & \multicolumn{2}{c}{\textbf{Predicho}}                             \\ \cmidrule(l){3-4}
			\multicolumn{2}{l}{}                  & \multicolumn{1}{c}{\textbf{0}} & \multicolumn{1}{c}{\textbf{1}} \\ \midrule
			\multicolumn{1}{c}{\multirow{2}{*}{\textbf{Real}}} & \multicolumn{1}{c}{\textbf{0}} & \multicolumn{1}{c}{a} & \multicolumn{1}{c}{b} \\ \cmidrule(l){2-4}
			\multicolumn{1}{c}{}  & \textbf{1}  & c                               & d                               \\ \bottomrule
		\end{tabularx}
		\label{tab:matriz-confusion}
	\end{table}

	Los indicadores de rendimiento seleccionados para ser evaluados en los experimentos realizados con fines comparativos, son los siguientes:
	\begin{itemize}[<*>]
		\item \textbf{Exactitud:} \((a+d)/(a+b+c+d)\).
		\item \textbf{Error:} \((b+c)/(a+b+c+d)\).
	\end{itemize}
	\bigskip
	En estos indicadores se cumple la condición \(a+b+c+d=1\).
\end{frame}

\begin{frame}[allowframebreaks]
	\frametitle{Preparación de los datos}
	Muestras de pares de preguntas que se utilizó como entrada del método EQuAL:
	\begin{table}[h!]
		\footnotesize
		\centering
		\begin{tabularx}{0.8\textwidth}{*{7}{>{\centering\arraybackslash}c}}
			\toprule
			\textbf{sequence\_id} & \textbf{question\_pair\_id} & \textbf{question\_1} & \textbf{question\_2} & \textbf{equal} \\
			\midrule
			0                     & 123004                      & question\_10         & question\_11         & 1              \\
			1                     & 98776                       & question\_11         & question\_21         & 0              \\
			\bottomrule
		\end{tabularx}
		\label{tab:muestra-validacion}
	\end{table}

	Matriz de co-asociación generada por el proceso EQuAL:
	\begin{table}[h!]
		\footnotesize
		\begin{tabularx}{\textwidth}{*{7}{>{\centering\arraybackslash}X}}
			\toprule
			\textbf{question\_id\_1} & \textbf{question\_id\_2} & \textbf{question\_1} & \textbf{question\_2} & \textbf{similarity} \\
			\midrule
			\rowcolor[HTML]{D9EAD3}
			question\_10 & question\_11 & contenido & contenido & 0.857 \\
			question\_10 & question\_20 & contenido & contenido & 0.210 \\
			question\_10 & question\_21 & contenido & contenido & 0.126 \\
			question\_11 & question\_20 & contenido & contenido & 0.006 \\
			\rowcolor[HTML]{D9EAD3}
			question\_11 & question\_21 & contenido & contenido & 0.368 \\
			question\_20 & question\_21 & contenido & contenido & 0.146 \\
			\bottomrule
		\end{tabularx}
		\label{tab:coasociacion-validacion}
	\end{table}

	Se filtra la tabla anterior con los pares de preguntas que se encuentran en el conjunto de datos de entrada:
	\begin{table}[h!]
		\footnotesize
		\begin{tabularx}{\textwidth}{*{7}{>{\centering\arraybackslash}X}}
			\toprule
			\textbf{question\_id\_1} & \textbf{question\_id\_2} & \textbf{question\_1} & \textbf{question\_2} & \textbf{similarity} \\
			\midrule
			question\_10             & question\_11             & contenido            & contenido            & 0.857               \\
			question\_11             & question\_21             & contenido            & contenido            & 0.368               \\
			\bottomrule
		\end{tabularx}
		\label{tab:filtrado-validacion}
	\end{table}
\end{frame}

\begin{frame}[allowframebreaks]
	\frametitle{Elección del umbral correcto}
	La similaridad \(S\) entre un par de preguntas (\(q_1,q_2)\) es igual o superior a cierto umbral \(t\) se considera que son iguales (valor \(1\)) y distintas si sucede lo contrario (valor \(0\)). De esta forma
	\[f(x) = \left\{ \begin{array}{lcc} 1 & si & S(q_1, q_2)\geq t
		\\ 0 & si & S(q_1, q_2) < t
	\end{array} \right..\]

	\begin{center}
		\centering	Cual de los valores de umbral \(t\) tiene mejor rendimiento?
	\end{center}

	\framebreak

	Se toma valores potenciales de umbral con intervalos \(0.05\) y se forma un arreglo como \textbf{\([0.05, 0.1, 0.15, ..., 0.90, 0.95]\)} y se itera sobre cada uno de ellos. Por cada uno de los valores en el arreglo:
	\begin{enumerate}[<*>]
		\item Se consideran todos los pares de preguntas tomados para realizar la comparación, provenientes de la matriz de co-asociación.
		\item Por cada uno de los valores de similaridad, se asigna \(1\) si son mayores o iguales al umbral, \(0\) si pasa lo contrario.
		\item Si los valores asignados en el paso anterior coinciden con el valor real, se asigna un valor \textit{true} (verdadero), si no coinciden, se asigna \textit{false} (falso).
		\item Se calcula la proporción de pares de preguntas asignadas con \textit{true}, es decir, que el valor real coincide con el predicho, y se obtiene la \textit{exactitud} del método.
	\end{enumerate}

	\framebreak

	Ejemplo de comparación de valores reales y predichos para construcción de matrices de confusión:
	\begin{table}[!htbp]
		\scriptsize
		\centering
		\begin{tabularx}{\textwidth}{*{9}{>{\centering\arraybackslash}c}}
			\toprule
			\textbf{sequence\_id} & \textbf{question\_pair\_id} & \textbf{question\_1} & \textbf{question\_2} & \textbf{real} &  \textbf{predicted} & \textbf{equal}\\
			\midrule
			0 & 123004 & question\_10 & question\_20 & 1 & 1 & true \\
			1 & 98776 & question\_11 & question\_21 & 1 & 1 & true \\
			2 & 14422 & question\_12 & question\_22 & 1 & 0 & false \\
			3 & 12321 & question\_13 & question\_23 & 1 & 1 & true \\
			4 & 999 & question\_14 & question\_24 & 0 & 1 & false \\
			5 & 7448 & question\_15 & question\_25 & 0 & 0 & true \\
			6 & 69553 & question\_16 & question\_26 & 0 & 0 & true \\
			7 & 2447 & question\_17 & question\_27 & 1 & 1 & true \\
			\bottomrule
		\end{tabularx}
		\label{tab:validacion-reales}
	\end{table}

	\framebreak
	Matriz de confusión obtenida:
	\begin{table}[!htbp]
		\footnotesize
		\centering
		\begin{tabularx}{0.35\textwidth}{*{7}{>{\centering\arraybackslash}X}}
			\toprule
			\multicolumn{2}{l}{\multirow{2}{*}{}} & \multicolumn{2}{c}{\textbf{Predicho}}                             \\ \cmidrule(l){3-4}
			\multicolumn{2}{l}{}                  & \multicolumn{1}{c}{\textbf{0}} & \multicolumn{1}{c}{\textbf{1}} \\ \midrule
			\multicolumn{1}{c}{\multirow{2}{*}{\textbf{Real}}} & \multicolumn{1}{c}{\textbf{0}} & \multicolumn{1}{c}{0.25} & \multicolumn{1}{c}{0.125} \\ \cmidrule(l){2-4}
			\multicolumn{1}{c}{}  & \textbf{1}  & 0.125                               & 0.5                               \\ \bottomrule
		\end{tabularx}
		\label{tab:validacion-confusion-ejemplo}
	\end{table}

	\begin{itemize}
		\item \textbf{Exactitud:} \(0.75\).
		\item \textbf{Error:} \(0.25\).
	\end{itemize}
\end{frame}