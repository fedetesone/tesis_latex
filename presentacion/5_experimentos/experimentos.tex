\section{Experimentos}


\begin{frame}
	\frametitle{Objetivos de experimentación}
	En este capítulo se describirá:
	\begin{itemize}[<*>]
		\item \textbf{Resultados obtenidos} en el trabajo del estado del arte.
		\item \textbf{Método de experimentación} en forma de detallada.
		\item \textbf{Método de validación y presentación} de resultados.
	\end{itemize}

	\bigskip

	Los experimentos se realizaron con las siguientes tecnologías:
	\begin{itemize}[<*>]
		\item Código de experimentación en \textbf{Python}, con la utilización de \textbf{Apache Spark} y librerías de soporte.
		\item Validación estadística del método EQuAL en \textbf{R}.
	\end{itemize}
\end{frame}

\begin{frame}
	\frametitle{Matrices de confusión}
	\textbf{Matrices de confusión utilizadas}
	\begin{figure}[!htb]
		\centering
		\begin{minipage}{.45\textwidth}
			\begin{table}[h!]
				\footnotesize
				\centering
				\begin{tabularx}{0.8\textwidth}{*{7}{>{\centering\arraybackslash}X}}
					\toprule
					\multicolumn{2}{l}{\multirow{2}{*}{}} & \multicolumn{2}{c}{\textbf{Predicho}}                             \\ \cmidrule(l){3-4}
					\multicolumn{2}{l}{}                  & \multicolumn{1}{c}{\textbf{0}} & \multicolumn{1}{c}{\textbf{1}} \\ \midrule
					\multicolumn{1}{c}{\multirow{2}{*}{\textbf{Real}}} & \multicolumn{1}{c}{\textbf{0}} & \multicolumn{1}{c}{a} & \multicolumn{1}{c}{b} \\ \cmidrule(l){2-4}
					\multicolumn{1}{c}{}  & \textbf{1}  & c                               & d                               \\ \bottomrule
				\end{tabularx}
				\label{tab:matriz-confusion}
			\end{table}
		\end{minipage}%
		\begin{minipage}{.45\textwidth}
			\begin{footnotesize}
				\begin{center}
					\begin{itemize}[<*>]
						\item a =  Verdaderos Negativos.
						\item b =  Falsos Positivos.
						\item c =  Falsos Negativos.
						\item d =  Verdaderos Positivos.
					\end{itemize}
				\end{center}
			\end{footnotesize}
		\end{minipage}
	\end{figure}

	\bigskip
	Los indicadores de rendimiento seleccionados para ser evaluados en los experimentos realizados con fines comparativos, son los siguientes:
	\begin{itemize}[<*>]
		\item \textbf{Exactitud (accuracy):} \((a+d)/(a+b+c+d)\).
		\item \textbf{Error:} \((b+c)/(a+b+c+d)\).
	\end{itemize}
	En estos indicadores se cumplen la condiciones \(a+b+c+d=1\) y \(error = 1 - exactitud\).
 \end{frame}

\subsection{Estado del arte}
\begin{frame}
	\frametitle{Estado del arte}
		\begin{table}[!htbp]
			\footnotesize
			\begin{tabularx}{\textwidth}{*{8}{>{\centering\arraybackslash}X}}
				\toprule
				\multicolumn{3}{l}{\multirow{2}{*}{}} &
				\multicolumn{2}{c}{\textbf{Predicho}} &
				\multirow{2}{*}{\textbf{Exactitud}} &
				\multirow{2}{*}{\textbf{\begin{tabular}[c]{@{}c@{}}Falsos\\ Positivos\end{tabular}}} &
				\multirow{2}{*}{\textbf{\begin{tabular}[c]{@{}c@{}}Falsos\\ Negativos\end{tabular}}} \\
				\cmidrule(lr){4-5}
				\multicolumn{3}{l}{} &
				\multicolumn{1}{c}{\textbf{0}} &
				\multicolumn{1}{c}{\textbf{1}} &
				&
				\\ \midrule
				\multicolumn{1}{c}{\multirow{2}{*}{\textbf{TF}}} &
				\multicolumn{1}{c}{\multirow{2}{*}{\textbf{Real}}} &
				\multicolumn{1}{c}{\textbf{0}} &
				\multicolumn{1}{c}{\cellcolor[HTML]{00BFFF}0.4355} &
				\multicolumn{1}{c}{0.1953} &
				\multicolumn{1}{c}{\multirow{2}{*}{0.6776}} &
				\multicolumn{1}{c}{\multirow{2}{*}{0.1953}} &
				\multicolumn{1}{c}{\multirow{2}{*}{0.1271}} \\
				\cmidrule(lr){3-5}
				\multicolumn{1}{c}{} &
				\multicolumn{1}{c}{} &
				\multicolumn{1}{c}{\textbf{1}} &
				\multicolumn{1}{c}{0.1271} &
				\multicolumn{1}{c}{\cellcolor[HTML]{FF0000}0.2421} &
				\multicolumn{1}{c}{} &
				\multicolumn{1}{c}{} \\ \midrule
				\multicolumn{1}{c}{\multirow{2}{*}{\textbf{TF/IDF}}} &
				\multicolumn{1}{c}{\multirow{2}{*}{\textbf{Real}}} &
				\multicolumn{1}{c}{\textbf{0}} &
				\multicolumn{1}{c}{\cellcolor[HTML]{00BFFF}0.4477} &
				\multicolumn{1}{c}{0.1831} &
				\multicolumn{1}{c}{\multirow{2}{*}{0.6685}} &
				\multicolumn{1}{c}{\multirow{2}{*}{0.1831}} &
				\multicolumn{1}{c}{\multirow{2}{*}{0.1484}} \\ \cmidrule(lr){3-5}
				\multicolumn{1}{c}{} &
				\multicolumn{1}{c}{} &
				\multicolumn{1}{c}{\textbf{1}} &
				\multicolumn{1}{c}{0.1484} &
				\multicolumn{1}{c}{\cellcolor[HTML]{FF0000}0.2208} &
				\multicolumn{1}{c}{} &
				\multicolumn{1}{c}{} \\ \midrule
				\multicolumn{1}{c}{\multirow{2}{*}{\textbf{Word2Vec}}} &
				\multicolumn{1}{c}{\multirow{2}{*}{\textbf{Real}}} &
				\multicolumn{1}{c}{\textbf{0}} &
				\multicolumn{1}{c}{\cellcolor[HTML]{00BFFF}0.4343} &
				\multicolumn{1}{c}{0.1965} &
				\multicolumn{1}{c}{\multirow{2}{*}{0.6788}} &
				\multicolumn{1}{c}{\multirow{2}{*}{0.1965}} &
				\multicolumn{1}{c}{\multirow{2}{*}{0.1247}} \\ \cmidrule(lr){3-5}
				\multicolumn{1}{c}{} &
				\multicolumn{1}{c}{} &
				\multicolumn{1}{c}{\textbf{1}} &
				\multicolumn{1}{c}{0.1247} &
				\multicolumn{1}{c}{\cellcolor[HTML]{FF0000}0.2445} &
				\multicolumn{1}{c}{} &
				\multicolumn{1}{c}{} \\ \midrule
				\multicolumn{1}{c}{\multirow{2}{*}{\textbf{FastText}}} &
				\multicolumn{1}{c}{\multirow{2}{*}{\textbf{Real}}} &
				\multicolumn{1}{c}{\textbf{0}} &
				\multicolumn{1}{c}{\cellcolor[HTML]{00BFFF}0.5033} &
				\multicolumn{1}{c}{0.1275} &
				\multicolumn{1}{c}{\multirow{2}{*}{0.6725}} &
				\multicolumn{1}{c}{\multirow{2}{*}{0.1275}} &
				\multicolumn{1}{c}{\multirow{2}{*}{0.2}} \\ \cmidrule(lr){3-5}
				\multicolumn{1}{c}{} &
				\multicolumn{1}{c}{} &
				\multicolumn{1}{c}{\textbf{1}} &
				\multicolumn{1}{c}{0.2} &
				\multicolumn{1}{c}{\cellcolor[HTML]{FF0000}0.1692} &
				\multicolumn{1}{c}{} &
				\multicolumn{1}{c}{} \\ \midrule
				\multicolumn{1}{c}{\multirow{2}{*}{\textbf{Semantic Distance}}} &
				\multicolumn{1}{c}{\multirow{2}{*}{\textbf{Real}}} &
				\multicolumn{1}{c}{\textbf{0}} &
				\multicolumn{1}{c}{\cellcolor[HTML]{00BFFF}0.4877} &
				\multicolumn{1}{c}{0.1431} &
				\multicolumn{1}{c}{\multirow{2}{*}{0.6797}} &
				\multicolumn{1}{c}{\multirow{2}{*}{0.1431}} &
				\multicolumn{1}{c}{\multirow{2}{*}{0.1772}} \\ \cmidrule(lr){3-5}
				\multicolumn{1}{c}{} &
				\multicolumn{1}{c}{} &
				\multicolumn{1}{l}{1} &
				\multicolumn{1}{l}{0.1772} &
				\multicolumn{1}{l}{\cellcolor[HTML]{FF0000}0.192} &
				\multicolumn{1}{c}{} &
				\multicolumn{1}{c}{} \\

				\bottomrule
			\end{tabularx}
			\label{tab:desempeno-estado-del-arte}
		\end{table}

	\textbf{Nota}: el conjunto de datos original tiene \(36, 9\%\) pares de preguntas de clase \(1\) y \(63,1\%\) de clase \(0\).
\end{frame}

\subsection{Preprocesamiento y muestreo del conjunto de datos}
\begin{frame}
	\frametitle{Preprocesamiento y muestreo del conjunto de datos}
	\begin{itemize}
		\item
		Preprocesamiento:
		\begin{enumerate}[<*>]
			\item Convertir el texto en minúscula.
			\item Eliminar fórmulas; las cuales están encerradas entre etiquetas [math][/math] y [code][/code].
			\item Reemplazar números por letras.
			\item Eliminar caracteres especiales, ya que los datos deben ser uniformes.
		\end{enumerate}

		\bigskip
		\item
		Muestreo:
		\begin{itemize}[<*>]
			\item Generación pseudoaleatoria con criterios de aceptación.
			\item Garantizar subconjuntos estadísticamente significativos.
		\end{itemize}
	\end{itemize}

\end{frame}

\subsection{Generación de particiones}
\begin{frame}[allowframebreaks]
	\frametitle{Generación de particiones}

	Ejemplo de la estructura de los subconjuntos de muestreo:
	\begin{table}[h!]
		\footnotesize
		\begin{tabularx}{\textwidth}{*{7}{>{\centering\arraybackslash}X}}
			\toprule
			\textbf{question\_pair\_id} & \textbf{question\_id\_1} & \textbf{question\_id\_2} \\
			\midrule
			123004                      & question\_0         & question\_2         \\
			98776                       & question\_1         & question\_3         \\
			\bottomrule
		\end{tabularx}
		\label{tab:archivo-entrada}
	\end{table}

	Combinación  de todas las preguntas individuales de una muestra:
	\begin{table}[h!]
		\footnotesize
		\begin{tabularx}{\textwidth}{*{7}{>{\centering\arraybackslash}X}}
			\toprule
			\textbf{sequence\_id\_1} & \textbf{question\_id\_1} & \textbf{sequence\_id\_2} & \textbf{question\_id\_2} \\
			\midrule
			0 & question\_0 & 1 & question\_1 \\
			0 & question\_0 & 2 & question\_2 \\
			0 & question\_0 & 3 & question\_3 \\
			1 & question\_1 & 2 & question\_2 \\
			1 & question\_1 & 3 & question\_3 \\
			2 & question\_2 & 3 & question\_3 \\
			\bottomrule
		\end{tabularx}
		\label{tab:matriz-triangular}
	\end{table}

	\framebreak
	\textbf{Cálculo de similaridad}
	Ejemplo de la estructura de matriz de similaridad en formato de tabla.
	\begin{table}[h!]
		\footnotesize
		\begin{tabularx}{\textwidth}{*{7}{>{\centering\arraybackslash}X}}
			\toprule
			\textbf{sequence\_id\_1} & \textbf{question\_id\_1} & \textbf{sequence\_id\_2} & \textbf{question\_id\_2} & \textbf{similarity} \\
			\midrule
			0 & question\_0 & 1 & question\_1 & similarity\_01 \\
			0 & question\_0 & 2 & question\_2 & similarity\_02 \\
			0 & question\_0 & 3 & question\_3 & similarity\_03 \\
			1 & question\_1 & 2 & question\_2 & similarity\_12 \\
			1 & question\_1 & 3 & question\_3 & similarity\_13 \\
			2 & question\_2 & 3 & question\_3 & similarity\_23 \\
			\bottomrule
		\end{tabularx}
		\label{tab:matriz-similaridad}
	\end{table}

	También se puede ver como una matriz \(N \times N\) triangular superior:
	\[\begin{bmatrix}0 & similarity\_01 & similarity\_02 & similarity\_03 \\ 0 & 0 & similarity\_12 & similarity\_13  \\ 0 & 0  & 0 & similarity\_23  \\ 0 & 0 & 0 & 0 \end{bmatrix}.\]
\end{frame}

\begin{frame}[fragile]
	\frametitle{Clustering y etiquetado}
	\begin{itemize}
		\item Por cada una de las matrices de similaridad se realizan varias ejecuciones de clustering PAM (Partición Alrededor de Medoides).
		\item Por cada una de las ejecuciones PAM se proporciona un \(k\) inicial.
		\item Los resultados poseen la siguiente estructura:
		\bigskip
			\begin{table}
				\scriptsize
				\begin{tabularx}{0.9\textwidth}{ccc}
					\toprule
					\textbf{run\_uuid}& \textbf{question\_id} & \textbf{assigned\_medoid} \\
					\midrule
					63815467136575428551131593057064980770 & 336 & 856 \\
					63815467136575428551131593057064980770 & 342& 856 \\
					63815467136575428551131593057064980770 & 26 & 358 \\
					63815467136575428551131593057064980770 & 1364 & 437 \\
					\bottomrule
				\end{tabularx}
				\label{tab:salida-clustering-1}
			\end{table}
	\end{itemize}
\end{frame}

\subsection{Ensamble de Clustering}
\begin{frame}[allowframebreaks,fragile]
	\frametitle{Ensamble de Clustering}
	\begin{footnotesize}
		Se consideran 3 resultados de ejecuciones ejemplo:
	\end{footnotesize}

	\medskip

	\begin{scriptsize}
		\textbf{Ejecución 1}
		\begin{table}[h!]
			\vspace{-0.3cm}
			\begin{tabularx}{\textwidth}{*{7}{>{\centering\arraybackslash}X}}
				\toprule
				\textbf{run\_uuid} & \textbf{question\_id} & \textbf{cluster\_id} \\
				\midrule
				run\_uuid\_1       & 1                     & 1                    \\
				run\_uuid\_1       & 2                     & 1                    \\
				run\_uuid\_1       & 3                     & 1                    \\
				run\_uuid\_1       & 4                     & 4                    \\
				\bottomrule
			\end{tabularx}
			\label{tab:run1}
		\end{table}

		\textbf{Ejecución 2}
		\begin{table}[h!]
			\vspace{-0.3cm}
			\begin{tabularx}{\textwidth}{*{7}{>{\centering\arraybackslash}X}}
				\toprule
				\textbf{run\_uuid} & \textbf{question\_id} & \textbf{cluster\_id} \\
				\midrule
				run\_uuid\_2       & 1                     & 1                    \\
				run\_uuid\_2       & 2                     & 2                    \\
				run\_uuid\_2       & 3                     & 1                    \\
				run\_uuid\_2       & 4                     & 2                    \\
				\bottomrule
			\end{tabularx}
			\label{tab:run2}
		\end{table}

		\textbf{Ejecución 3}
		\begin{table}[h!]
			\vspace{-0.3cm}
			\begin{tabularx}{\textwidth}{*{7}{>{\centering\arraybackslash}X}}
				\toprule
				\textbf{run\_uuid} & \textbf{question\_id} & \textbf{cluster\_id} \\
				\midrule
				run\_uuid\_3       & 1                     & 3                    \\
				run\_uuid\_3       & 2                     & 2                    \\
				run\_uuid\_3       & 3                     & 3                    \\
				run\_uuid\_3       & 4                     & 2                    \\
				\bottomrule
			\end{tabularx}
			\label{tab:run3}
		\end{table}
	\end{scriptsize}

	\begin{footnotesize}
		Luego, el resultado de todas las ejecuciones se agrupa por pregunta individual, generando un \textbf{conjunto de tuplas} de la siguiente forma:
	\[[(id\_de\_ejecución\_1,cluster\_asignado), ...,(id\_de\_ejecución\_N,cluster\_asignado)]\]
	\end{footnotesize}

	\bigskip
	\begin{footnotesize}
	Agrupando las particiones 1, 2 y 3:
	\end{footnotesize}
	\begin{table}[h!]
		\scriptsize
		\begin{tabularx}{\textwidth}{>{\centering\arraybackslash}p{1.5cm}>{\centering\arraybackslash}p{10cm}}
			\toprule
			\textbf{question\_id} & \textbf{tuples}                                          \\
			\midrule
			1                     & {[}(run\_uuid\_1,1),(run\_uuid\_2,1),(run\_uuid\_3,3){]} \\
			2                     & {[}(run\_uuid\_1,1),(run\_uuid\_2,2),(run\_uuid\_3,2){]} \\
			3                     & {[}(run\_uuid\_1,1),(run\_uuid\_2,1),(run\_uuid\_3,3){]} \\
			4                     & {[}(run\_uuid\_1,4),(run\_uuid\_2,2),(run\_uuid\_3,2){]} \\
			\bottomrule
		\end{tabularx}
		\label{tab:tuplas}
	\end{table}

	\framebreak

	\begin{footnotesize}
		Se genera un conjunto de datos intermedio con la interseccion de los conjuntos para la combinación de todas las preguntas individuales, por ejemplo:
		\begin{scriptsize}
			\begin{itemize}[<*>]
				\item pregunta 1 = \({[}\textbf{(run\_uuid\_1,1)},(run\_uuid\_2,1),(run\_uuid\_3,3){]}\),
				\item pregunta 2 = \({[}\textbf{(run\_uuid\_1,1)},(run\_uuid\_2,2),(run\_uuid\_3,2){]}\).
			\end{itemize}
		\end{scriptsize}
	\end{footnotesize}

	\bigskip

	\begin{table}[h!]
		\scriptsize
		\begin{tabularx}{\textwidth}{>{\centering\arraybackslash}p{2.0cm}>{\centering\arraybackslash}p{2.0cm}>{\centering\arraybackslash}p{7cm}}
			\toprule
			\textbf{question\_id\_1} & \textbf{question\_id\_2} & \multicolumn{1}{c|}{\textbf{tuples}}                     \\
			\midrule
			1 & 2 & {[}(run\_uuid\_1,1){]} \\
			1                        & 3                        & {[}(run\_uuid\_1,1),(run\_uuid\_2,1),(run\_uuid\_3,3){]} \\
			1 & 4 & {[}{]}                 \\
			2 & 3 & {[}(run\_uuid\_1,1){]} \\
			2 & 4 & {[}(run\_uuid\_2,2){]} \\
			3 & 4 & {[}{]}                 \\
			\bottomrule
		\end{tabularx}
		\label{tab:interseccion}
	\end{table}

	\framebreak

	Se cuenta la cantidad de veces que una pregunta coincide con otra para una misma ejecución.
	\begin{center}
		\begin{footnotesize}
			\begin{verbatim}
			          	len(set(tuples_1).intersection(set(tuples_2))) / total_runs
			\end{verbatim}
		\end{footnotesize}
	\end{center}

	\bigskip

	Respondiendo a la formula de Ensamble de Clustering de Acumulación de Evidencias.
	\[C(i,j)=\frac{n_{ij}}{N}.\]

	\framebreak

	Y se genera la siguiente estructura como resultado (\(total\_runs = 3\)):
	\begin{table}[h!]
		\footnotesize
		\begin{tabularx}{\textwidth}{*{7}{>{\centering\arraybackslash}X}}
			\toprule
			\textbf{question\_id\_1} & \textbf{question\_id\_2} & \textbf{similarity} \\
			\midrule
			1                        & 2                        & 0.3333              \\
			1                        & 3                        & 1.0                 \\
			1                        & 4                        & 0                   \\
			2                        & 3                        & 0.3333              \\
			2                        & 4                        & 0.3333              \\
			3                        & 4                        & 0                   \\
			\bottomrule
		\end{tabularx}
		\label{tab:coasociacion}
	\end{table}

	\begin{center}
		La estructura resultante es una \textbf{\emph{matriz de co-asociación}}.
	\end{center}
\end{frame}

\begin{frame}[allowframebreaks]
	\frametitle{Método de validación}
	Muestras de pares de preguntas que se utilizó como entrada del método EQuAL:
	\begin{table}[h!]
		\footnotesize
		\centering
		\begin{tabularx}{\textwidth}{*{6}{>{\centering\arraybackslash}X}}
			\toprule
			 \textbf{question\_pair} & \textbf{question\_1} & \textbf{question\_2} & \textbf{equal} \\
			\midrule
			123004                      & question\_10         & question\_11         & 1              \\
			98776                       & question\_20         & question\_21         & 0              \\
			\bottomrule
		\end{tabularx}
		\label{tab:muestra-validacion}
	\end{table}

	Matriz de co-asociación generada por el proceso EQuAL:
	\begin{table}[h!]
		\footnotesize
		\begin{tabularx}{\textwidth}{*{7}{>{\centering\arraybackslash}X}}
			\toprule
			\textbf{question\_id\_1} & \textbf{question\_id\_2} & \textbf{question\_1} & \textbf{question\_2} & \textbf{similarity} \\
			\midrule
			\rowcolor[HTML]{D9EAD3}
			question\_10 & question\_11 & contenido & contenido & 0.857 \\
			question\_10 & question\_20 & contenido & contenido & 0.210 \\
			question\_10 & question\_21 & contenido & contenido & 0.126 \\
			question\_11 & question\_20 & contenido & contenido & 0.006 \\
			question\_11 & question\_21 & contenido & contenido & 0.146 \\
			\rowcolor[HTML]{D9EAD3}
			question\_20 & question\_21 & contenido & contenido & 0.368 \\
			\bottomrule
		\end{tabularx}
		\label{tab:coasociacion-validacion}
	\end{table}

	\framebreak
	Muestra de datos original
	\begin{table}[h!]
		\footnotesize
		\centering
		\begin{tabularx}{\textwidth}{*{4}{>{\centering\arraybackslash}X}}
			\toprule
			\textbf{question\_pair} & \textbf{question\_1} & \textbf{question\_2} & \textbf{equal} \\
			\midrule
			123004                      & question\_10         & question\_11         & 1              \\
			98776                       & question\_20         & question\_21         & 0              \\
			\bottomrule
		\end{tabularx}
	\end{table}

	\bigskip

	Se filtra la matriz de co-asociación con los pares de preguntas que se encuentran en el conjunto de datos de entrada:
	\begin{table}[h!]
		\footnotesize
		\begin{tabularx}{\textwidth}{*{5}{>{\centering\arraybackslash}X}}
			\toprule
			\textbf{question\_id\_1} & \textbf{question\_id\_2} & \textbf{question\_1} & \textbf{question\_2} & \textbf{similarity} \\
			\midrule
			question\_10             & question\_11             & contenido            & contenido            & 0.857               \\
			question\_20             & question\_21             & contenido            & contenido            & 0.368               \\
			\bottomrule
		\end{tabularx}
		\label{tab:filtrado-validacion}
	\end{table}

	\begin{center}
		¿Cómo sabemos cuando este valor de similaridad nos indica que dos preguntas son iguales (\(1\)) o distintas (\(0\))?
	\end{center}

\end{frame}

\begin{frame}[allowframebreaks]
	\frametitle{Elección del umbral correcto}
	La similaridad \(S\) entre un par de preguntas (\(q_1,q_2)\) es igual o superior a cierto umbral \(t\) se considera que son iguales (valor \(1\)) y distintas si sucede lo contrario (valor \(0\)). De esta forma
	\[f(x) = \left\{ \begin{array}{lcc} 1 & si & S(q_1, q_2)\geq t
		\\ 0 & si & S(q_1, q_2) < t
	\end{array} \right..\]

	\framebreak

	Asignación de resultados utilizando un umbral de \(0.65\):
	\begin{table}[h!]
		\scriptsize
		\begin{tabularx}{\textwidth}{cccccc}
			\toprule
			\textbf{question\_id\_1} & \textbf{question\_id\_2} & \textbf{question\_1} & \textbf{question\_2}  & \textbf{similarity} & \textbf{equal} \\
			\midrule
			question\_10             & question\_11             & contenido            & contenido            & 0.857  & 1              \\
			question\_20             & question\_21             & contenido            & contenido            & 0.368  & 0              \\
			\bottomrule
		\end{tabularx}
		\label{tab:umbral-validacion-1}
	\end{table}

	Asignación de resultados utilizando un umbral de \(0.9\):
	\begin{table}[h!]
		\scriptsize
		\begin{tabularx}{\textwidth}{cccccc}
			\toprule
			\textbf{question\_id\_1} & \textbf{question\_id\_2} & \textbf{question\_1} & \textbf{question\_2}  & \textbf{similarity} & \textbf{equal} \\
			\midrule
			question\_10             & question\_11             & contenido            & contenido            & 0.857 & 0              \\
			question\_20             & question\_21             & contenido            & contenido            & 0.368 & 0              \\
			\bottomrule
		\end{tabularx}
		\label{tab:umbral-validacion-2}
	\end{table}

	\begin{center}
		¿Cómo elegimos los valores de umbral \(t\) tienen \textbf{mejor rendimiento}? \\
		\bigskip
		\textbf{Proceso iterativo} para identificar el valor de \textbf{umbral óptimo} \(t*\).
	\end{center}
\end{frame}