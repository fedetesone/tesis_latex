\section{Conclusiones}

\subsection{Contribuciones realizadas}
\begin{frame}
	\frametitle{Contribuciones realizadas}
	\textbf{Contribuciones realizadas}
	\bigskip

	\begin{itemize}
		\item Se diseñó un método que utiliza una medida de similaridad de texto confiable y efectiva entre preguntas de un sitio de CQA.
		\bigskip
		\item Se diseñó y desarrolló una arquitectura de software de procesamiento distribuido con un enfoque Big Data.
	\end{itemize}
\end{frame}

\subsection{Futuras investigaciones}
\begin{frame}
	\frametitle{Futuras investigaciones}
	El presente trabajo sirve como estado del arte para las siguientes lineas de investigación/desarrollo:
	\bigskip

	\begin{footnotesize}
		\footnotesize
			\begin{itemize}
				\item Continuar con el desarrollo para lograr un RS operativo en su totalidad utilizando el método propuesto basado en ensamble de clustering y similaridad entre ítems.
				\item Elaborar una arquitectura Big Data adaptable que mejore y optimice el funcionamiento de algunos aspectos.
				\item Utilizar los aprendizajes obtenidos en otros tipos de sitios donde se puedan aplicar RS basados en texto, tales como sitios de e-commerce, portales académicos o redes sociales.
				\item Continuar el desarrollo para crear un framework adaptable a distintas técnicas de distancias de texto.
				\item Crear y estructurar información para instituciones de ciencia, tecnología, innovación y desarrollo con el objetivo de construir insumos para el diseño, implementación, ejecución y evaluación de políticas públicas y educativas.
			\end{itemize}
	\end{footnotesize}
\end{frame}