\section{Fundamentación}

\subsection{Motivación de la tesis}
\begin{frame}{Motivación de la tesis}
	\textbf{Motivación}
	\bigskip
	\begin{itemize} [<+>]
		\item Las medidas de similaridad del estado del arte tienen conocidos problemas.
		\item Creación de un método novedoso que combine medidas de similaridad existentes, que pueda aplicarse como entrada para un RS.
		\item Es difícil identificar un algoritmo de Clustering que pueda manejar todos los tipos de formas y tamaños de cluster.
		\item Arquitectura de software que soporte el procesamiento del método propuesto de una forma eficiente y escalable.
	\end{itemize}
\end{frame}

\begin{frame}{Importancia científico-tecnológica}
	\textbf{Importancia científico-tecnológica}
	\bigskip
	\begin{itemize} [<+>]
		\item Mejorar recomendaciones en sitios de CQA.
		\item Explorar desafíos tecnológicos que sirvan como fuente de futuras investigaciones o desarrollos de software.
		\item Aplicación de los conocimientos obtenidos en este trabajo para mejorar otros sitios basadas en busquedas textuales.
	\end{itemize}
\end{frame}

\begin{frame}{Formación de recursos humanos}
	\textbf{Formación de recursos humanos}
	\bigskip
	\begin{itemize} [<+>]
		\item Capacitar a grupos de estudiantes de la UTN FRRo.
		\item Presentación en congresos tales como AGRANDA, CONAIISI, o RecSys.
		\item Elaborar material de estudio relacionado con la temática para materias de grado, cursos, o disertaciones.
	\end{itemize}
\end{frame}