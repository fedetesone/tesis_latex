\abstract

Los \textit{Sistemas de Recomendación} (Recommender Systems o RS) tienen la tarea de recomendar ítems a los usuarios de un sitio o aplicación. Los mismos pueden ser aplicados a sitios de preguntas y respuestas colaborativas, llamados \textit{Community Question Answering} (CQA por sus siglas en inglés) y las preguntas que realizan los usuarios de la aplicación pueden considerarse como los ítems a recomendar. En este trabajo, son de interés las preguntas pendientes de ser respondidas, ya que la tarea de recomendar otras preguntas similares que hayan sido formuladas por otros usuarios y tengan la respuesta deseada, puede ser realizada por un RS, minimizando así el tiempo en que un usuario puede encontrar lo que estaba buscando.

\bigskip Un buen RS debería utilizar una medida de \textit{similaridad} confiable entre preguntas, por lo cual proponemos crear una nueva medida combinada de distancia para textos a través de un método de ensamble de clustering basado en acumulación de evidencias, utilizando una arquitectura Big Data. Para este fin, dispondremos de un conjunto de datos de pares de preguntas reales, extraídos del sitio web Quora. Se realizará un análisis comparativo entre el método de ensamble de clustering y las medidas de similaridad utilizadas como punto de partida del mismo.

\bigskip Este tipo de enfoque es necesario para trabajar con grandes conjuntos de datos y así recuperar, analizar y procesar los mismos con precisión, variabilidad y velocidad, con el propósito de encontrar una medida de similaridad que pueda presentarse como una alternativa a las actuales en términos de mejorar la experiencia del usuario en sitios de CQA, mejorar las medidas de rendimiento y reducir las probabilidades de error en la búsqueda de preguntas similares.

\bigskip

\noindent\textbf{Palabras clave:} Community Question Answering, Recommender Systems, Big Data, Clustering, Clustering Ensemble, Evidence Accumulation, Text Similarity.