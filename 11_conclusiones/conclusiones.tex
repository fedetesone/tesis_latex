\chapter*{Capitulo 7}\label{ch:conclusiones}
\addcontentsline{toc}{chapter}{Capitulo 7. Conclusiones}

\section{Conclusiones}
\subsection{Contribuciones realizadas}
A lo largo de este trabajo de Tesis se realizaron diversos aportes relativos a la propuesta de un nuevo método de similaridad de texto basado en ensamble de clustering, una propuesta de arquitectura de software para el procesamiento de datos y su futura aplicación en sistemas de recomendación de tiempo real. El objetivo final del mismo es la aplicación de estos aportes en un sistema de recomendación de un sitio de Community Question Answering (CQA) para agregar la capacidad de recomendar preguntas similares a una pregunta dada, y así minimizar el tiempo en que un usuario puede encontrar la respuesta que necesita.

\bigskip Como primer aporte, se diseñó un método que utiliza una medida de similaridad de texto confiable y efectiva entre preguntas de un sitio de CQA. El conjunto de datos utilizado proviene de un sitio web llamado Quora. Para el mismo, se utilizó una técnica de ensamble de clustering, utilizando como base 5 técnicas de similaridad verificadas y utilizadas ampliamente por la comunidad, generando una salida con información de similaridad entre pares de preguntas. El método, llamado EQuAL, fue basado en dos etapas principales: una generación de particiones y la generación de una matriz de coasociación que contiene la información de similaridad entre las preguntas. La primera etapa realiza un muestreo del conjunto de datos original y aplica cada uno de los métodos del estado del arte (Term Frequency, Term Frequency/Inverse Document Frequency, Word2Vec, FastText y Semantic Distance) y genera una matriz de similaridad por cada uno de ellos; para luego asignar cada una de las preguntas en un cluster, utilizando una técnica de clustering PAM (Partición Alrededor de Medoids). Adicionalmente, la segunda etapa obtiene todas las etiquetas generadas por el paso anterior para generar una matriz de coasociación mediante un algoritmo de ensamble de clustering. La matriz de coasociación contiene todas las relaciones entre preguntas individuales en forma de similaridad de forma estándar y adimensional (en un rango \([0,1]\)). La experimentación del método se implementó mediante código basado en Python y Apache Spark. Como resultado de los experimentos, se observa que el método se comporta comparativamente similar a los del estado del arte. El método EQuAL muestra muy buenos resultados para todos los tamaños de muestra en los que se realizaron experimentos de validación,  manteniendo buenos indicadores como una media de error de \(0.32286\) y una varianza media de \(0.00049\). Se encontró, también, que existe una independencia del tamaño de muestra, lo que indica que la variabilidad agregada por el método de ensamble provoca buenos resultados. Lo anterior significa que los indicadores de rendimiento obtenidos son significativamente comparables con los métodos del estado del arte, e inclusive superador en algunos casos.

\bigskip Por otro lado, y como medio posibilitador del primer aporte, se propuso un segundo aporte: una arquitectura de software de procesamiento distribuido basado en Big Data. La misma está basada en una arquitectura Hadoop que funciona sobre un cluster de computadoras y permite la ejecución de cálculos en forma distribuida. Se demostró, mediante los experimentos realizados para este trabajo de tesis, que la misma posibilita buena flexibilidad a la hora de escalar horizontalmente y de configurar una infraestructura adecuada. Esto significa que el paralelismo otorgado por la arquitectura propuesta habilita un desempeño que permitirá escalar la cantidad de datos de una forma sencilla y eficiente.

\bigskip Si bien el método fue probado mediante un desarrollo que se ejecuta de forma batch, el mismo puede ser utilizado en tiempo real para generar información de similaridad de forma eficiente para un ítem de recomendación nuevo,  proporcionando una base teórica y práctica que puede ser implementada de forma eficiente en varios aspectos de un sistema de recomendación en tiempo real basado en similaridad de texto.

\subsection{Futuras investigaciones}
A partir de las contribuciones realizadas, es posible derivar nuevas investigaciones. Considerando la generación de una medida de similaridad de texto, no solo para preguntas en sitios de CQA, sino para cualquier tipo de fragmentos de texto que se desee comparar y, teniendo la posibilidad de extrapolarlo a cualquier sitio Web, sistema o almacén de datos en general, es posible reconocer algunas líneas potenciales de investigación tales como:
\begin{itemize}
	\item Elaborar una arquitectura Big Data adaptable que mejore y optimice el funcionamiento de algunos aspectos. Por ejemplo, expandir su adaptabilidad para poder utilizarla para la aplicación de otros procesos de Clustering o algoritmos de Deep Learning.
	\item Utilizar los resultados obtenidos en otros tipos de sitios donde se puedan aplicar RS basados en texto, tales como sitios de e-commerce, portales académicos o redes sociales.
	\item Utilizar el método de ensamble de clustering y la arquitectura desarrollada para sistemas de recomendación que contengan ítems más complejos que texto, pero que puedan ser representados de forma vectorial o que sea posible el cálculo de distancia/similaridad entre ellos. Por ejemplo, construir una representación vectorial de restaurantes a recomendar, basándose en el nombre del mismo, su localización geográfica, tipo de cocina, menú disponible, entre otras características.
	\item Experimentar con el método propuesto basado en ensamble de clustering y similaridad entre ítems, desde el punto de vista de una arquitectura streaming que sea adaptable a sistemas de recomendación en tiempo real. Este cambio de enfoque, sumado al propuesto en este trabajo, habilita la posibilidad de implementar un sistema de recomendación operativo en su totalidad.
	\item Continuar el desarrollo para crear un framework adaptable a distintas técnicas de distancias de texto para que sea sencillo agregar/quitar cada una de ellas, y así facilitar la investigación en la combinación de las mismas mediante ensamble de clustering.
	\item Datos disponibles para policy makers e instituciones de ciencia, tecnología, innovación y desarrollo con el objetivo de construir insumos para el diseño, implementación, ejecución y evaluación de políticas públicas. De esta manera, se colaborará en la mejora y optimización de todos los aspectos referidos a los procesos de búsqueda (encontrar información relevante, país donde fue publicada, naturaleza -sea institucional o personal-, fuentes -bibliotecas virtuales, repositorios digitales, bases científicas, revistas científicas open source-), y sus resultados (construir datos en tiempo real, identificar referentes en los temas, establecer contactos y poder construir prácticas colaborativas). El conjunto de estas acciones, y su análisis, ofrece un panorama para diseñar y construir acciones de manera estratégica, adecuándose a fines y objetivos específicos y coyunturales.
\end{itemize}
