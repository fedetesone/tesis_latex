\documentclass[openright,final,fmstyle]{utnfrrotesis}
\usepackage[utf8]{inputenc}
\usepackage[spanish]{babel}
\usepackage[authoryear,datebegin,nonamereplace,sort&compress]{flexbib}
\usepackage{titlesec}
\usepackage{amsmath}
\usepackage{paralist} % it provides the inparaenum environment
\usepackage{pgfgantt}
\graphicspath{{imagenes/}} % Necesario para importar las figuras en PDF.
% (with an optional formatting specification in square brackets): To change the styles of the counter,
% tokens A, a, I, i, and 1 can be used in the optional argument to produce the counter with one of the styles
% \Alph, \alph, \Roman, \roman and \arabic. For example:
%
% \begin{inparaenum}[(i)]
%
% produces the labels (i), (ii), (iii) ...

%%%%%%%%%%%%%%%%%%%%%%%%%%%%%%%%%%%%%%%%%%%%%%%%%%%%%%%%%%%%%%%%%%%%%%%%%%%%%%%%%%%%%%%%%%%%%%%%%%%%%%%%
%%% paquete para comillas
%%%%%%%%%%%%%%%%%%%%%%%%%%%%%%%%%%%%%%%%%%%%%%%%%%%%%%%%%%%%%%%%%%%%%%%%%%%%%%%%%%%%%%%%%%%%%%%%%%%%%%%%
\usepackage{quotmark}

%%%%%%%%%%%%%%%%%%%%%%%%%%%%%%%%%%%%%%%%%%%%%%%%%%%%%%%%%%%%%%%%%%%%%%%%%%%%%%%%%%%%%%%%%%%%%%%%%%%%%%%%
%%% para guionado
%%%%%%%%%%%%%%%%%%%%%%%%%%%%%%%%%%%%%%%%%%%%%%%%%%%%%%%%%%%%%%%%%%%%%%%%%%%%%%%%%%%%%%%%%%%%%%%%%%%%%%%%
\usepackage{hyphenat}
\usepackage{booktabs}
\usepackage{bm}

\usepackage{multicol}
\usepackage{supertabular}
\usepackage{multirow}
\usepackage[english,vlined,linesnumbered]{algorithm2e}
\usepackage{pdflscape}
\usepackage{fullpage,graphicx}
\usepackage{rotating}
\usepackage[hidelinks]{hyperref}
\hypersetup{
	pdftitle={Tesis de Maestría en Ingeniería en Sistemas de Información},
	pdfauthor={Ing. Federico Tesone},
	pdfsubject={Desarrollo de una medida de similaridad para Sistemas de Recomendación en sitios de Community Question Answering. Análisis desde un enfoque Big Data y usando un método de ensamble de clustering},
	pdfkeywords={},
	bookmarksnumbered=true,     
	bookmarksopen=true,         
	bookmarksopenlevel=1,       
	colorlinks=false,            
	pdfstartview=Fit,
	allcolors=blue           
}

\usepackage{nowidow}
\widowpenalty=5000  
\clubpenalty=10000

\SetKwIF{If}{ElseIf}{Else}{si}{entonces}{sin\'o, si}{de lo contrario}{fin si}
\SetKwInput{KwIn}{Entrada}%
\SetKwInput{KwOut}{Salida}%
\SetKwInput{KwData}{Datos}%
\SetKwInput{KwResult}{Resultado}%
\SetKw{KwTo}{a}%
\SetKw{KwRet}{devolver}%
\SetKw{Return}{devolver}%
\SetKwBlock{Begin}{inicio}{fin}%
\SetKwRepeat{Repeat}{repetir}{hasta que}%
\SetKwFor{ForEach}{para cada}{hacer}{fin para cada}
\SetKwFor{While}{mientras}{hacer}{fin mientras}
\SetKwFor{For}{para}{hacer}{fin para}%

\setcounter{secnumdepth}{4}
\setcounter{tocdepth}{4}

%%%%%%%%%%%%%%%%%%%%%%%%%%%%%%%%%%%%%%%%%%%%%%%%%%%%%%%%%%%%%%%%%%%%%%%%%%%%%%%%%%%%%%%%%%%%%%%%%%%%%%%%
% para remediar el problema de babel de abreviaciones de teclado para la virgulilla
%%%%%%%%%%%%%%%%%%%%%%%%%%%%%%%%%%%%%%%%%%%%%%%%%%%%%%%%%%%%%%%%%%%%%%%%%%%%%%%%%%%%%%%%%%%%%%%%%%%%%%%%
\addto\shorthandsspanish{\spanishdeactivate{~}}

%para los nombres a lo largo del documento
\newcommand{\gss}{$\gamma$-AM}
\newcommand{\disc}{percentil}
\newcommand{\Disc}{Percentil}
\newcommand{\Sim}{\text{Sim}}
\newcommand{\tfill}{1mm}
\newcommand{\tffill}{-10mm}
\newcommand{\tftfill}{-6mm}
\newcommand{\fffill}{-6mm}
\newcommand{\bfill}{-10mm}
\newcommand{\clusteryea}{8}
\newcommand{\clusterara}{9}
\newcommand{\geneexampleyea}{POP2}
\newcommand{\geneexampleara}{AT2G19450}
\newcommand{\classico}{Rhee \emph{et al.} approach \cite{rhee2008use}}
\newcommand{\classic}{Rhee \emph{et al.} approach}
\newcommand{\classi}{Rhee \emph{et al.} \cite{rhee2008use}}
\newcommand{\arafull}{\emph{Arabidopsis thaliana}}
\newcommand{\ara}{\emph{A. thaliana}}
\newcommand{\yeafull}{\emph{Saccharomyces cerevisiae}}
\newcommand{\yea}{\emph{S. cerevisiae}}
\newcommand{\km}{$k$-medias}

\newcommand{\given}{\Big\vert}

\usepackage[a4paper,footskip=.5in]{geometry}

%subsubsub
\titleformat{\paragraph}
{\normalfont\normalsize\bfseries}{\theparagraph}{1em}{}
\titlespacing*{\paragraph}{0pt}{3.25ex plus 1ex minus .2ex}{1.5ex plus .2ex}

%para quitar de a una línea en algorithm2e
\makeatletter
\newcommand{\nosemic}{\renewcommand{\@endalgocfline}{\relax}}% Drop semi-colon ;
\newcommand{\dosemic}{\renewcommand{\@endalgocfline}{\algocf@endline}}% Reinstate semi-colon ;
\newcommand{\pushline}{\Indp}% Indent
\newcommand{\popline}{\Indm\dosemic}% Undent
\let\oldnl\nl% Store \nl in \oldnl
\newcommand{\nonl}{\renewcommand{\nl}{\let\nl\oldnl}}% Remove line number for one line
\makeatother

\title{Desarrollo de una medida de similaridad para Sistemas de Recomendación en sitios de Community Question Answering. Análisis desde un enfoque Big Data y usando un método de ensamble de clustering}
\author{Ing. Federico Tesone \vspace{-6mm} }
\degree{Maestría en Ingeniería en Sistemas de Información}
\advisor{Dr.} {Guillermo Leale}
\coadvisor{Dra.} {Soledad Ayala}
\institution{Universidad Tecnológica Nacional}
\faculty{Facultad Regional Rosario}

\renewcommand\thesection{\arabic{section}}

\begin{document}
	\maketitle
	
	\newgeometry{left=4.06cm,top=2.54cm,right=2.54cm,bottom=2.54cm}
	
	\abstract

\noindent Los \textit{sistemas de recomendación} (Recommender Systems o RS) tienen la tarea de recomendar ítems a los usuarios de un sitio o aplicación. Los mismos pueden ser aplicados a sitios de preguntas y respuestas colaborativas, llamados \textit{Community Question Answering} (CQA por sus siglas en inglés) y las preguntas que realizan los usuarios de la aplicación pueden considerarse como los ítems a recomendar. En este trabajo, son de interés las preguntas pendientes de ser respondidas, ya que la tarea de recomendar otras preguntas similares que hayan sido formuladas por otros usuarios y tengan la respuesta deseada, puede ser realizada por un RS, minimizando así el tiempo en que un usuario puede encontrar lo que estaba buscando.

Un buen RS debería utilizar una medida de similaridad confiable entre preguntas, por lo cual proponemos crear una nueva medida combinada de distancia para textos a través un método de ensamble de clustering basado en acumulación de evidencias, utilizando una arquitectura Big Data. Para este fin, dispondremos de un conjunto de datos de pares de preguntas reales, extraídos del sitio web Quora.

Este tipo de enfoque es necesario para trabajar con grandes conjuntos de datos y así recuperar, analizar y procesar los mismos con precisión y velocidad, con el propósito de encontrar una medida de similaridad que asegure resultados que aumenten considerablemente la experiencia del usuario en sitios de CQA, mejorar las medidas de rendimiento y reducir las probabilidades de error en la búsqueda de preguntas similares.

\bigskip

\noindent\textbf{Palabras clave:}  Community Question Answering, Recommender Systems, Big Data, Ensemble Clustering, Evidence accumulation.
	\tableofcontents % Tabla de contenidos
	\sloppy % Indica a LaTex que debe minimizar el corte de las palabras para pasar de una línea a otra.
	% \mainmatter % para reiniciar numeración.
	\newpage \chapter*{tesis}\label{ch:plan}\addcontentsline{toc}{chapter}{Plan de tesis}

	\section{Justificación del tema elegido} \label{ch:justificacion}
\noindent Los sitios de CQA brindan servicios que permiten a los usuarios formular y contestar preguntas sobre temas de cualquier índole. Miles de nuevas preguntas son subidas diariamente en sitios de CQA como Yahoo! Answers\footnote{Yahoo! Answers: \url{https://answers.yahoo.com/}. Último acceso Agosto 2018.}, Stackexchange\footnote{Stackexchange: \url{https://stackexchange.com/}. Último acceso Agosto 2018.}, Stackoverflow\footnote{Stackoverflow: \url{https://stackoverflow.com/}. Último acceso Agosto 2018.} o Quora\footnote{Quora: \url{https://www.quora.com/}. Último acceso Agosto 2018.}. Estos son portales muy populares donde los usuarios suben diariamente una cantidad importante de preguntas de varios dominios para obtener respuestas de otros usuarios de la comunidad \citep{anuyah2017can}. Del análisis de sitios de CQA, puede observarse que muchas de las preguntas no están respondidas correctamente o no tienen respuestas específicas, ya que en estas comunidades hay típicamente un pequeño número de expertos entre la gran población de usuarios \citep{yang2013cqarank}. Por lo tanto, cuando un usuario realiza una pregunta es de interés buscar si la misma ha sido formulada por otro usuario con anterioridad y que, además, tenga la respuesta deseada. Gracias a estos mecanismos, el usuario podría leer las respuestas de dicha pregunta sin tener que esperar que su pregunta sea respondida. Esto no siempre es una tarea fácil, ya que esta pregunta previamente existente en el sitio (y respondida), puede estar formulada de una manera completamente diferente en el sentido léxico. Por esta razón una correspondencia exacta (o casi exacta) no es aplicable. Consideramos el siguiente ejemplo de dos preguntas iguales: \textit{¿Cómo elijo una revista para publicar mi artículo?} y \textit{¿Dónde publico mi artículo?}\footnote{Traducción de las preguntas \textit{``How do I choose a journal to publish my paper?, Where do I publish my paper?''} extraídas desde el conjunto de datos de Quora que se utilizará en el presente trabajo de tesis \url{https://data.quora.com/First-Quora-Dataset-Release-Question-Pairs}. Último acceso Agosto 2018.}. Entre estas dos frases, existe apenas una superposición de palabras, sin tener en cuenta \textit{stopwords}\footnote{En informática, se llama \textit{stopword} a palabras que se filtran antes o después del procesamiento de datos del lenguaje natural \citep{leskovec2014mining}.}. Sin embargo, ambas preguntas tienen la misma respuesta, que referirá a revistas o sitios donde publicar un artículo científico. Con el fin de comparar dos preguntas, se establece una medida de similaridad que se puede intuir como máxima cuando son idénticas y que es inversamente proporcional a las diferencias entre ellas \citep{lin1998information}. Una medida de similaridad de texto entre preguntas basada en características léxicas no las detectaría como preguntas iguales. Esto deja en evidencia la necesidad de utilizar enfoques que además consideren características semánticas.

\bigskip A partir del análisis anterior puede decirse que la tarea de encontrar preguntas similares en sitios de CQA puede ser llevada a cabo por un RS. Los RS son herramientas de software y técnicas que proveen sugerencias de ítems que los usuarios pueden querer utilizar \citep{ricci2011introduction}. Las sugerencias relacionan varios procesos de toma de decisiones, como por ejemplo qué ítem comprar o qué música escuchar. ``Ítem'' es el término general usado para denotar lo que los RS recomiendan a los usuarios. Los ítems son objetos que pueden estar caracterizados por su valor o utilidad. El valor de un ítem puede ser positivo si el ítem es útil para el usuario y negativo si no es apropiado y, en ese caso, el usuario tomaría una mala decisión seleccionándolo. A partir de la dinámica que se construye en los RS, estos posibilitan generar recomendaciones al usuario que pueden ser personalizadas o no personalizadas. Las primeras, se basan en comportamientos del usuario o en grupos de usuarios para encontrar sugerencias adecuadas a sus preferencias; las segundas, son inherentes a los ítems que el RS sugerirá. Cada una de estas estrategias de recomendación se elabora a partir de diferentes conocimientos y datos recopilados por el sitio o sistema donde el RS esté aplicado. Ejemplos de tales aplicaciones incluyen la recomendación de libros, películas o ítems de compra \citep{adomavicius2005toward}. En particular, para los sitios de CQA, los algoritmos de recomendación se aplican principalmente a elementos de texto. Este trabajo se centrará en ese tipo de recomendaciones, que pueden estar clasificadas dentro de RS basados en contenido de texto no personalizados, ya que las mismas están basadas únicamente en la estructura sintáctica y semántica de las preguntas existentes en los sitios de CQA. Los usuarios pueden navegar esas recomendaciones. Luego, pueden aceptarlas o no y, además, proveer inmediatamente o en un paso posterior, una retroalimentación explícita o implícita.

\bigskip A consecuencia de lo expuesto anteriormente, y sumado a que se dispone de un gran conjunto de datos, se propondrá una arquitectura para utilizar Big Data con el fin de crear una medida de similaridad de texto que alimente a un RS especializado en la tarea de encontrar preguntas similares en sitios de CQA basado en análisis de contenido de texto. Este tipo de enfoque es necesario para procesar una gran cantidad de datos y, de esta manera, optimizar el procesamiento de los mismos, logrando velocidad y con la ventaja de poder aprovechar toda la variabilidad que provee un conjunto de datos de gran volumen.
Con el fin de dejar en claro el alcance de este trabajo, se toma como punto de partida el trabajo de investigación de la Universidad Tecnológica Nacional, Facultad Regional Rosario: ``Comparative Analysis on Text Distance Measures Applied to Community Question Answering Data'', el cual se centra en el Paso 1 del pipeline que se describe en la Figura 1. 
\begin{figure}
	\def\svgwidth{\linewidth}
	\input{imagenes/figura1.pdf_tex}
	\caption{Pipeline para un RS basado en contenido de CQA y en una nueva medida de similaridad.}
\end{figure}

\bigskip Este proceso descrito en tres pasos, tiene como objetivo construir un RS basado en una medida novedosa de similaridad de texto. En el Paso 1 se realiza un análisis comparativo desarrollado a partir de medidas basadas en distancia, obtenidas de análisis de texto, con el fin de evaluar RS a partir de grandes conjuntos de datos; en el Paso 2, a partir de los resultados arrojados por el Paso 1, se crea una nueva medida construyendo una matriz de similaridad basada en análisis de clustering, como una de las propuestas en los métodos \textit{Algoritmo de Particionamiento de Similitud basado en Cluster} (Cluster-based Similarity Partitioning Algorithm o CSPA) \citep{strehl33knowledge} y \textit{Clustering de Acumulación de Evidencias} (Evidence Accumulation Clustering o EAC) \citep{fred2005combining}. Esta medida intenta mejorar la calidad de salida para una representación de similaridad basada en texto; por último, en el Paso 3 se debe aplicar la matriz de distancias obtenida en el Paso 2 en un RS basado en contenido, con el fin de evaluar su eficacia en sitios de CQA.

\bigskip Para tomar dimensión del volumen de datos que es necesario manejar con este enfoque basado en clustering, si, por ejemplo, tomáramos el conjunto de datos Quora (404301 pares de preguntas, es decir, 808602 preguntas totales), y quisiéramos generar una sola matriz de distancias cruzando absolutamente todas las preguntas entre sí, estaríamos calculando $\frac{n(n+1)}{2} = 326919001503$ distancias, donde $n = 808602$ y el resultado es la cantidad de elementos en la triangular superior. Esta matriz considera sólo una de las distancias de similaridad de texto del estado del arte, por lo cual si deseamos combinar varias medidas mediante un método de ensamble, deberíamos generar al menos una matriz por cada distancia del estado del arte, y luego usar las mismas para aplicar EAC, por lo cual estaríamos generando un número total de cálculos considerablemente mayor al que puede procesar una computadora clásica. Es necesario además tener en cuenta que las distancias pueden llegar a considerar características diversas entre sí, como morfología, sintaxis y semántica de los textos, lo cual añade complejidad y variedad al volumen considerado. Esto conlleva considerar una arquitectura Big Data, optimización de código y técnicas de ejecución paralela entre gran número de núcleos de procesamiento. Con respecto al espacio de almacenamiento, debe notarse que las matrices requieren doble precisión para la representación interna de cada uno de sus elementos (distancias), es decir 750 KB cuando están almacenados en un archivo, por lo cual, si consideramos la matriz ejemplificada anteriormente, necesitaríamos aproximadamente 12 TB para almacenarla, por lo cual, es necesario un esquema de almacenamiento optimizado para la implementación de este método. Con respecto a la velocidad de procesamiento, pruebas preliminares en un procesador potente brindan una estimación de alrededor de 3 años para completar el procesamiento. Con el problema planteado de esta forma, es indispensable aplicar un enfoque de Big Data para satisfacer los requerimientos de volumen, variedad y velocidad que requiere el contexto de análisis, de tal forma de brindar resultados veraces, y de esa forma cumplir con las premisas de las “V” del Big Data\footnote{Las “V” del Big Data refieren a Volumen, Variedad y Velocidad. También se consideran los conceptos de Valor y Veracidad con respecto al resultado de la aplicación del enfoque Big Data \citep{gandomi2015beyond}.}. 

\bigskip Este trabajo de tesis, apuntará entonces a construir una medida de similaridad novedosa desde un enfoque Big Data, tal como se describe en el Paso 2 del pipeline, por lo cual es necesario crear un nuevo software basado en una arquitectura y patrones de Big Data, tomando como punto de partida el desarrollo del estado de arte.

\subsection{Importancia científico-tecnológica}
\noindent Con respecto a los sitios de CQA en particular, la importancia de este trabajo radica tanto en la posibilidad de construir un RS que, desde el punto de vista del usuario, reduzca tanto el tiempo promedio en que se encuentra una respuesta como, a su vez, mejore la experiencia del sitio. En este sentido, en la mayoría de los casos no será necesario escribir múltiples versiones de la misma pregunta y los lectores podrán encontrar rápidamente la respuesta que están buscando.
Por otro lado, se evitará que se creen preguntas duplicadas, lo que significaría un aumento considerable en la calidad y cantidad de la base de conocimiento del sitio, construyendo una relación biunívoca entre una pregunta y su correspondiente respuesta. Además, se logrará optimizar el tamaño de la base de datos, la integridad de la información, mejorar la velocidad en búsquedas e incrementar de la satisfacción y fidelidad del usuario \citep{ricci2011introduction}.

\bigskip Por último, el resultado de la presente investigación también puede ser utilizado para sitios que son fuente de consulta para diversos investigadores dentro del ámbito de la Universidad Tecnológica Nacional, Facultad Regional Rosario, tales como bibliotecas virtuales o foros de consulta para investigaciones científicos-tecnológicos que incluyan I+D+i. Esto permitiría no solo conocer los intereses de otros investigadores y en qué términos formularon sus interrogaciones, sino también conocer quién o quiénes elaboraron las respuestas a dichas preguntas y a qué campo disciplinar pertenecen. 
\subsubsection{Futuras investigaciones}
\noindent Considerando la generación de una medida de similaridad de texto, no solo para preguntas en sitios de CQA, sino para cualquier tipo de fragmentos de texto que se desee comparar y, teniendo la posibilidad de extrapolarlo a cualquier sitio Web, sistema o almacén de datos en general, es posible reconocer algunas líneas potenciales de investigación tales como:
\begin{itemize}
	\item Elaborar una arquitectura Big Data adaptable que mejore y optimice el funcionamiento de algunos aspectos. Por ejemplo, poder usarla para la aplicación de otros procesos de Clustering o algoritmos de Deep Learning.
	\item Utilizar los resultados obtenidos en otros tipos de sitios donde se puedan aplicar RS basados en texto.
	\item Análisis disponible para policy makers e instituciones de ciencia, tecnología, innovación y desarrollo para mejorar y optimizar todos los aspectos referidos a los procesos de búsqueda (encontrar información relevante, país donde fue publicada, fuente -sea institucional o personal-, datos para establecer contactos y poder construir prácticas colaborativas). 
\end{itemize}
\subsubsection{Formación de recursos humanos}
\noindent El presente trabajo de tesis, en relación a la formación de recursos humanos, tiene los siguientes objetivos: 
\begin{itemize}
	\item Capacitar a un grupo de estudiantes de la UTN FRRo, con elementos para la investigación y desarrollo en aplicaciones Big Data.
	\item Realizar grupalmente conocimiento científico, con base teórica sustentable y ejemplos empíricos de aplicaciones funcionales, para presentar en congresos tales como AGRANDA\footnote{AGRANDA: Simposio Argentino de GRANdes DAtos.}, CONAIISI\footnote{CONAIISI: Congreso Nacional de Ingeniería Informática - Sistemas de Información.}, o RecSys\footnote{RecSys: The ACM Conference Series on Recommender Systems.}; o eventos relacionados con Ingeniería en Sistemas de Información.
	\item Elaborar material de estudio relacionado con la temática de la minería de datos para materias de grado y/o posgrado.
	\item Lograr que los estudiantes puedan entender cómo está formado en la actualidad el estado del arte sobre el presente tema y que esto sirva de base para futuras investigaciones en UTN FRRo, ya sean proyectos de investigación, tesis de maestría o de doctorado.
	\item Desarrollar insumos para el armado de cursos tanto de formación académica como abiertos a la comunidad relacionados con Big Data o análisis de texto.
\end{itemize}
\subsection{Importancia socio-económica}
\noindent El tema posee una importancia social y económica que permitiría construir contactos y alianzas -económicas, académicas y de naturaleza mixta- con instituciones extranjeras. En otras palabras, a nivel social podría utilizarse para actividades de investigación y en los diferentes niveles educativos, según se adecúen las explicaciones y el vocabulario utilizado, las actividades y los diversos usos. Buscar información en bibliotecas digitales y virtuales, en bases de datos científicos, repositorios digitales, foros especializados de temáticas específicas y diversas o plataformas educativas, son algunos de los sitios donde los RS pueden ser utilizados y aplicados para determinadas actividades cognitivas. Además, el tema puede ser complementado en un futuro con otras líneas de investigación, tales como políticas educativas para la alfabetización mediática, análisis y datos online, fuentes abiertas o la relación entre tecnología y democracia. Estas líneas, de prioridad en la agenda de ciencia y tecnología de países del primer mundo, están siendo desarrolladas entre academia, instituciones de gubernamentales y policy makers, de manera interdisciplinaria y con el objetivo de mejorar las herramientas que poseen los ciudadanos en relación a la cultura digital y sus mecanismos de funcionamiento estrictamente técnicos y los aspectos culturales que la atraviesan.  
Por otro lado, en el marco económico, los resultados de la presente investigación posibilitarán continuar con futuras indagaciones referidas a la temática y diseñar/construir nuevas herramientas de software adecuadas en función de ciertos usos y usuarios específicos. Estas acciones permitirían llevar adelante: nuevos proyectos de investigación interdisciplinarios y con subsidios de naturaleza mixta (público-privada), formación de formadores, pequeños emprendimientos para estudiantes avanzados y/o la postulación a becas de formación (nacionales e internacionales).
	\newpage \section{Fundamentación}\label{ch:fundamentacion}
\subsection{Estado del arte}
\subsubsection{Sitios de CQA}
\noindent Los servicios de Community Question Answering CQA, son un tipo especial de servicios \textit{Question Answering} (QA), los cuales permiten a los usuarios registrados responder a preguntas formuladas por otras personas. Los mismos atrajeron a un número creciente de usuarios en los últimos años \citep{li2010routing}. Una pregunta formulada en Quora, y respondida por su fundador y CEO, Adam D\textsc{\char13}Angelo, revela que el sitio recibe más de 200 millones de visitantes únicos mensualmente (información actualizada a Junio de 2017), lo que denota la popularidad de este tipo de portales\footnote{Pregunta formulada en el sitio Quora \textit{``How many people use Quora?''}: \url{https://www.quora.com/How-many-people-use-Quora-7}. Último acceso Agosto 2018.}. Desde la creación de este tipo de servicios, se han aplicado diferentes técnicas de software para que los usuarios encuentren respuestas a sus preguntas en el menor tiempo posible y aprovechar al máximo el valor de las bases de conocimiento, por ejemplo, un framework para predecir la calidad de las respuestas con características no textuales \citep{jeon2006framework}, incorporar información de legibilidad en el proceso de recomendación \citep{anuyah2017can}, encontrar a los expertos apropiados \citep{li2010routing} o recomendar la mejor respuesta a una pregunta dada, entre otros. Sin embargo, el mecanismo existente en el cual se responden las preguntas en los sitios de CQA todavía no alcanza a satisfacer las expectativas de los usuarios por varias razones: 
\begin{inparaenum}[(1)]
	\item baja probabilidad de encontrar al experto: una nueva pregunta, en muchos casos, puede no encontrar a la persona con la habilidad de responderla de manera correcta, resultando en respuestas tardías y que distan de ser óptimas; 
	\item respuestas de baja calidad: los sitios de CQA suelen contener respuestas de baja calidad, maliciosas y spam. Estas suelen recibir baja calificación de los miembros de la comunidad; \item preguntas archivadas y poco consultadas: muchas preguntas de los usuarios son similares. Antes de formular una pregunta, un usuario podría beneficiarse de buscar ya formuladas, y por consiguiente, sus respuestas \citep{yang2013cqarank}.
\end{inparaenum}

\subsubsection{Sistemas de recomendación}
\noindent Es muy frecuente tener que tomar decisiones sin la suficiente experiencia personal sobre las alternativas disponibles. En la vida cotidiana, confiamos en recomendaciones de otras personas ya sea de boca en boca o cartas de recomendación, reseñas de libros y películas o encuestas generales. Los sistemas de recomendación asisten este proceso natural en el ámbito de los sistemas de información \citep{resnick1997recommender}. El primer RS, \textit{Tapestry} \citep{goldberg1992using}, fue un sistema experimental de correo electrónico destinado a resolver el problema de manejar grandes cantidades de emails filtrando según cuán interesantes son los documentos, utilizando un enfoque basado en el contenido de los mismos y también filtros colaborativos, lo que después se denominaría RS no personalizados y personalizados por \citeauthor{ricci2011introduction} en el año \citeyear{ricci2011introduction}. Se ha trabajado mucho en mejorar y desarrollar nuevos enfoques con respecto a RS en los últimos años, y el interés en esta área sigue vigente debido a la abundancia de aplicaciones prácticas en las cuales es necesario ayudar a los usuarios a lidiar con la sobrecarga de información\footnote{El concepto de sobrecarga de información, del inglés \textit{information overload}, hace referencia a cuando los usuarios reciben demasiada información, por lo cual, la precisión en sus decisiones empieza a decrecer \citep{eppler2004concept}.} y proveer recomendaciones personalizadas, contenidos y servicios. Sin embargo, a pesar de todos estos avances, la generación actual de RS todavía requiere mejoras para que los métodos de recomendación sean más efectivos y aplicables a una gama más amplia de sistemas y/o sitios. Aunque las raíces de los RS se remontan a trabajos en ciencia cognitiva \citep{rich1979user}, teoría de aproximación \citep{powell1981approximation}, recuperación de información \citep{salton1989automatic}, ciencias de las predicciones \citep{armstrong2001principles}, ciencias de la gestión \citep{murthi2003role} y también al modelado de la elección de consumidor en marketing \citep{lilien1992marketing}, los RS recién surgen como un área de investigación independiente en la década de 1990, cuando los investigadores comenzaron a centrarse en problemas de recomendación que se basan específicamente en calificaciones \citep{adomavicius2005toward}. En su formulación más común, el problema de recomendación se reduce a estimar calificaciones para los ítems que no han sido vistos por un usuario.

\subsubsection{Big Data}
\noindent Al igual que todos los términos que surgen a partir de avances tecnológicos, no existe un consenso claro de cómo definir \textit{Big Data}. \cite{manyika2011big} definen este concepto como los conjuntos de datos cuyo tamaño está más allá de la habilidad de las herramientas software de base de datos para capturar, almacenar, gestionar y analizar. Nótese que esta definición es agnóstica del tamaño del conjunto de datos, y no define un tamaño mínimo del mismo, sino que, asume que la tecnología avanza constantemente como así también las herramientas, por lo cual, la definición se “mueve” con el tiempo. Por otro lado, también es interesante tomar otra arista en la definición de este concepto. La consultora Gartner en su sitio web\footnote{Concepto de Big Data en el glosario de Gartner: \url{https://www.gartner.com/it-glossary/big-data}. Último acceso Agosto 2018.} lo define como “Big Data son activos de información caracterizados por su alto volumen, velocidad y variedad que demandan formas innovadoras y rentables de procesamiento de información para mejorar la compresión y la toma de decisiones”, haciendo énfasis en la multiplicidad de características de Big Data. 

\bigskip El comienzo de sobrecarga de información, recientemente mencionado, data del año 1880, cuando el censo de los Estados Unidos tarda 8 años en tabularse. Ante esta situación Herman Hollerith inventó la máquina tabuladora eléctrica basada en tarjetas perforadas\footnote{Herman Hollerith, US Census Boureau: \url{https://www.census.gov/history/www/census_then_now/notable_alumni/herman_hollerith.html}. Último acceso Agosto 2018.}. El censo en 1890 fue un éxito rotundo e, incluso, la máquina que él diseñó fue  utilizada para los censos de Canadá, Noruega y Austria al año siguiente.

\bigskip En el año 1941, los científicos empiezan a utilizar el término “explosión de la información”, que fuera  citado en el periódico The Lawton Constitution\footnote{The Lawton Constitution: \url{http://www.swoknews.com/}. Último acceso Agosto 2018.}, haciendo alusión a la dificultad de administrar toda la información disponible. Gradualmente, se identificaron avances concretos en materia de procesamiento de datos y criptografía, motivados particularmente por los sucesos bélicos de la época. Un ejemplo es el dispositivo llamado Colossus \citep{copeland2004colossus} que buscaba e interceptaba mensajes a una tasa de miles de caracteres por segundo. Unos años más tarde, en 1951 el concepto de \textit{memoria virtual} es introducido por el físico alemán Fritz-Rudolf Güntsch, como una idea que trataba el almacenamiento finito como infinito.

\bigskip A partir de la década del 80', los avances tecnológicos, especialmente en sistemas MRP (planificación de recursos de fabricación), permitieron nuevas formas de organizar, almacenar y generar datos. En este sentido, IBM se destaca y define una arquitectura para los informes y análisis de negocio (EBIS)\footnote{Acrónimo para EMEA (Europe, Middle East and Africa) Business Information System.}, que se convierte en la base del almacenamiento de datos en forma centralizada para usuarios finales \citep{devlin1988architecture}; es decir, el \textit{data warehousing}. Hacia finales de los 80’, Tim Berners-Lee, inventa la \textit{World Wide Web} \citep{berners1992world}. Invento que implicaría el impacto más grande hasta la actualidad con respecto a la generación, identificación, almacenamiento y análisis de grandes volúmenes de datos de diversa naturaleza.

\bigskip El inicio de los años 90’ marcan un antes y un después en lo relativo al tratamiento y almacenamiento de datos. El crecimiento tecnológico fue explosivo, tal es así que el almacenamiento digital empieza a ser más conveniente y rentable que el papel para almacenar datos \citep{morris2003evolution}. Es en 1990 cuando surgen las plataformas de \textit{Business Intelligence} (BI) y los rediseños de software al estilo \textit{Enterprise Resource Planning} (ERP). En este contexto, \cite{cox1997application} afirman que el crecimiento de la cantidad de datos que debe manejar un sistema de información empieza a ser un problema en materia de almacenamiento y visualización de los datos, situación que denominaron como “el problema del Big Data”. Así, 1997 es un año clave, en el que se realizan un gran porcentaje de estudios y publicaciones que se enfocan en averiguar cuánta información hay disponible a nivel mundial y su crecimiento\footnote{Michael Lesk publica “How much information is there in the world?” (1997): \url{http://www.lesk.com/mlesk/ksg97/ksg.html}. Último acceso Agosto 2018.}, y, en consecuencia, se estima que el crecimiento de Internet es aproximadamente del 100\% anual y que superaría el tráfico de voz para el año 2002 \citep{coffman1998size}.

\bigskip En el año 2001, se introduce el concepto de \textit{las 3 V’s: Volumen, Velocidad y Variabilidad de los datos} \citep{laney20013d} fundantes sobre la temática y que sería mundialmente aceptado una década más tarde. Por otro lado, también, en 2001 aparece el concepto de \textit{Software como un Servicio} (SaaS) \citep{hoch2001software}, un modelo disruptivo de servicios centralizados y acceso a los mismos mediante clientes finos (típicamente exploradores web), dando la posibilidad del escalamiento horizontal de sistemas de información y la generación de estándares de comunicación. Esta situación provocó que empresas como Oracle\footnote{Oracle: \url{https://www.oracle.com}. Último acceso Agosto 2018.}, SAP\footnote{SAP: \url{https://www.sap.com}. Último acceso Agosto 2018.}, y Peoplesoft\footnote{Peoplesoft: adquirida por Oracle en Enero de 2005.} empiecen a centrarse en el uso de servicios web, permitiendo así la generación de datos en forma masiva por usuarios finales. Así, en 2006, nace Apache Hadoop\footnote{Apache Hadoop: \url{http://hadoop.apache.org/}. Último acceso Agosto 2018.}, una solución de código abierto que permite el procesamiento en paralelo y distribuido de enormes cantidades de datos en forma escalable. Posteriormente, en 2008, se empieza a pensar al Big Data como la mayor innovación en informática en la última década, ya que ha transformado la forma en que los motores de búsqueda acceden a la información, las actividades de las compañías, las investigaciones científicas, la medicina, y las operaciones de defensa e inteligencia de los países, entre otras tantas actividades. Más aún, se ha comenzado a ver su potencial para recopilar y organizar datos en todos los ámbitos de la vida cotidiana \citep{bryant2008big}, tales como redes sociales, estadísticas deportivas o avances médicos y genéticos.

\subsection{La propuesta}
\noindent La calidad de un RS tiene una relación directa con los datos de entrada que se han generado para alimentarlo. Con el fin de generar una entrada basada en medidas de similaridad, es necesaria la comparación de preguntas formuladas en sitios de CQA usando técnicas de análisis de texto.
Un problema importante inherente al análisis de texto, con el fin de cuantificar relaciones entre distintos fragmentos o documentos, es encontrar la medida apropiada de representación. Algunas medidas de similaridad resultantes de algoritmos de recomendación en análisis de texto, son obtenidas mediante algoritmos puramente sintácticos, léxicos, tales como: \textit{Term Frequency} \citep{salton5mcgill}, \textit{Term Frequency/Inverse Document Frequency} \citep{baeza1999modern}, basados en ventanas como \textit{FastText} \citep{joulin2016fasttext} o \textit{Word2Vec} \citep{mikolov2013efficient}, o semánticos, como \textit{Semantic Distance} \citep{li2006sentence}. Los algoritmos puramente sintácticos como Term Frequency y Term Frequency/Inverse Document Frequency tienen conocidos problemas, tales como ser invariantes respecto al orden de las palabras o ser sensibles a stopwords, por lo cual, necesitan un gran trabajo de pre-procesamiento. FastText y Word2Vec están fuertemente afectados en el orden en el cual aparecen las palabras. Adicionalmente, ninguna de estas técnicas tiene en cuenta la semántica de las palabras y sus relaciones, como si lo hace Semantic Distance. Sin embargo, esta última técnica, según el trabajo tomado como estado del arte, tampoco alcanza medidas de rendimiento apropiadas para un RS en un sitio de CQA.

\bigskip Resultados experimentales de medidas de rendimiento obtenidas en el trabajo que se toma como punto de partida de esta tesis, arrojan entre un 66\% y un 68\% de precisión y entre un 32\% y un 33.5\% de error usando cada uno de los algoritmos de recomendación descritos anteriormente. Estos valores son considerados prometedores, ya que las medidas de rendimiento son consistentes en todos los algoritmos seleccionados, lo que denota que la complejidad inherente del conjunto de datos no afecta significativamente la performance de cada uno de ellos. Además, los resultados de prueba no varían significativamente con respecto a los resultados de validación. Dicho esto, la motivación de este trabajo de tesis, así como el de las futuras líneas de investigación, es la creación de una medida de similaridad de texto novedosa que sirva como entrada para un RS aplicable a sitios de CQA. Para tal fin, se crearán matrices de distancias, usando cada una de las preguntas del conjunto de datos en estudio, para luego combinarlas usando métodos de ensamble de clustering, ya que, como existen cientos algoritmos de clustering, es difícil identificar un solo algoritmo que pueda manejar todos los tipos de forma y tamaños de cluster, e incluso, decidir qué algoritmo sería el mejor para un conjunto de datos en particular. \cite{fred2005combining} introducen el concepto de clustering de acumulación de evidencias, que mapea las particiones de datos individuales en un ensamble de clustering dentro de una nueva medida de similaridad entre patrones, sumarizando la estructura entre-patrón percibido de esos clusters. La partición de datos final es obtenida aplicando el método \textit{single-linkage} a la nueva matriz de similaridad. El resultado de este método muestra que, la combinación de algoritmos de clustering “débiles” como el \textit{k-means}, pueden conducir a la identificación de clusters subyacentes verdaderos con formas, tamaños y densidades arbitrarias. Por lo cual, teniendo en cuenta diferentes particiones creadas con el método de ensamble desde los mismos datos originales, objetos de textos similares probablemente pertenecerán al mismo cluster.

\bigskip El desarrollo de matrices de similaridad para la aplicación del EAC que se utilizarán como entrada de RS, claramente implica manipular un gran volumen de datos complejos y realizar un elevado número de cálculos en tiempo real, ya que nos estamos refiriendo a conjuntos de datos cuyo tamaño supera la capacidad de las herramientas tradicionales de bases de datos de recopilar, almacenar, gestionar y analizar la información \citep{de2016mineria}. Esto implica, en principio, considerar una \textit{matriz de co-asociación} entre elementos realizando varias series de corridas y aplicación de clustering. Cada una de esas series es basada en una de las medidas de similaridad. El resultado será un un valor adimensional e insesgado que puede mejorar la representación para la estructura subyacente de relaciones de texto. El volumen de datos ejemplificado en las secciones anteriores, deja expuesta necesidad de investigar y desarrollar el tema aquí propuesto con un enfoque distinto al tradicional. Esto implica realizar un muestreo aleatorio de pares de preguntas dentro de una arquitectura que permita generar la mayor cantidad posible de subconjuntos de datos extraídos aleatoriamente. Además, posibilitará que cada uno de ellos sea lo más grande posible para aprovechar toda la variedad de los datos. Mientras más se aproveche la variedad de los datos (más subconjuntos de datos y de mayor tamaño), más afectará negativamente en el tiempo de de procesamiento, razones por las cuales se hace necesaria una arquitectura e infraestructura preparada para tal desafío, con una \textit{velocidad} que haga posible obtener resultados en un período de tiempo razonablemente corto. Un enfoque Big Data es imprescindible  para este tipo de procesamiento de datos. No solo se desea hacer referencia a la gran cantidad y complejidad de los datos, sino también a las herramientas utilizadas para procesarlos y las posibilidades de extraer conocimiento útil a partir del análisis de los mismos. Estos procesos y herramientas son el eje central de la definición de Big Data de la consultora Gartner (2012), la cual hace foco en los procesos para manipular activos de gran volumen y variedad con una gran velocidad. Por lo cual, si bien Big Data se refiere a estos activos, demanda formas innovadoras y efectivas de procesarlos, que habiliten tomas de decisiones y automatización de procesos.

\bigskip Por todos estos motivos, se propone la elaboración de un nuevo método y una arquitectura que lo soporte, que genere una entrada de datos correctamente estructurada para RS y que pueda ser utilizada da en sitios de CQA, de una forma eficiente y eficaz.

	\newpage \section{Objetivos del trabajo de tesis}\label{ch:objetivos}
\subsection{Objetivo general}
\noindent El presente trabajo de investigación tiene como objetivo construir una arquitectura Big Data que se aplique a grandes conjuntos de datos de preguntas de CQA y permita encontrar nuevas medidas de similaridad entre textos que puedan ser utilizadas en sistemas de recomendación.
\subsection{Objetivos específicos}
\noindent Se detallan a continuación, los objetivos específicos que son necesarios para lograr el objetivo principal.
\begin{enumerate}
	\item Identificar medidas de similaridad de texto existentes y un método efectivo de aplicación de las mismas en grandes volúmenes de datos.
	\item Diseñar y desarrollar una arquitectura Big Data para cálculo de similaridad en grandes matrices, que requerirá nuevas estrategias para recolectar, procesar y manejar grandes volúmenes de datos.
	\item Encontrar nuevas medidas de similaridad de texto que sean mejores que las existentes respecto al manejo del volumen, variedad, velocidad y veracidad inherentes a grandes volúmenes de datos.
	\item Mejorar las medidas de desempeño y error en sistemas de recomendación del estado del arte.
\end{enumerate}
	\newpage \section{Metodología de desarrollo} \label{ch:metodologia}
\subsection{Hipótesis de trabajo}
\noindent A partir del relevamiento del estado del arte se infiere que las medidas de rendimiento obtenidas en las entradas de RS no son los suficientemente eficientes para mejorar la experiencia de usuario y reducir las probabilidades de error en sitios CQA.

\bigskip Por tal motivo, y como respuesta a la hipótesis planteada, se presentará un método basado en una arquitectura Big Data que posibilite aplicar ensamble de clustering a grandes conjuntos de datos y lograr medidas de rendimiento superadoras.

\subsection{Procedimiento de desarrollo}
\noindent Este trabajo comenzará con una búsqueda de material científico relacionado a RS en general, RS no personalizados basados en análisis de texto, su aplicación en sitios de CQA, un análisis de algoritmos de comparación de texto del estado del arte y su aplicación a grandes volúmenes de datos mediante métodos de ensamble de clustering y, también, una evaluación de arquitecturas de software adecuadas para un enfoque Big Data e infraestructuras acordes. Esto puede ser realizado mediante sitios o librerías digitales, tales como Google Scholar\footnote{Google Scholar: \url{https://scholar.google.com.ar}. Último acceso Agosto 2018.}, IEEExplore Digital Library\footnote{IEEExplore Digital Library: \url{http://ieeexplore.ieee.org}. Último acceso Agosto 2018.}, ScIELO\footnote{SciELO: \url{http://www.scielo.org}. Último acceso Agosto 2018.}, Harvard Library\footnote{Harvard library: \url{https://library.harvard.edu}. Último acceso Agosto 2018.} o el portal del CAICYT-CONICET\footnote{Centro Argentino de Información Científica y Tecnológica del CONICET: \url{http://www.caicyt-conicet.gov.ar/sitio}. Último acceso Agosto 2018.}, entre otros.
Definida la hipótesis correctamente y el plan de trabajo, se iniciará el desarrollo de un software de código abierto partiendo del proyecto "text comparison"\footnote{Repositorio GitHub: \url{https://github.com/Departamento-Sistemas-UTNFRRO/text_comparison}.} perteneciente al repositorio Git del departamento de Ingeniería en Sistemas de Información de la UTN FRRo. Se importarán las piezas de software del código del proyecto del estado del arte recientemente mencionado para usarlas mediante un enfoque Big Data, con nuevas herramientas basadas en Cloud Computing, Hadoop y una arquitectura de software completamente nueva que optimice este tipo de desarrollo. Una vez que se inicie el desarrollo del proyecto, serán evaluadas distintas opciones de herramientas y entornos que se utilizarán, Esto incluye:
\begin{itemize}
	\item Lenguajes de programación y librerías inherentes al mismo.
	\item Almacenes de datos, frameworks y proyectos de terceros que puedan ser incorporados en la arquitectura Big Data.
	\item Arquitecturas de software, patrones, modelos y buenas prácticas.
	\item Infraestructura: local, distribuida en una red de computadoras físicas, o distribuida y virtualizada en la nube.
\end{itemize}

Paralelamente al desarrollo, se identificará y documentará la nueva solución de acuerdo con los requerimientos de la Maestría en Ingeniería en Sistemas de Información, a fin de obtener un trabajo de investigación de tesis de maestría de excelencia, y acorde con los parámetros que caracterizan a la institución.
Por último, una vez finalizado el desarrollo, se realizará un registro con los indicadores resultantes, se validará la propuesta, se explicitarán los resultados obtenidos y se elaborarán las conclusiones, a fin de abrir y/o profundizar en nuevas líneas de investigación. 

\subsubsection{Método propuesto}
\noindent Se propone el método EQuAL (\textit{Ensemble method for community Question Answering sites based on cLustering}), que mejora la calidad y eficiencia para recomendar preguntas en un sitio de CQA. Este método está basado en una arquitectura Big Data distribuida y tiene en cuenta diversas distancias de texto, combinadas mediante un método de ensamble de clustering.
\begin{figure}
	\def\svgwidth{\linewidth}
	\input{imagenes/figura2.pdf_tex}
	\caption{Método EQuAL para la generación de matrices de co-asociación desde el conjunto de datos original.}
\end{figure}

\bigskip El desarrollo para este trabajo de tesis está basado en dos pasos, como se muestra en la Figura 2. El primer paso es la generación de un conjunto de particiones. El mismo comenzará aplicando los distintos algoritmos de medidas de similaridad de texto del estado del arte al conjunto de datos de entrada. Este procedimiento tendrá como resultado un número $D$ de matrices de distancias. Por cada matriz de distancias, se aplicarán $N$ corridas de algoritmos de clustering, cada uno con un número $k$ de elementos seleccionados al azar, que es un parámetro de entrada del algoritmo de clustering. Esta combinación de $D$ matrices y $N$ clusters, resultará en $D \times N$ corridas del proceso de clustering en total. Esta configuración obtendrá un conjunto de particiones como resultado, con el fin de resumir la estructura de cada una de las particiones generadas por los algoritmos de clustering. El segundo paso es construir una matriz de co-asociación a partir del conjunto de particiones. Para tal fin, se aplica un algoritmo de ensamble de clustering de acumulación de evidencias, que combinará cada una de estas particiones, dando como salida una matriz de co-asociación, que contiene en cada posición la proporción de veces que los elementos $i,j$ caen juntos en el mismo grupo de la salida de clustering, a lo largo de las $D \times N$ particiones. La matriz de co-asociación, que es una representación integrada de las relaciones subyacentes entre los datos originales, será la entrada para RS en sitios CQA. Además, tiene la característica de ser adimensional, insesgada y comprende toda la variabilidad propia de los algoritmos de clustering, por lo cual, mejora la estructura de distancia item-item que es necesaria como entrada para un RS basado en contenido, incorporando varios aspectos de las distancias entre elementos de texto, en lugar de usar solo una simple medida basada individualmente en aspectos de cada una de las medidas de distancia.

\bigskip El armado de matrices, la combinación de las mismas y la aplicación de estrategias estadísticas, implica un aumento significativo del volumen de datos y requiere una capacidad de cálculo intensiva. Una arquitectura Big Data que realice el procesamiento distribuido de los mismos es fundamental para este proceso. Además del volumen de datos con el cual se trabajará, se variarán distintos parámetros, tales como la medida de similaridad y valores de umbral involucrados en procesos de clustering, con el fin de obtener resultados confiables; lo cual redunda en múltiples ejecuciones de toda la solución. Debe destacarse que, en un primer momento, se implementarán experimentos basados en una infraestructura MapReduce aplicados con frameworks basados en Hadoop y cluster computing, desplegados en servidores elásticos en la nube, lo cual provee la ventaja de procesar grandes cantidades de datos en instancias dinámicamente escalables.

	\begin{landscape}
	\section{Cronograma}\label{ch:cronograma}
	\newganttchartelement*{mymilestone}{
		mymilestone/.style={
			shape=isosceles triangle,
			inner sep=0pt,
			draw=cyan,
			top color=white,
			bottom color=cyan!50
		},
		mymilestone incomplete/.style={
			/pgfgantt/mymilestone,
			draw=yellow,
			bottom color=yellow!50
		},
		mymilestone label font=\slshape,
		mymilestone left shift=0pt,
		mymilestone right shift=0pt,
		title label font=\normalfont,
		bar label font=\normalfont,
		y unit chart=0.65cm,
	}
	
	\newgantttimeslotformat{stardate}{%
		\def\decomposestardate##1.##2\relax{%
			\def\stardateyear{##1}\def\stardateday{##2}%
		}%
		\decomposestardate#1\relax%
		\pgfcalendardatetojulian{\stardateyear-01-01}{#2}%
		\advance#2 by-1\relax%
		\advance#2 by\stardateday\relax%
	}
	
	\begin{ganttchart}[vgrid, hgrid]{1}{26}
		\gantttitle{Junio}{4}
		\gantttitle{Julio}{4}
		\gantttitle{Agosto}{4}
		\gantttitle{Sept.}{4}
		\gantttitle{Oct.}{4}
		\gantttitle{Nov.}{4}
		\gantttitle{Dic.}{2} \\
		\gantttitlelist{1,...,26}{1}\\
		
		\ganttgroup{Evaluación bibliográfica}{1}{24} \\
		\ganttbar{Búsqueda bibliográfica}{1}{24} \\
		\ganttbar{Lectura y análisis bibliográfico}{1}{24} \\

		\ganttgroup{Desarrollar y documentar el proceso propuesto}{3}{21} \\
		\ganttbar{Investigación sobre arquitecturas y buenas prácticas}{3}{4} \\
		\ganttbar{Desarrollo e implementación de la propuesta}{4}{19} \\
		\ganttbar{Documentación del proceso de desarrollo}{11}{19}\\
		\ganttbar{Aplicación de conjuntos de datos de validación}{17}{20}\\
		\ganttbar{Testing, refactoring de código y análisis de performance}{18}{21} \\
		
		\ganttgroup{Obtención y evaluación de resultados}{21}{22} \\
		\ganttbar{Aplicación del conjunto datos en estudio}{21}{22} \\
		\ganttbar{Evaluación de medidas de rendimiento}{21}{22} \\
		
		\ganttgroup{Cierre de redacción de trabajo de tesis}{23}{26} \\
		\ganttbar{Documentación de resultados obtenidos}{23}{24} \\
		\ganttbar{Redacción final y corrección de errores}{24}{26}  \\
	\end{ganttchart}
\end{landscape}

	\newpage \section{Condiciones institucionales para el desarrollo de la tesis. Infraestructura y equipamiento}\label{ch:condiciones}
El presente trabajo se lleva a cabo en el marco del Proyecto PID UTN: Minería de Datos aplicado a problemáticas de Big Data de la Universidad Tecnológica Nacional, Facultad Regional Rosario\footnote{Código del PID: SIUTNRO0005006. Bajo la dirección del Ing. Eduardo Amar.}.

\bigskip El candidato cursó la Maestría en Ingeniería en Sistemas de Información en dicha Facultad y tendrá a disposición el equipamiento e instalaciones del Departamento en Ingeniería en Sistemas de Información.
Con el objetivo de realizar el desarrollo tecnológico de este trabajo de tesis, el candidato utilizará equipos propios así como también los equipos de la universidad, en caso de que sea necesario.
Los conjuntos de datos para la experimentación y validación de la solución propuesta están disponibles libremente en Internet, así como también el software y herramientas necesarias.

	\newpage
	
	\begin{onecolumn}
		\bibliographystyle{flexbib}
		\bibliography{7_bibliografia/bibliografia}
	\end{onecolumn}
\end{document}
