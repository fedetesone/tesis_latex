\chapter*{Fundamentación}\label{ch:fundamentacion}
\addcontentsline{toc}{chapter}{Capitulo 2. Fundamentación}

\section*{}
\addtocounter{section}{1}
\setcounter{subsection}{0}

\subsection{Motivación de la tesis}
 La calidad de un RS tiene una relación directa con los datos de entrada que se han generado para alimentarlo. Con el fin de generar una entrada basada en medidas de similaridad, es necesaria la comparación de preguntas formuladas en sitios de CQA usando técnicas de análisis de texto. Un problema importante inherente al análisis de texto, con el fin de cuantificar relaciones entre distintos fragmentos o documentos, es encontrar la medida apropiada de representación (González y otros, 2017). Algunas medidas de similaridad resultantes de algoritmos de recomendación en análisis de texto, son obtenidas mediante algoritmos puramente sintácticos, léxicos, tales como: Term Frequency \citep{salton5mcgill}, Term Frequency/Inverse Document Frequency \citep{baeza1999modern}, basados en ventanas como FastText \citep{joulin2016fasttext} o Word2Vec \citep{mikolov2013efficient}, o semánticos, como Semantic Distance \citep{li2006sentence}. Los algoritmos puramente sintácticos como Term Frequency y Term Frequency/Inverse Document Frequency tienen conocidos problemas, tales como ser invariantes respecto al orden de las palabras o ser sensibles a stopwords, por lo cual, necesitan un gran trabajo de pre-procesamiento. FastText y Word2Vec están fuertemente afectados en el orden en el cual aparecen las palabras. Adicionalmente, ninguna de estas técnicas tiene en cuenta la semántica de las palabras y sus relaciones, como si lo hace Semantic Distance. Sin embargo, esta última técnica, según el trabajo tomado como estado del arte, tampoco alcanza medidas de rendimiento apropiadas para un RS en un sitio de CQA, como por ejemplo un buen desempeño debido a su complejidad inherente.

\bigskip Resultados experimentales de medidas de rendimiento obtenidas en el trabajo que se toma como punto de partida de esta tesis, arrojan entre un 66\% y un 68\% de exactitud, y entre un 32\% y un 33.5\% de error usando cada uno de los algoritmos de recomendación descritos anteriormente. Estos valores son considerados prometedores, ya que las medidas de rendimiento son consistentes en todos los algoritmos seleccionados, lo que denota que la complejidad inherente del conjunto de datos no afecta significativamente el rendimiento de cada uno de ellos. Además, los resultados de prueba no varían significativamente con respecto a los resultados de validación. Dicho esto, la motivación de este trabajo de tesis, es la creación de un método novedoso que combine medidas de similaridad existentes que pueda aplicarse como entrada para un RS, para ser implementado en sitios de CQA. Adicionalmente, se propone una arquitectura de software que soporte el procesamiento del método propuesto de una forma eficiente y escalable. Para tal fin, se crearon matrices de distancias (o similaridad), usando cada una de las preguntas del conjunto de datos en estudio, para luego combinarlas usando métodos de ensamble de clustering, ya que, como existen cientos algoritmos de clustering, es difícil identificar un solo algoritmo que pueda manejar todos los tipos de forma y tamaños de cluster, e incluso, decidir qué algoritmo sería el mejor para un conjunto de datos en particular. \cite{fred2005combining} introducen el concepto de Clustering de Acumulación de Evidencias (EAC), que mapea las particiones de datos individuales en un ensamble de clustering dentro de una nueva medida de similaridad entre patrones, sumarizando la estructura entre-patrón percibido de esos clusters. La partición de datos final es obtenida aplicando el método \textit{single-linkage} a la nueva matriz de similaridad. El resultado de este método muestra que, la combinación de algoritmos de clustering “débiles” como el \textit{k-means}, pueden conducir a la identificación de clusters subyacentes verdaderos con formas, tamaños y densidades arbitrarias. Por lo cual, teniendo en cuenta diferentes particiones creadas con el método de ensamble desde los mismos datos originales, objetos de textos similares probablemente pertenecerán al mismo cluster.

\bigskip El desarrollo de matrices de similaridad para la aplicación del EAC que se utilizarán como entrada de RS, claramente implica manipular un gran volumen de datos complejos y realizar un elevado número de cálculos en tiempo real, ya que nos estamos refiriendo a conjuntos de datos cuyo tamaño supera la capacidad de las herramientas tradicionales de bases de datos de recopilar, almacenar, gestionar y analizar la información \citep{de2016mineria}. Esto implica, en principio, considerar una \textit{matriz de co-asociación} entre elementos realizando varias series de corridas y aplicación de clustering. Cada una de esas series está basada en una de las medidas de similaridad. El resultado será un valor adimensional e insesgado que puede mejorar la representación para la estructura subyacente de relaciones de texto. El volumen de datos ejemplificado en las secciones anteriores, deja expuesta la necesidad de investigar y desarrollar el tema aquí propuesto con un enfoque distinto al tradicional. Esto implica realizar un muestreo aleatorio de pares de preguntas dentro de una arquitectura que permita generar la mayor cantidad posible de subconjuntos de datos extraídos aleatoriamente. Además, posibilitará que cada uno de ellos sea lo más grande posible para aprovechar toda la \textit{variedad} de los datos. Mientras más se aproveche la variedad de los datos (más subconjuntos de datos y de mayor tamaño), se afectará negativamente en el tiempo de procesamiento, razones por las cuales se hace necesaria una arquitectura e infraestructura preparada para tal desafío, con una velocidad que haga posible obtener resultados en un período de tiempo razonablemente corto. Un enfoque Big Data es imprescindible  para este tipo de procesamiento de datos. No solo si se desea hacer referencia a la gran cantidad y complejidad de los datos, sino también a las herramientas utilizadas para procesarlos y las posibilidades de extraer conocimiento útil a partir del análisis de los mismos. Estos procesos y herramientas son el eje central de la definición de Big Data de la consultora Gartner (2012), la cual hace foco en los procesos para manipular activos de gran volumen y variedad con una gran velocidad. Por lo cual, si bien Big Data se refiere a estos activos, demanda formas innovadoras y efectivas de procesarlos, que habiliten tomas de decisiones y automatización de procesos.

\bigskip Por todos estos motivos, se propone la elaboración de un nuevo método y una arquitectura que lo soporte, que genere una entrada de datos correctamente estructurada para RS y que pueda ser utilizada en sitios de CQA, de una forma eficiente y eficaz.

\subsection{Importancia científico-tecnológica}
Con respecto a los sitios de CQA en particular, la importancia de este trabajo radica tanto en la posibilidad de construir un RS que, desde el punto de vista del usuario, reduzca tanto el tiempo promedio en que se encuentra una respuesta como, a su vez, mejore la experiencia del sitio. En este sentido, en la mayoría de los casos no será necesario escribir múltiples versiones de la misma pregunta y los lectores podrán encontrar rápidamente la respuesta que están buscando. Por otro lado, se podría evitar, si se desea, que se creen preguntas duplicadas, lo que significaría un aumento considerable en la calidad y cantidad de la base de conocimiento del sitio, construyendo una relación biunívoca entre una pregunta y su correspondiente respuesta. Además, se logrará optimizar el tamaño de la base de datos, la integridad de la información, mejorar la velocidad en búsquedas e incrementar de la satisfacción y fidelidad del usuario \citep{ricci2011introduction}.

\bigskip Por último, el resultado de la presente investigación también puede ser utilizado para sitios que son fuente de consulta para diversos investigadores dentro del ámbito de la Universidad Tecnológica Nacional, Facultad Regional Rosario, tales como bibliotecas virtuales o foros de consulta para investigaciones científicos-tecnológicos que incluyan I+D+i. Esto permitiría no solo conocer los intereses de otros investigadores y en qué términos formularon sus interrogaciones, sino también conocer quién o quiénes elaboraron las respuestas a dichas preguntas y a qué campo disciplinar pertenecen.


\subsection{Formación de recursos humanos}
El presente trabajo de tesis, en relación a la formación de recursos humanos, tiene los siguientes objetivos:
\begin{itemize}
	\item Capacitar a un grupo de estudiantes de la UTN FRRo, con elementos para la investigación y desarrollo en aplicaciones Big Data.
	\item Realizar grupalmente conocimiento científico, con base teórica sustentable y ejemplos empíricos de aplicaciones funcionales, para presentar en congresos tales como AGRANDA\footnote{AGRANDA: Simposio Argentino de GRANdes DAtos.}, CONAIISI\footnote{CONAIISI: Congreso Nacional de Ingeniería Informática - Sistemas de Información.}, o RecSys\footnote{RecSys: The ACM Conference Series on Recommender Systems.}; o eventos relacionados con Ingeniería en Sistemas de Información.
	\item Elaborar material de estudio relacionado con la temática de la minería de datos para materias de grado y/o posgrado.
	\item Lograr que los estudiantes puedan entender cómo está formado en la actualidad el estado del arte sobre el presente tema y que esto sirva de base para futuras investigaciones en UTN FRRo, ya sean proyectos de investigación, tesis de maestría o de doctorado.
	\item Desarrollar insumos para el armado de cursos tanto de formación académica como abiertos a la comunidad relacionados con Big Data o análisis de texto.
\end{itemize}

\subsection{Importancia socio-económica}
El tema posee una importancia social y económica que permitirá construir contactos y alianzas -económicas, académicas y de naturaleza mixta- con instituciones extranjeras. En otras palabras, a nivel social podría utilizarse para actividades de investigación y en los diferentes niveles educativos, según se adecúen las explicaciones y el vocabulario utilizado, las actividades y los diversos usos. Buscar información en bibliotecas digitales y virtuales, en bases de datos científicos, repositorios digitales, foros especializados de temáticas específicas y diversas o plataformas educativas, son algunos de los sitios donde los RS pueden ser utilizados y aplicados para determinadas actividades cognitivas. Además, el tema puede ser complementado en un futuro con otras líneas de investigación, tales como políticas educativas para la alfabetización mediática, análisis y datos online, fuentes abiertas o la relación entre tecnología y democracia. Estas líneas, de prioridad en la agenda de ciencia y tecnología de países del primer mundo, están siendo desarrolladas entre academia, instituciones de gubernamentales y policy-makers, de manera interdisciplinaria y con el objetivo de mejorar las herramientas que poseen los ciudadanos en relación a la cultura digital y sus mecanismos de funcionamiento estrictamente técnicos y los aspectos culturales que la atraviesan.
Por otro lado, en el marco económico, los resultados de la presente investigación posibilitarán continuar con futuras indagaciones referidas a la temática y diseñar/construir nuevas herramientas de software adecuadas en función de ciertos usos y usuarios específicos. Estas acciones permitirían llevar adelante: nuevos proyectos de investigación interdisciplinarios y con subsidios de naturaleza mixta (público-privada), formación de formadores, pequeños emprendimientos para estudiantes avanzados y/o la postulación a becas de formación (nacionales e internacionales).



